\section{3.B Limits of Integrals \& Integrals of Limits}

\begin{theorem}[3.26]\label{3.26}
    设\((X, \mathcal{S}, \mu)\)是测度空间且\(\mu(X)<\infty\).\(f_1, f_2, \dots: X \to \mathbb{R}\)逐点收敛至\(f\),
    
    且满足\(\forall x \in X, k \in \mathbb{Z}^+, \exists c \in (0,\infty), \abs*{f_k(x)} \leq c\),则\(\lim_{k \to \infty} \int f_k d\mu=\int f d\mu\).
\end{theorem}

\begin{proof}
    根据\textit{egorov}定理,\(\forall \varepsilon>0, \exists E \in \mathcal{S}, \mu(X \setminus E)<\varepsilon/4c\)且\(f_1, f_2, \dots\)在\(E\)上一致收敛至\(f\).
    \begin{align*}
        \abs*{\int f_k d\mu-\int f d\mu}&=\abs*{\int_{X \setminus E} f_k d\mu-\int_{X \setminus E} f d\mu+\int_E (f_k-f) d\mu} \\
        &\leq \abs*{\int_{X \setminus E} f_k d\mu}+\abs*{\int_{X \setminus E} f d\mu}+\abs*{\int_E (f_k-f) d\mu} \\
        &<2c \mu(X \setminus E)+\sup_E \abs*{f_k-f} \mu(E)<\varepsilon/2+\sup_E \abs*{f_k-f} \mu(E)
    \end{align*}
    由于\(f_1, f_2, \dots\)在\(E\)上一致收敛至\(f\),故\(\forall \varepsilon>0, \exists k \in \mathbb{Z}^+, \forall m \geq k, \sup_E \abs*{f_m-f}<\varepsilon/2\mu(X)\).

    于是\(\exists m \in \mathbb{Z}^+, \forall k \geq m, \sup_E \abs*{f_k-f} \mu(E)<\mu(E) \varepsilon/2\mu(X)<\varepsilon/2\),证毕.

    {\kaishu 在有限测度上可以直接考虑使用\textit{egorov}定理,分为一致收敛部分和小集合部分.
    
    对一致收敛部分控制函数差值的上确界,对小集合控制其测度即可.}
\end{proof}

\begin{theorem}[3.28]\label{3.28}
    设\((X, \mathcal{S}, \mu)\)是测度空间,\(g: X \to [0,\infty]\)是\(\mathcal{S}-\)可测函数且\(\int g d\mu<\infty\).

    则\(\forall \varepsilon>0, \exists \delta>0, \forall B \in \mathcal{S}, \mu(B)<\delta, \int_B g d\mu<\varepsilon\).
\end{theorem}

\begin{proof}
    设简单\(\mathcal{S}-\)可测函数\(0 \leq h \leq g\)满足\(\int g d\mu-\int h d\mu<\varepsilon/2\),定义\(H=\max_X h\).

    令\(\delta<\varepsilon/2H\),考虑\(\forall B \in \mathcal{S}, \mu(B)<\delta\).
    \begin{align*}
        \int_B g d\mu=\int_B (g-h) d\mu+\int_B h d\mu<\int (g-h) d\mu+\int_B h d\mu<\varepsilon/2+\mu(B)H<\varepsilon/2+\varepsilon/2=\varepsilon
    \end{align*}
    {\kaishu 这是控制收敛定理的一个引理,旨在通过简单函数控制小集合上的积分行为.}
\end{proof}

\begin{theorem}[3.29]\label{3.29}
    设\((X, \mathcal{S}, \mu)\)是测度空间,\(g: X \to [0,\infty]\)是\(\mathcal{S}-\)可测函数且\(\int g d\mu<\infty\).

    则\(\forall \varepsilon>0, \exists E \in \mathcal{S}, \mu(E)<\infty, \int_{X \setminus E} g d\mu<\varepsilon\).
\end{theorem}

\begin{proof}
    设简单\(\mathcal{S}-\)可测函数\(h=\sum_{k=1}^m a_k \chi_{A_k} \leq g\)满足\(\int g d\mu-\int h d\mu<\varepsilon\)且\(a_1, \dots, a_m>0\).

    显然\(\mu(A_k)<\infty\),否则\(\int h d\mu \geq a_k \mu(A_k)=\infty>\int g d\mu\).令\(E=\bigcup_{k=1}^m A_k\).
    \begin{align*}
        \int_{X \setminus E} g d\mu=\int g d\mu-\int \chi_E g d\mu<\varepsilon+\int h d\mu-\int \chi_E h d\mu
    \end{align*}
    由于\(\chi_E h=h\),故\(\int_{X \setminus E} g d\mu<\varepsilon+\int h d\mu-\int \chi_E h d\mu=\varepsilon\),证毕.

    {\kaishu 这是控制收敛定理的一个引理,旨在将可积函数的积分控制在有限区域以内.}
\end{proof}

\newpage

\begin{theorem}[3.31]\label{3.31}
    设\((X, \mathcal{S}, \mu)\)是测度空间.\(f_1, f_2, \dots: X \to [-\infty,\infty]\)逐点收敛至\(f\),
    
    且满足\(\forall x \in X, k \in \mathbb{Z}^+\),存在\(\mathcal{S}-\)可测函数\(g: X \to [0,\infty]\)使得\(\abs*{f_k(x)} \leq g(x), \int g d\mu<\infty\),
    
    则\(\lim_{k \to \infty} \int f_k d\mu=\int f d\mu\).
\end{theorem}

\begin{proof}
    设\(E \in \mathcal{S}\)且存在\(\mathcal{S}-\)可测函数\(g\)满足条件.
    \begin{align*}
        \abs*{\int f_k d\mu-\int f d\mu}&=\abs*{\int_{X \setminus E} f_k d\mu-\int_{X \setminus E} f d\mu+\int_E f_k d\mu-\int_E f d\mu} \\
        &\leq \abs*{\int_{X \setminus E} f_k d\mu}+\abs*{\int_{X \setminus E} f d\mu}+\abs*{\int_E f_k d\mu-\int_E f d\mu} \\
        &\leq 2\int_{X \setminus E} g d\mu+\abs*{\int_E f_k d\mu-\int_E f d\mu}
    \end{align*}
    考虑\(\mu(X)<\infty\)的情况.根据\cref{3.28},\(\forall \varepsilon>0, \exists \delta>0, \forall B \in \mathcal{S}, \mu(B)<\delta, \int_B g d\mu<\varepsilon/4\).

    同时根据\textit{egorov}定理\(\forall \delta>0, \exists E \in \mathcal{S}, \mu(X \setminus E)<\delta\)且\(f_1, f_2, \dots\)在\(E\)上一致收敛至\(f\).

    那么\(2\int_{X \setminus E} g d\mu<2\varepsilon/4=\varepsilon/2\),同时\(\abs*{\int_E (f_k-f) d\mu} \leq \mu(E) \sup_E\abs*{f_k-f}\).

    由于\(f_1, f_2, \dots\)在\(E\)上一致收敛,故\(\forall \varepsilon>0, \exists m \in \mathbb{Z}^+, \forall k \geq m, \sup_E \abs*{f_k-f}<\varepsilon/2\mu(X)\).

    于是\(\exists m \in \mathbb{Z}^+, \forall k \geq m, \sup_E \abs*{f_k-f} \mu(E)<\mu(E) \varepsilon/2\mu(X)<\varepsilon/2\),证毕.

    考虑\(\mu(X)=\infty\)的情况.根据\cref{3.29},\(\forall \varepsilon>0, \exists E \in \mathcal{S}, \mu(E)<\infty\)且\(\int_{X \setminus E} g d\mu<\varepsilon/4\).
    \begin{align*}
        \abs*{\int f_k d\mu-\int f d\mu}<\frac{\varepsilon}{2}+\abs*{\int_E f_k d\mu-\int_E f d\mu}
    \end{align*}
    由于\(\mu(E)<\infty\),按照{\kaishu 情况一}的方法.对于充分大的\(k\),有\(\abs*{\int_E f_k d\mu-\int_E f d\mu}<\varepsilon/2\).
\end{proof}

\begin{problem}[1]\label{3.B.1}
    构造函数序列\(\{f_k\}_{k \in \mathbb{Z}^+}: \mathbb{Z}^+ \to [0,\infty)\)满足\(\forall m \in \mathbb{Z}^+, \lim_{k \to \infty}f_k(m)=0\).

    然而当\(\mu\)为计数测度时,\(\lim_{k \to \infty} \int f_k d\mu=1\).
\end{problem}

\begin{proof}
    定义\(f_k(m)=\chi_{\{k\}}(m)\),显然\(\forall m \in \mathbb{Z}^+, \lim_{k \to \infty}f_k(m)=0\)且\(\forall k \in \mathbb{Z}^+, \int f_k d\mu=1\).
\end{proof}

\begin{problem}[2]\label{3.B.2}
    构造连续函数序列\(\{f_k\}_{k \in \mathbb{Z}^+}: \mathbb{R} \to [0,1]\)满足\(\forall x \in \mathbb{Z}^+, \lim_{k \to \infty}f_k(x)=0\).

    然而当\(\lambda\)为lebesgue测度时,\(\lim_{k \to \infty} \int f_k d\lambda=\infty\).
\end{problem}

\begin{proof}
    对于任意\(x \in \mathbb{R}\),定义\(f_k(x)\)为
    \begin{align*}
        f_k(x)=
        \begin{cases}
            \abs*{\sin(2\pi x)}, x \geq k \\
            0, x<k
        \end{cases}
    \end{align*}
    显然\(f_k\)在\(x=k\)处连续,且满足\(\forall x \in \mathbb{Z}^+, \lim_{k \to \infty}f_k(x)=0\)和\(\lim_{k \to \infty} \int f_k d\lambda=\infty\).
\end{proof}

\newpage

\begin{problem}[3]\label{3.B.3}
    设\(\lambda\)是\(\mathbb{R}\)上的lebesgue测度,\(f: \mathbb{R} \to \mathbb{R}\)是borel可测函数且\(\int \abs*{f} d\lambda<\infty\).

    定义\(g: \mathbb{R} \to \mathbb{R}\)为\(g(x)=\int_{(-\infty,x)} f d\lambda\),证明:\(g\)在\(\mathbb{R}\)上一致连续.
\end{problem}

\begin{proof}
    根据\cref{3.28},对于\(\forall \varepsilon>0, \exists \delta>0, \forall \lambda(B)<\delta, \int_B f d\lambda<\varepsilon\).
    \begin{align*}
        \abs*{g(x_1)-g(x_2)}=\abs*{\int_{(-\infty,x_1)} f d\lambda-\int_{(-\infty,x_2)} f d\lambda}=\abs*{\int_{(x_1,x_2)} f d\mu}<\varepsilon
    \end{align*}
    因此只要\(\abs*{x_1-x_2}<\delta\),就有\(\abs*{g(x_1)-g(x_2)}<\varepsilon\).
\end{proof}

\begin{problem}[4]\label{3.B.4}
    a.设\((X, \mathcal{S}, \mu)\)是测度空间且\(\mu(X)<\infty\).设\(f: X \to [0,\infty)\)是有界\(\mathcal{S}-\)可测函数.

    证明:\(\int f d\mu=\inf\{\sum_{k=1}^m \mu(A_k)\sup_{A_k}f: A_1, \dots, A_m \text{是} X \text{的} \mathcal{S}\text{-划分}\}\).

    b.证明在有界条件被替换为\(\int f d\mu<\infty\)后结论不成立.

    c.证明去掉\(\mu(X)<\infty\)后结论不成立.
\end{problem}

\begin{proof}[证明a]
    令上和集合的元素为\(U(P)\),\(P\)是\(X\)的\(\mathcal{S}\)-划分,则下确界为\(\inf_P U(P)\).

    设\(M\)是\(f\)的一个上界,令\(I_{m,k}=(M(k-1)/m, Mk/m], k=1, \dots, m\).

    令\(A_{m,k}=f^{-1}(I_{m,k})\),显然\(A_{m,1}, \dots, A_{m,m}\)是\(X\)的\(\mathcal{S}\)-划分,记为\(P_m\).

    定义简单函数\(f_m(x)=\sum_{k=1}^m Mk/m \chi_{A_{m,k}}\),因此\(0 \leq f_m(x)-f(x) \leq M/m\).

    有\(\int f_m d\mu=U(P_m)<\infty\)且\(\lim_{m \to \infty} f_m(x)=f(x)\),得到\(\inf_P U(P) \leq \lim_{k \to \infty} \int f_m d\mu\).

    同时由于\(\int f_1 d\mu<\infty\),根据\cref{3.A.20}有\(\lim_{m \to \infty} \int f_m d\mu=\int f d\mu\).

    由于\(\int f d\mu \leq \inf_P U(P)\),故\(\int f d\mu \leq \inf_P U(P) \leq \lim_{k \to \infty} \int f_m d\mu\),证毕.

    {\kaishu 证明的关键在于从上和集合中抽取一个递减的收敛子列.}
\end{proof}

\begin{proof}[证明b]
    令\(X=[0,1], \mathcal{S}, \mu=\lambda\)是限制在\([0,1]\)上的\textit{borel}代数和\textit{lebesgue}测度.

    令\(f(x)=1/\sqrt{x}\),显然\(\int_0^1 f d\lambda=[2\sqrt{x}]_0^1=2<\infty\),但\(\forall P, \inf_P U(P)=U(P)=\infty\).
\end{proof}

\begin{proof}[证明c]
    令\(X=[0,\infty), \mathcal{S}, \mu=\lambda\)是限制在\([0,1]\)上的\textit{borel}代数和\textit{lebesgue}测度.

    令\(f(x)=e^{-x}\),显然\(\int_0^\infty f d\lambda=[-e^{-x}]_0^\infty=1<\infty\).

    但是\(\forall P, \exists A_P \in \mathbb{S}, \lambda(A_P)=\infty\),于是\(\inf_P U(P)=U(P)=\infty\).
\end{proof}

\begin{problem}[5]\label{3.B.5}
    设\(f: \mathbb{R} \to \mathbb{R}\)是borel可测函数且\(\int \abs*{f} d\lambda<\infty\).证明\(\lim_{k \to \infty} \int_{[-k,k]} f d\mu=\int f d\mu\).
\end{problem}

\begin{proof}
    令\(f_k(x)=\chi_{[-k,k]} f(x)\),显然\(\lim_{k \to \infty}f_k(x)=f(x)\)且\(\forall k \in \mathbb{Z}^+, x \in X, f_k(x) \leq f(x)\).

    由\(\int \abs*{f} d\lambda<\infty\),根据控制收敛定理\(\lim_{k \to \infty} \int_{[-k,k]} f d\lambda=\lim_{k \to \infty} \int f_k d\lambda=\int f d\lambda\).
\end{proof}

\newpage

\begin{problem}[8]\label{3.B.8}
    证明\(\{x \in [a,b]: f^L(x) \ne f^U(x)\}=\{x \in [a,b]: f \text{在} x \text{处不连续}\}\).
\end{problem}

\begin{proof}
    显然\(\abs*{f^L(x)-f^U(x)}>0\)等价于\(\lim_{n \to \infty}(\sup_{J_n(x)} f-\inf_{J_n(x)} f)>0\).

    若\(x=a+t(b-a)/2^n, t=0, \dots, 2^n\),则\(J_n(x)=I_{t,2^n} \cup I_{t+1,2^n}\),否则\(J_n(x)=I_{t,2^n}\).

    因此\(\exists \varepsilon>0, k \in \mathbb{Z}^+, (\lim_{n \to \infty} \sup_{J_n(x)} f-\inf_{J_n(x)} f) \geq \varepsilon>1/k>0\).

    因此总存在\(\delta>0, B(x,\delta) \subseteq J_n(x)\).定义\(G_k\),则\(x \notin G_k\).

    由于\(x \notin \bigcap_{k=1}^\infty G_k\)等价于\(f\)在\(x\)处不连续,结合\(x \notin G_k\)证毕.
\end{proof}

\begin{problem}[10]\label{3.B.10}
    a.设\((X, \mathcal{S}, \mu)\)是测度空间且\(\mu(X)<\infty\).设\(f: X \to [0,\infty)\)是有界\(\mathcal{S}-\)可测函数.

    证明:若\(0<p<r\)且\(\int f^r d\mu<\infty\),则\(\int f^p d\mu<\infty\).

    b.证明若无\(\mu(X)<\infty\)条件结论不成立.
\end{problem}

\begin{proof}[证明a]
    令\(A=f^{-1}((1,\infty))\),则\(\forall x \in A, f^p(x)<f^r(x)\),因此\(\int_A f^p d\mu<\int_A f^r d\mu<\infty\).

    而\(\int_{X \setminus A} f^p d\mu \leq \int \chi_{X \setminus A} d\mu=\mu(X \setminus A) \leq \mu(X)<\infty\),故\(\int f^p d\mu<\int_A f^r d\mu+\int \chi_{X \setminus A}<\infty\).
\end{proof}

\begin{proof}[证明b]
    设\(X=[1,\infty), \mu=\lambda\),令\(f(x)=1/x, r=2, p=1\).

    于是\(\int 1/x^2 d\lambda=\pi^2/6, \int 1/x d\lambda=[\ln x]_1^\infty=\infty\),证毕.

    {\kaishu 精髓在于那些值域小于\(1\)的函数不能再依赖有限测度控制,使用了调和级数和\(p\)级数.}
\end{proof}

\begin{problem}[11]\label{3.B.11}
    设\((X, \mathcal{S}, \mu)\)是测度空间且\(f \in \mathcal{L}^1(\mu)\).
    
    证明:\(\{x \in X: f(x) \ne 0\}\)由可数的有限测度集组成.
\end{problem}

\begin{proof}
    令\(A_n=\{x \in X: \abs*{f(x)}>1/n\}\),于是\(\{x \in X: f(x) \ne 0\}=\bigcup_{n=1}^\infty A_n\).

    由于\(\forall n \in \mathbb{Z}^+, \int_{A_n} \abs*{f} d\mu<\int \abs*{f} d\mu<\infty\),因此\(\forall n \in \mathbb{Z}^+, \mu(A_n)<\infty\),证毕.
\end{proof}

\begin{problem}[12]\label{3.B.12}
    令\(f_k(x)=\dfrac{(1-x)^k \cos(x/k)}{\sqrt{x}}\),证明\(\lim_{k \to \infty} \int_0^1 f_k=0\).
\end{problem}

\begin{proof}
    由于\(x \in (0,1)\),故\(\lim_{k \to \infty} (1-x)^k=0, \lim_{k \to \infty} \cos(x/k)=1, \lim_{k \to \infty} f_k(x)=0\).

    另一方面,由于\(\forall x \in (0,1), (1-x)^k \cos(x/k)<1\),故令\(g(x)=1/\sqrt{x}, f_k(x) \leq g(x)\).

    由于\(\int_0^1 g=[2\sqrt{x}]_0^1=2<\infty\),故根据控制收敛定理\(\lim_{k \to \infty} \int_0^1 f_k=\int_0^1 \lim_{k \to \infty} f_k=0\).
\end{proof}

\begin{problem}[13]\label{3.B.13}
    构造borel可测函数\(\{f_t\}_{t \in \mathbb{Z}^+}: [0,1] \to \mathbb{R}\),
    
    满足\(\limsup_{k \to \infty} f_k(x)=\infty\)且\(\lim_{k \to \infty} \int f=0\).
\end{problem}

\begin{proof}
    令\(g_{n,k}=n \chi_{[(k-1)/2^n, k/2^n]}, k=1, \dots, 2^n\).将\(g_{n,k}\)按照{\kaishu 字典顺序}排序得到\(\{f_k\}_{k \in \mathbb{Z}^+}\).

    由于\(\forall x \in [0,1], \exists k_1, k_2, \dots \in \mathbb{Z}^+, f_{k_p}(x)=p\),故\(\limsup_{k \to \infty} f_k(x)=\infty\).
    
    显然\(\lim_{k \to \infty} \int f_k d\mu=\lim_{k \to \infty} k/2^k=0\),则证毕.

    {\kaishu 本题使用“有丝分裂”的形式构造分层函数,意在指出\(\mathcal{L}^1\)收敛和点态收敛的无关性.}
\end{proof}

\newpage

\begin{problem}[16]\label{3.B.16}
    设\(f \in \mathcal{L}^1(\mathbb{R})\).

    a.定义\(\forall t \in \mathbb{R}, f_t(x)=f(x-t)\).证明:\(\lim_{t \to 0} \norm*{f-f_t}_1=0\).

    b.定义\(\forall t \in \mathbb{R}, f_t(x)=f(tx)\).证明:\(\lim_{t \to 1} \norm*{f-f_t}_1=0\).
\end{problem}

\begin{proof}[证明a]
    根据定理3.48\(\forall \varepsilon>0, f \in \mathcal{L}^1(\mathbb{R}), \exists g \in \mathcal{L}^1(\mathbb{R}), \norm*{f-g}_1<\varepsilon/3\),其中\(g\)是连续函数.

    考虑\(\norm*{g-g_t}_1\),令\(K=\{x \in X: g(x) \ne 0\}\),则\(K+t=\{x \in X: g_t(x) \ne 0\}\).于是
    \begin{align*}
        \forall \varepsilon>0, \exists t \in \mathbb{R}, \norm*{g-g_t}_1=\int_{K \cup (K+t)} \abs*{g-g_t} \leq 2\abs*{K} \sup_{K \cup (K+t)}(g-g_t)<\varepsilon/3
    \end{align*}
    由于\(\norm*{f_t-g_t}_1=\norm*{f-g}_1<\varepsilon/3\),因此
    \begin{align*}
        \norm*{f-f_t}_1 \leq \norm*{f-g}_1+\norm*{g-g_t}_1+\norm*{f_t-g_t}_1<\varepsilon/3+\varepsilon/3+\varepsilon/3=\varepsilon
    \end{align*}
    {\kaishu \textit{b}部分的证明类似,将\(K+t\)改为\(1/t K\)即可.}
\end{proof}