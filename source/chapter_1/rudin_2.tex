\section{Basic Topology}

{\kaishu 以下本章节将按照度量与完备化,开、闭、连通集,相对拓扑与环绕空间,稠密性与可分度量空间,紧集和完全集的顺序组织习题,以突出增强集合控制的主线.}

{\kaishu 下面是度量与完备化的习题.}

\begin{theorem}[0.1]
    设\((X,d)\)是度量空间,按照下述步骤将该\((X,d)\)完备化.
    \begin{enumerate}
        \item 对\(X\)中的基本序列\(\{f_n\},\{g_n\}\)定义\(\{f_n\} \sim \{g_n\}\)为\(\lim_{n \to \infty} d(f_n,g_n)=0\).
        \item 定义\([\{f_n\}]=\{g_n: f_n \sim g_n\}, \widehat{X}=\{[\{f_n\}]: \{f_n\}\text{是} X \text{中的基本序列}\}\).
        \item 定义\(\widehat{d}: \widehat{X} \times \widehat{X} \to [0,\infty)\)为\(\widehat{d}([\{f_n\}],[\{g_n\}])=\lim_{n \to \infty} d(f_n,g_n)\),验证其良定义性.
        \item 设\(\varphi: X \to \widehat{X}\)为\(\varphi(f)=\{f, f, \dots\}\),验证\(\widehat{d}(\varphi(f),\varphi(g))=d(f,g)\).
        \item 验证\((\widehat{X}, \widehat{d})\)是完备度量空间.
    \end{enumerate}
\end{theorem}

\begin{proof}[证明3]
    由三角不等式有\(\forall m,n \in \mathbb{Z}^+, \abs*{d(f_m,g_m)-d(f_n,g_n)} \leq d(f_m,f_n)+d(g_m,g_n)\).

    由于\(\forall \varepsilon>0, \exists N_1,N_2 \in \mathbb{Z}^+, \forall m,n>N_1, d(f_m,f_n)<\varepsilon/2; \forall m,n>N_2, d(g_m,g_n)<\varepsilon/2\),

    因此\(\forall \varepsilon>0, \exists N=\max\{N_1,N_2\}, \forall m,n>N, \abs*{d(f_m,g_m)-d(f_n,g_n)}<\varepsilon/2+\varepsilon/2=\varepsilon\).

    因而\(\{d(f_n,g_n)\}\)是实数基本序列,根据完备性定理\(\lim_{n \to \infty} d(f_n,g_n)\)存在.

    下证良定义性.设\(\{f_n^1\} \sim \{f_n^2\}\),则\(\abs*{d(f_n^1,g_n)-d(f_n^2,g_n)} \leq d(f_n^1,f_n^2)\).

    因此\(\lim_{n \to \infty} \abs*{d(f_n^1,g_n)-d(f_n^2,g_n)} \leq \lim_{n \to \infty} d(f_n^1,f_n^2)=0, \lim_{n \to \infty} d(f_n^1,g_n)-d(f_n^2,g_n)=0\).

    这即\(\widehat{d}([\{f_n^1\}],[\{g_n\}])=\lim_{n \to \infty} d(f_n^1,g_n)=\lim_{n \to \infty} d(f_n^2,g_n)=\widehat{d}([\{f_n^2\}],[\{g_n\}])\).
\end{proof}

\begin{proof}[证明5]
    设\([\{f_n^1\}], [\{f_n^2\}], \dots\)是\(\widehat{X}\)中的基本序列,下证\(\lim_{k \to \infty} [\{f_n^k\}]\)存在.

    \(\forall k \in \mathbb{Z}^+, \lim_{n \to \infty} \varphi(f_n^k)=[\{f_n^k\}]\),故\(\exists N_k \in \mathbb{Z}^+, \widehat{d}(\varphi(f_{N_k}^k), [\{f_n^k\}])<1/k\).

    由\([\{f_n^1\}], [\{f_n^2\}], \dots\)为基本序列,\(\forall \varepsilon>0, \exists M_1 \in \mathbb{Z}^+, \forall p,q>M_1, d([\{f_n^p\}],[\{f_n^q\}])<\varphi(\varepsilon/3)\).

    令\(c_n=f_{N_n}^n\).下证\(\{c_n\}\)是基本序列且\(\lim_{k \to \infty} [\{f_n^k\}]=[\{c_n\}]\).取\(\forall p,q>\max\{M_1,\left\lceil 3/\varepsilon \right\rceil\}\),
    \begin{align*}
        \widehat{d}(\varphi(c_p),\varphi(c_q)) &\leq \widehat{d}(\varphi(c_p), [\{f_n^p\}])+\widehat{d}([\{f_n^p\}], [\{f_n^q\}])+\widehat{d}(\varphi(c_q), [\{f_n^q\}]) \\
        &<\varphi(\varepsilon/3)+\varphi(\varepsilon/3)+\varphi(\varepsilon/3)=\varphi(\varepsilon)
    \end{align*}
    因此\(\widehat{d}(\varphi(c_p),\varphi(c_q))=d(c_p,c_q)<\varepsilon\),即\(\{c_n\}\)确为基本序列.
    
    由\(\lim_{n \to \infty} \varphi(c_n)=[\{c_n\}]\),故\(\forall \varepsilon>0, \exists M_2 \in \mathbb{Z}^+, \forall k>M_2, \widehat{d}(\varphi(c_k), [\{c_n\}])<\varphi(\varepsilon/2)\).
    \begin{align*}
        \forall k>\max\{M_2,\left\lceil 2/\varepsilon \right\rceil\}, \widehat{d}([\{f_n^k\}], [\{c_n\}]) &\leq \widehat{d}([\{f_n^k\}], \varphi(c_k))+\widehat{d}(\varphi(c_k), [\{c_n\}]) \\
        &<\varphi(\varepsilon/2)+\varphi(\varepsilon/2)=\varphi(\varepsilon)
    \end{align*}
    因此\(\lim_{k \to \infty} [\{f_n^k\}]=[\{c_n\}]\),完备性证毕.
\end{proof}

\newpage

\begin{problem}[2]\label{pb2.2}
    设存在不全为零的整数\(a_0, \dots, a_n\),且复数\(z\)满足\(\sum_{k=0}^n a_n z^{n-k}=0\).

    这样的\(z\)被称为代数数.证明:所有代数数构成可数集.
\end{problem}

\begin{proof}
    由于\(a_0, \dots, a_n\)不全为零,故\(\sum_{k=0}^n \abs*{a_k}>0\).固定次数\(n\),设\(\sum_{k=0}^n \abs*{a_k}=N\).

    对于固定的\(n, N \in \mathbb{Z}^+\),只有有限个\(a_0, \dots, a_n\)的取值可能.记\(p_{n,N}, A_{n,N}\)为
    \begin{align*}
        p_{n,N}=\left\{p(z)=\sum_{k=0}^n a_k z^{n-k}: \sum_{k=0}^n \abs*{a_k}=N \right\},
        A_{n,N}=\bigcup_{p \in p_{n,N}}\left\{z \in \mathbb{C}: p(z)=0\right\}
    \end{align*}
    对于任意的\(p \in p_{n,N}\),最多只有\(n\)个复数\(z\)满足\(p(z)=0\),因而\(p_{n,N}, A_{n,N}\)是有限集.

    于是所有代数数的集合是\(\bigcup_{n=1}^\infty \bigcup_{N=1}^\infty A_{n,N}\),证毕.
\end{proof}

{\kaishu 下面是开集、闭集、联通集相关的习题.

以下\(\mathrm{int} E\)指代\(E\)的内部,\(\mathrm{cl} E\)指代\(E\)的闭包,\(\lim E\)指代\(E\)的导集.}

\begin{problem}[6]\label{pb2.6}
    证明\(\lim E\)是闭集且\(\lim (\mathrm{cl} E)=\lim E, \lim (\lim E) \ne \lim E\).
\end{problem}

\begin{proof}
    设\(\forall p_0 \in \lim (\lim E)\),从而\(\forall \varepsilon>0, \exists p_1 \in \lim E, d(p_0,p_1)<\varepsilon/2\).
    
    由于\(p_1\)是\(E\)的极限点,故\(\forall \varepsilon>0, \exists p_2 \in E, d(p_1,p_2)<\varepsilon/2\).

    因此\(\forall \varepsilon>0, p_0 \in \lim (\lim E), \exists p_1 \in \lim E, p_2 \in E, d(p_0,p_2) \leq d(p_0,p_1)+d(p_1,p_2)<\varepsilon\).
    
    即\(p_0 \in \lim E, \lim (\lim E) \subseteq \lim E\).若\(\lim (\lim E)=\varnothing \subseteq \lim E\),即\(\lim E\)依旧是闭集.

    下证\(\lim (\mathrm{cl} E) \subseteq \lim E\).设\(\forall p_0 \in \lim (\mathrm{cl} E)\),从而\(\forall \varepsilon>0, \exists p_1 \in \mathrm{cl} E, d(p_0,p_1)<\varepsilon/2\).

    由于\(p_1 \in \mathrm{cl} E\),故\(\forall \varepsilon>0, \exists p_2 \in E, d(p_1,p_2)<\varepsilon/2\).

    因此\(\forall \varepsilon>0, p_0 \in \lim (\mathrm{cl} E), \exists p_1 \in \mathrm{cl} E, p_2 \in E, d(p_0,p_2) \leq d(p_0,p_1)+d(p_1,p_2)<\varepsilon\).

    即\(p_0 \in \lim E, \lim (\mathrm{cl} E) \subseteq \lim E\).结合显然的\(\lim E \subseteq \lim (\mathrm{cl} E)\),有\(\lim (\mathrm{cl} E)=\lim E\).

    设\(E=\{0\} \cup \{1/n: n \in \mathbb{Z}^+\}\),则\(\lim E=\{0\}, \lim (\lim E)=\varnothing, \lim E \ne \lim (\lim E)\).
\end{proof}

\begin{problem}[7]\label{pb2.7}
    设\((X,d)\)是度量空间,\(A_1, A_2, \dots \subseteq X\).
    
    证明若\(A=\bigcup_{k=1}^n A_k\),则\(\mathrm{cl} A=\bigcup_{k=1}^n \mathrm{cl} A_k\);若\(A=\bigcup_{k=1}^\infty A_k\),则\(\mathrm{cl} A \supset \bigcup_{k=1}^\infty \mathrm{cl} A_k\).
\end{problem}

\begin{proof}
    由于\(\forall k \in \mathbb{Z}^+, A_k \subseteq A\),因此\(\mathrm{cl} A_k \subseteq \mathrm{cl} A\),即\(\forall n \in \mathbb{Z}^+, \bigcup_{k=1}^n \mathrm{cl} A_k \subseteq \mathrm{cl} A\).

    下证\(\mathrm{cl} A \subseteq \bigcup_{k=1}^n \mathrm{cl} A_k\).设\(\forall p_0 \in \mathrm{cl} A\),则\(\forall \varepsilon>0, \exists p_1 \in A, d(p_0,p_1)<\varepsilon\).

    由于\(A=\bigcup_{k=1}^n A_k\),故\(\exists k=1, \dots, n, \forall \varepsilon>0, \exists p_1 \in A_k, d(p_0,p_1)<\varepsilon\).

    这即\(p_0 \in \mathrm{cl} A_k, \mathrm{cl} A \subseteq \bigcup_{k=1}^n \mathrm{cl} A_k\),因此\(\mathrm{cl} A=\bigcup_{k=1}^n \mathrm{cl} A_k\).

    考虑\(A_k=\{1/k\}\),从而\(\mathrm{cl} A=\{1/n: n \in \mathbb{Z}^+\} \subset \{0\} \cup \{1/n: n \in \mathbb{Z}^+\}=\bigcup_{k=1}^\infty \mathrm{cl} A_k\).
\end{proof}

\newpage

\begin{problem}[8]\label{pb2.8}
    a.证明\(\mathrm{int} E\)是开集且\(\mathrm{int} E=E\)等价于\(E\)是开集.

    b.证明若开集\(G \subseteq E\),则\(G \subseteq \mathrm{int} E\);证明\((\mathrm{int} E)^c=\mathrm{cl} (E^c)\).

    c.证明\(\mathrm{int} E \ne \mathrm{int} (\mathrm{cl} E); \mathrm{cl} E \ne \mathrm{cl} (\mathrm{int} E)\).
\end{problem}

\begin{proof}[证明a]
    若\(p_0 \in \mathrm{int} E\),则\(\exists r_0>0, B(p_0,r_0) \subseteq E\),考虑\(\forall p_1 \in B(p_0,r_0)\).

    显然\(\exists r_1 \in (0, r_0-d(p_0,p_1)), \forall p_2 \in B(p_1,r_1), p_2 \in B(p_0,r_0)\),从而\(p_2 \in \mathrm{int} E\)是\(\mathrm{int} E\)的内点.

    若\(E\)为开集显然有\(\mathrm{int} E=E\).若\(\mathrm{int} E=E\),则\(E\)的所有点都是其内点,即\(E\)是开集.
\end{proof}

\begin{proof}[证明b]
    \(\forall p \in G, \exists r>0, B(p,r) \subseteq G \subseteq E\),则\(p\)是\(E\)的内点,即\(p \in \mathrm{int} E, G \subseteq \mathrm{int} E\).

    下证\((\mathrm{int} E)^c=\mathrm{cl} (E^c)\).\(p_0 \in \mathrm{cl} (E^c)\)等价于\(\forall \varepsilon>0, \exists p_1 \notin E, d(p_0,p_1)<\varepsilon\).

    而这等价于\(\forall r>0, \exists p_1 \notin E, p_1 \in B(p_0,r)\),即\(p_0\)不是\(E\)的内点,等价于\(p_0 \notin \mathrm{int} E\).
\end{proof}

\begin{proof}[证明c]
    令\(X=\mathbb{R}, E=\mathbb{Q}\),则\(\mathrm{int} E=\varnothing \ne X=\mathrm{int} (\mathrm{cl} E); \mathrm{cl} E=X \ne \varnothing=\mathrm{cl} (\mathrm{int} E)\).
\end{proof}

\begin{problem}[19]\label{pb2.19}
    a.设\(A,B\)是度量空间\(X\)的不交闭集,证明它们是分离的.

    b.设\(A,B\)是度量空间\(X\)的不交开集,证明它们是分离的.

    c.固定\(p \in X, r>0\).令\(A=\{q \in X: d(p,q)<r\}, B=\{q \in X: d(p,q)>r\}\).证明它们分离.

    d.证明至少含有两个点的联通度量空间是不可数的.
\end{problem}

\begin{proof}[证明a]
    显然\(\mathrm{cl} A \cap B=A \cap \mathrm{cl} B=A \cap B=\varnothing\),即\(A,B\)是分离的.
\end{proof}

\begin{proof}[证明b]
    若\(\mathrm{cl} A \cap B \ne \varnothing\),即\(\exists p_0 \in B, \forall \varepsilon>0, \exists p_1 \in A, d(p_0,p_1)<\varepsilon\).然而\(p_0\)是\(B\)的内点,
    
    即\(\exists r>0, B(p_0,r) \subseteq B\).取\(\varepsilon<r, \exists p_1 \in A \cap B(p_0,r) \subseteq A \cap B\),与\(A \cap B=\varnothing\)矛盾.
\end{proof}

\begin{proof}[证明c]
    \(A=B(p,r)\)是开集.考虑\(\forall p_1, d(p,p_1)>r\).令\(r_1=(d(p,p_1)-r)/2\),考虑\(B(p_1,r_1)\).

    \(\forall q \in B(p_1,r_1), d(p,q) \geq \abs*{d(p,p_1)-d(p_1,q)} \geq d(p,p_1)-r_1=(d(p,p_1)+r)/2>r\).

    因此\(\forall p_1 \in B, \exists r_1>0, B(p_1,r_1) \subseteq B\),即\(B\)也是开集.结合\(A \cap B=\varnothing\),故\(A,B\)是分离的.
\end{proof}

\begin{proof}[证明d]
    设存在一个{\kaishu 含可数元素的联通度量空间}\(X\),\(D=\{d(x,y): x,y \in X\}\)是可数集.

    由于\(\mathbb{R}\)是不可数集,故\(\exists r \in (\inf D, \sup D), r \notin D\),固定\(p \in X\).

    令\(A=\{q \in X: d(p,q)<r\}, B=\{q \in X: d(p,q)>r\}\).
    
    由于不存在\(q \in X, d(p,q)=r\),故\(X=A \cup B\),但\(A,B\)是分离集,矛盾.
\end{proof}

\begin{problem}[20]\label{pb2.20}
    证明任意度量空间\(X\)中的联通集\(E\)都保证\(\mathrm{cl} E\)是联通集,但不保证\(\mathrm{int} E\)联通.
\end{problem}

\begin{proof}
    令\(X=\mathbb{R}^2, E=\{(x,y): x \in [0,1], y \in [0,1]\} \cup \{(x,y): x \in [-1,0], y \in [-1,0]\}\).

    那么\(\mathrm{int} E=\{(x,y): x \in (0,1), y \in (0,1)\} \cup \{(x,y): x \in (-1,0), y \in (-1,0)\}\).

    \(\{(x,y): x \in (0,1), y \in (0,1)\}\)和\(\{(x,y): x \in (-1,0), y \in (-1,0)\}\)分离,故\(\mathrm{int} E\)不连通.

    设\(\mathrm{cl} E\)不是联通集,则\(\exists U_1,U_2 \ne \varnothing \subseteq X\)使得\(U_1,U_2\)分离且\(\mathrm{cl} E=U_1 \cup U_2\).

    由于\(E \subseteq \mathrm{cl} E\),故\(E=(E \cap U_1) \cup (E \cap U_2)\).\(E \cap U_1\)和\(E \cap U_2\)是分离的,下证均不为空.

    设\(E \cap U_1=\varnothing\),则\(E \subseteq U_2\).然而\(U_1 \subseteq \mathrm{cl} E \subseteq \mathrm{cl} U_2\),从而\(U_1,U_2\)不是分离的,矛盾.
\end{proof}

\begin{problem}[21]\label{pb2.21}
    设\(A,B\)是\(\mathbb{R}^n\)中的分离集,令\(a \in A, b \in B, t \in \mathbb{R}, p(t)=(1-t)a+tb\).

    设\(A_0=p^{-1}(A), B_0=p^{-1}(B)\).a.证明\(A_0,B_0\)是\(\mathbb{R}\)中的分离集.

    b.证明存在\(t_0 \in (0,1)\)使得\(p(t_0) \notin A \cup B\).

    c.证明\(\mathbb{R}^n\)的凸子集是联通集.
\end{problem}

\begin{proof}[证明a]
    若\(\mathrm{cl} A_0 \cap B_0 \ne \varnothing\),则\(p(\mathrm{cl} A_0 \cap B_0)=\mathrm{cl} A \cap B \ne \varnothing\),与\(A,B\)分离矛盾.
\end{proof}

\begin{proof}[证明b]
    设\(\forall t_0 \in (0,1), p^{-1}(t_0) \in A_0 \cup B_0\),令\(A'=A_0 \cap [0,1], B'=B_0 \cap [0,1]\).

    从而\((0,1)=A' \cup B'\)是两个不空分离集的并,\((0,1)\)不连通,显然矛盾.
\end{proof}

\begin{proof}[证明c]
    设\(a,b \in \mathbb{R}^n\).令\(A=\{p(t): t \in [0,1]\}\).若存在分离集\(V,W\)使得\(A=V \cup W\),

    那么\(p^{-1}(V) \cup p^{-1}(W)=[0,1]\),而\(p^{-1}(V), p^{-1}(W)\)分离,故\([0,1]\)不连通.
\end{proof}

{\kaishu 下面是相对拓扑和环绕空间的习题.}

\begin{theorem}\label{br2.30}
    设\(Y \subseteq X\),则\(E \subseteq Y\)是\(Y\)的开子集等价于存在\(X\)的开子集\(G\)使得\(E=Y \cap G\).
\end{theorem}

\begin{proof}
    若\(E \subseteq Y\)是\(Y\)的开子集,则\(\forall p \in E, \exists r_p>0, \forall q \in Y \cap B(p,r_p), q \in E\).

    令\(G=\bigcup_{p \in E} B(p,r_p)\),从而\(G\)是\(X\)的开子集,下证\(E=Y \cap G\).

    由于\(\forall p \in E\)都有\(p \in B(p,r_p) \subseteq G\)且\(p \in E \subseteq Y\),故\(E \subseteq Y \cap G\).

    由于\(\forall p \in E, B(p,r_p) \cap Y \subseteq E\),即\(\bigcup_{p \in E} (B(p,r_p) \cap Y)=\bigcup_{p \in E} B(p,r_p) \cap Y=G \cap Y \subseteq E\).

    若\(E=Y \cap G\),则\(\forall p \in E, \exists r_p>0, \forall q \in Y, d(p,q)<r_p, q \in E\),即\(E\)是\(Y\)的开子集,证毕.
\end{proof}

\begin{problem}[16]\label{pb2.16}
    \(E=\{p \in \mathbb{Q}: p>0, p^2 \in (2,3)\}\).证明\(E\)在\(\mathbb{Q}\)中是有界闭集、开集但不是紧集.
\end{problem}

\begin{proof}
    显然\(-2,2\)分别是\(E\)的下界和上界,故\(E\)是有界集.

    \(\forall p \in E\),定义\(r_p=\min \{(p^2-2)/(p+2), (3-p^2)/(p+3)\}\).考虑\(B(p,r_p) \cap \mathbb{Q}\).

    由于\(B(p,r_p) \cap \mathbb{Q} \subseteq E\),故\(\forall p \in E\)是\(E\)的内点,即\(E\)在\(\mathbb{Q}\)中是开集.

    考虑\(U_1=\{p \in \mathbb{Q}: p>0, p^2<2\}, U_2=\{p \in \mathbb{Q}: p>0, p^2>3\}, \forall p_1 \in U_1, p_2 \in U_2\).

    令\(r_1=(2-p^2)/(p+2), r_2=(p^2-3)/(p+3)\),则\(B(p_1,r_1) \cap E=\varnothing, B(p_2,r_2) \cap E=\varnothing\).

    然而考虑开覆盖序列\(\{G_k\}_{k \in \mathbb{Z}^+}, G_k=(q_k,2)\).其中\(q_0=3/2, q_k=(2q_{k-1}+2)/(q_{k-1}+2)\).

    显然\(E \subseteq \bigcup_{k=1}^\infty G_k\),但是若抽取有限子列\(G_{k_1}, \dots, G_{k_n}\),令\(\max\{k_1, \dots, k_r\}=k_0\),

    得到\(\bigcup_{r=1}^n G_{k_r}=G_{k_0}\),而\(\exists q_{k_0+1} \notin G_{k_0}\)且\(q_{k_0+1} \in E\),故\(E\)不是紧集.
\end{proof}

{\kaishu 下面是稠密性和可分度量空间的习题.}

\begin{problem}[22]\label{pb2.22}
    含有可数稠密子集的的度量空间是可分度量空间.证明\(\mathbb{R}^n\)是可分度量空间.
\end{problem}

\begin{proof}
    令\(Q=\{(q_1, \dots, q_n): \forall k=1, \dots, n, q_k \in \mathbb{Q}\}\),下证\(Q\)在\(\mathbb{R}^n\)中稠密.

    设\(x=(x_1, \dots, x_n) \in \mathbb{R}^n\),则\(\forall k=1, \dots, n, x_k\)有基本序列\(\{q_k^i\}, q_k^i \in \mathbb{Q}\).

    令\(q_i=(q_1^i, \dots, q_n^i)\),从而\(\lim_{i \to \infty} q_i=(q_1^i, \dots, q_n^i)=(x_1, \dots, x_n)=x\).
\end{proof}

\newpage

\begin{problem}[23]\label{pb2.23}
    证明可分度量空间\(X\)有一组可数基.
\end{problem}

\begin{proof}
    设\(X\)的可数稠密子集是\(U=\{u_1, u_2, \dots\}\).令\(\mathcal{C}=\{B(u_k,q_j): u_k \in U, q_j \in \mathbb{Q}^+\}\).

    显然\(\mathcal{C}\)是可数集,下证\(\forall x \in V \subseteq X, \exists j,k \in \mathbb{Z}^+, x \in B(u_k,q_j) \subseteq V\),其中\(V\)是开集.

    由\(V\)是开集,\(\exists r>0, B(x,r) \subseteq V\).由\(U\)的稠密性,选择\(u_k \in U, d(u_k,x)<r/2\).

    由有理数的稠密性,选择\(q_j \in (d(u_k,x), r/2)\).构造\(B(u_k,q_j)\),下证\(x \in B(u_k,q_j) \subseteq V\).

    \(\forall y \in B(u_k,q_j), d(x,y) \leq d(x,u_k)+d(u_k,y)<r/2+q_j<r/2+r/2=r\),故\(y \in B(x,r)\).

    由于\(d(u_k,x)<q_j\),故\(x \in B(u_k,q_j)\).结合两者,得到\(x \in B(u_k,q_j) \subseteq V\).
\end{proof}

\begin{problem}[24]\label{pb2.24}
    设度量空间\(X\)满足其中任一无限子集都有极限点,证明它是可分度量空间.
\end{problem}

\begin{proof}
    固定\(x_1 \in X, r>0\).依次选取\(x_2, x_3, \dots \in X\)使得\(\forall j<k \in \mathbb{Z}^+, d(x_j,x_k)>r\).

    若存在序列\(x_1, x_2, \dots \in X\)使得\(\forall k \in \mathbb{Z}^+, j<k \in \mathbb{Z}^+, d(x_j,x_k)>r\),下证\(\{x_k\}\)无极限点.
    
    设\(x\)是\(\{x_k\}\)的极限点,则\(\forall \varepsilon>0, \exists j \in \mathbb{Z}^+, d(x,x_j)<\varepsilon\),令\(\varepsilon=r/2\).

    则\(\forall k \in \mathbb{Z}^+, d(x,x_k) \geq \abs*{d(x,x_j)-d(x_j,x_k)} \geq r/2\),因此\(\forall k \in \mathbb{Z}^+, d(x,x_k) \geq r/2\).

    故\(\exists \varepsilon=r/2\)使得\(B(x,r/2) \cap \{x_k\}_{k \in \mathbb{Z}^+}=\varnothing\),即\(x\)不是极限点,从而序列无极限点,矛盾.

    故若给定\(r>0\),只存在有限的\(x_1^r, \dots, x_{m_r}^r\)使得\(\forall j,k=1, \dots, m_r, d(x_j^r,x_k^r)>r\),

    随后\(\forall x \in X, \exists k=1, \dots, m_r, d(x,x_k)<r\),从而\(X=\bigcup_{k=1}^{m_r} B(x_k,r)\).取\(r=1/n, n \in \mathbb{Z}^+\).
    
    从而\(\forall n \in \mathbb{Z}^+, \exists x_1^n, \dots, x_{m_n}^n \in X, \forall x \in X, \exists k=1, \dots, m_n, d(x,x_k^n)<1/n\).
    
    则\(\forall \varepsilon>0, \exists n \in \mathbb{Z}^+, 1/n<\varepsilon; \forall x \in X, \exists k=1, \dots, m_n, d(x,x_k^n)<1/n<\varepsilon\).

    令\(U=\bigcup_{n=1}^\infty \bigcup_{k=1}^{m_n} \{x_k^n\}\),因而\(U\)是\(X\)的可数稠密子集.
\end{proof}

{\kaishu 下面是紧集相关的习题.}

\begin{theorem}[2.36]\label{br2.36}
    紧集族\(\{F_\alpha\}_{\mathcal{A}}\)满足任意有限集\(\mathcal{S} \subseteq \mathcal{A}\)都有\(\bigcap_{\alpha \in \mathcal{S}} F_\alpha \ne \varnothing\),则\(\bigcap_{\alpha \in \mathcal{A}} F_\alpha \ne \varnothing\).
\end{theorem}

\begin{proof}
    设\(K \in \{F_\alpha\}_{\mathcal{A}}\).若\(K\)中不存在同时属于所有\(F_\alpha\)的点,则\(\{F_\alpha^c\}_{\mathcal{A}}\)是\(K\)的开覆盖.

    由于\(K\)是紧集,可以抽取一组有限子覆盖,即存在有限集\(\mathcal{S} \subseteq \mathcal{A}\)使得\(K \subseteq \bigcup_{\alpha \in \mathcal{S}} F_\alpha^c\).

    从而\(\bigcap_{\alpha \in \mathcal{S}} F_\alpha \subseteq K^c\),即\(K \cap \bigcap_{\alpha \in \mathcal{S}} F_\alpha=\varnothing\),与假设矛盾.
\end{proof}

\begin{theorem}[2.37]\label{br2.37}
    紧集\(K\)的无限子集\(E\)在\(K\)中必有极限点.
\end{theorem}

\begin{proof}
    若\(E\)在\(K\)中没有极限点,故\(\forall x \in K, \exists r_x>0, B(x,r_x) \cap E=\varnothing\)或\(\{x\}\).

    由于\(\{B(x,r_x)\}_{x \in K}\)是\(K\)的开覆盖,但它没有有限子覆盖,与\(K\)的紧性矛盾.

    {\kaishu 因此紧度量空间一定是可分度量空间,故紧度量空间拥有一组可数基.}
\end{proof}

\begin{problem}[12]\label{pb2.12}
    设\(K=\{0\} \cup \{1/n: n \in \mathbb{Z}^+\}\).证明\(K\)是紧集.
\end{problem}

\begin{proof}
    考虑\(K\)的可数开覆盖序列\(\{G_\alpha\}\),从中挑出任一包含\(0\)的开集\(G_0\).

    由\(0\)为\(G_0\)的内点,得\(\exists \delta>0, B(0,\delta) \subseteq G_0\),故\(\exists N \in \mathbb{Z}^+, \forall n>N, 1/n<\delta\).

    对于\(1, \dots, 1/N\),分别抽取\(G_1, \dots, G_N\)包含之,故\(G_0, G_1, \dots, G_N\)是一个有限子覆盖.
\end{proof}

\newpage

\begin{problem}[13]\label{pb2.13}
    构造一个\((0,1)\)的可数开覆盖,但它没有有限子覆盖.
\end{problem}

\begin{proof}
    令\(G_k=(1/3k, 1-1/3k), k \in \mathbb{Z}^+\),则\(\{G_k\}_{k \in \mathbb{Z}^+}\)是一个可数开覆盖.

    抽取有限子列\(G_{k_1}, \dots, G_{k_n}\),则令\(\max\{k_1, \dots, k_r\}=k_0\),得到\(\bigcup_{r=1}^n G_{k_r}=G_{k_0} \subseteq (0,1)\).
\end{proof}

\begin{comment}
    \begin{problem}[25]\label{pb2.25}
        证明紧度量空间\(X\)必有一组可数基.
    \end{problem}

    \begin{proof}
        由\(X\)为紧度量空间,设\(\forall n \in \mathbb{Z}^+, \exists x_1^n, \dots, x_{m_n}^n \in X, \bigcup_{k=1}^{m_n} B(x_k^n, 1/n)=X\).

        下证\(\bigcup_{n=1}^\infty \bigcup_{k=1}^{m_n} B(x_k^n, 1/n)\)是一组可数基.考虑开集\(V \subseteq X, x \in V\).

        存在\(r>0, B(x,r) \subseteq V\).由有理数的稠密性,选择\(N \in \mathbb{Z}^+, 1/N<r/2\).
        
        由\(X=\bigcup_{k=1}^{m_N} B(x_k^N, 1/n)\),于是\(\exists k \in \{1, \dots, m_N\}, x \in B(x_k^N, 1/N) \subseteq B(x,r) \subseteq V\).
    \end{proof}
\end{comment}

\begin{problem}[26]\label{pb2.26}
    设度量空间\(X\)满足其中任一无限子集都有极限点,证明它是紧度量空间.
\end{problem}

\begin{proof}
    \(X\)是{\kaishu 可分度量空间},故{\kaishu \(X\)的开覆盖总存在一组可数子覆盖}\(\{G_k\}_{k \in \mathbb{Z}^+}\).

    若\(X\)无有限子覆盖,那么\(\forall n \in \mathbb{Z}^+, X \setminus \bigcup_{k=1}^n G_k \ne \varnothing\),令\(F_n=X \setminus \bigcup_{k=1}^n G_k\).

    从每一\(F_n\)中抽取\(x_n\).序列\(x_1, x_2, \dots\)应有极限点\(x\)且\(\exists n \in \mathbb{Z}^+, x \in \bigcup_{k=1}^n G_k\),即\(x \notin F_n\).

    然而\(F_n\)为闭集,故\(x_n, x_{n+1}, \dots\)的极限点\(x\)应有\(x \in F_n\),与\(x \notin F_n\)矛盾.
\end{proof}

{\kaishu 下面是完全集相关习题.}

\begin{theorem}[2.43]\label{br2.43}
    非空完全集\(P\)不可数.
\end{theorem}

\begin{proof}
    取\(p_1 \in P, r_1>0\).由于\(P\)是完全集,故\(\exists p_2 \ne p_1 \in P, p_2 \in B(p_1,r_1)\).

    下面归纳地构造\(\{\overline{B}(x_k,r_k)\}\)序列.假设已经构造\(B(x_1,r_1), \dots, B(x_n,r_n)\).

    则\(\exists p_{n+1} \in P, p_{n+1} \in B(x_n,r_n)\).令\(r_{n+1}<\min \{d(x_n,x_{n+1}), r_n-d(x_n,x_{n+1})\}\).

    那么\(B(x_{n+1},r_{n+1})\)满足\(\overline{B}(x_{n+1},r_{n+1}) \subseteq \overline{B}(x_n,r_n), p_n \notin \overline{B}(x_{n+1},r_{n+1})\).

    且\(B(x_{n+1},r_{n+1}) \cap P\)始终非空,故归纳步骤可以一直进行.若\(P\)可数,则\(P=\{p_1, p_2, \dots\}\).

    然而\(\overline{B}(x_1,r_1) \supseteq \overline{B}(x_2,r_2) \supseteq \dots\)是嵌套紧集序列,故\(\bigcap_{n=1}^\infty \overline{B}(x_n,r_n) \ne \varnothing\).

    因此\(\exists p \in \bigcap_{n=1}^\infty \overline{B}(x_n,r_n) \cap P\),然而\(p \notin \{p_1, p_2, \dots\}=P\),矛盾.
\end{proof}

\begin{problem}[27]\label{pb2.27}
    设\(E \subseteq \mathbb{R}^n\),\(P\)是\(E\)的所有凝点的集,证明\(P\)是完全集,
    
    且\(E\)中最多有可数个点不在\(P\)中.
\end{problem}

\begin{proof}
    设\(\mathcal{S}=\{V_k\}_{k \in \mathbb{Z}^+}\)是\(\mathbb{R}^n\)的可数基.令\(\mathcal{C}=\{V_k: V_k \cap E \text{\kaishu 至多为可数集}\}\).

    下证\(\bigcup_{V_k \in \mathcal{C}} V_k=\mathbb{R}^n \setminus P\).\(\forall x \in \bigcup_{V_k \in \mathcal{C}} V_k , \exists V_k, V_k \cap E\)至多可数,故\(x \notin P\).

    若\(x \notin P\),\(\exists r>0, B(x,r) \cap E\)至多可数.\(\mathcal{S}\)为可数基,故\(\exists \mathcal{F}_x \subseteq \mathcal{S}, B(x,r)=\bigcup_{V_k \in \mathcal{F}_x} V_k\).

    显然\(\forall V_k \in \mathcal{F}_x, V_k \cap E\)至多可数,即\(\mathcal{F}_x \subseteq \mathcal{C}, x \in \bigcup_{V_k \in \mathcal{C}} V_k\).因而\(\mathbb{R}^n \setminus P=\bigcup_{V_k \in \mathcal{C}} V_k\).

    {\kaishu 由于\(\bigcup_{V_k \in \mathcal{C}} V_k\)是开集,故\(P\)是闭集,且至多只有可数点在\(E\)中,下证\(P\)是完全集.}

    取\(p \in P\),则\(\forall r>0, B(p,r) \cap E\)是不可数集.考虑\(B(p,r) \cap E\)的凝点集\(P'\).

    从\(P'\)中选择\(p' \ne p\),从而\(\forall r'>0, B(p',r') \cap B(p,r) \cap E\)也是不可数集,
    
    因此\(B(p',r') \cap E \supseteq B(p',r') \cap B(p,r) \cap E\)是不可数集,\(p'\)也是\(E\)的凝点.
    
    从而\(\forall p \in P, r>0, \exists p' \in P' \subseteq P, d(p,p')<r\),因此\(p'\)是\(p\)的极限点,故\(P\)是完全集.
\end{proof}

\newpage

\begin{problem}[28]\label{pb2.28}
    证明可分度量空间\(X\)的闭子集\(F\)都是一个完全集\(P\)和一个至多可数集的并.
\end{problem}

\begin{proof}
    令\(P\)是\(F\)所有凝点的集合,由于\(F\)是闭集,故\(P \subseteq F\).

    根据\cref{pb2.27},即可证明\(P\)是完全集且\(F \setminus P\)是至多可数集.
\end{proof}

\begin{problem}[29]\label{pb2.29}
    开集\(G \subset \mathbb{R}\)可以被写成至多可数不相交的开区间的并集.
\end{problem}

\begin{proof}
    \(\mathbb{R}\)的可数基为\(\mathcal{S}=\{(p,q): p,q \in \mathbb{Q}\}\).从而\(\forall G \subseteq \mathbb{R}, \exists \mathcal{C}_G \subseteq \mathcal{S}, G=\bigcup_{(p,q) \in \mathcal{C}_G}(p,q)\).

    由于\(\mathcal{C}_G\)是可数集,故\(\exists I_1, I_2, \dots \in \mathcal{C}_G, G=\bigcup_{k=1}^\infty I_k\).定义\(\mathbb{Z}^+\)的等价关系\(\sim\).

    定义\(n \sim m \in \mathbb{Z}^+\)若\(\exists m=k_1, \dots, k_p=n, \forall j=1, \dots, p-1, I_{k_j} \cap I_{k_{j+1}} \ne \varnothing\).

    定义\([k]=\{j \in \mathbb{Z}^+: j \sim k\}, C_k=\bigcup_{j \in [k]} I_j\),则\(G=\bigcup_{k=1}^\infty C_k\).

    若\(C_m \cap C_n \ne \varnothing\),则\(\exists m_k \in [m], n_k \in [n], I_{m_k} \cap I_{n_k} \ne \varnothing\).即\([m]=[n], C_m=C_n\).

    下证\(\forall k \in \mathbb{Z}^+, C_k\)是联通集.考虑\(\forall x<y \in C_k\),则\(\exists m,n \in [k], x \in I_m, y \in I_n\).

    因此\(\exists m=k_1, \dots, k_p=n, \forall j=1, \dots, p-1, I_{k_j} \cap I_{k_{j+1}} \ne \varnothing\).

    若\(\forall z \in (x,y), \exists q \in \{1, \dots, p\}, z \in I_{k_q}\),故而\(C_k\)是联通开集,即开区间.
\end{proof}

\begin{problem}[30]\label{pb2.30}
    若\(\forall k \in \mathbb{Z}^+, G_k\)是\(\mathbb{R}^n\)的稠密开子集,则\(\bigcap_{k=1}^\infty G_k \ne \varnothing\).
\end{problem}

\begin{proof}
    由\(G_1\)为开集,\(\exists x_1 \in G_1, r_1 \in (0,1), \overline{B}(x_1,r_1) \subseteq G_1\).下面{\kaishu 归纳地构造\(\overline{B}(x_k,r_k)\)序列}.

    假设\(B(x_1,r_1), \dots, B(x_n,r_n)\)已经被构造,且满足
    \begin{align*}
        \overline{B}(x_1,r_1) \supseteq \dots \supseteq \overline{B}(x_n,r_n), 
        \forall k=1, \dots, n, B(x_k,r_k) \subseteq G_k, r_k \in(0,1/k).
    \end{align*}
    由于\(G_{k+1}\)在\(\mathbb{R}^n\)中稠密,故\(\exists x_{k+1} \in G_{k+1}, x_{n+1} \in B(x_n,r_n)\).

    选取\(r_{n+1} \in (0,1/(n+1))\)使得\(\overline{B}(x_{n+1},r_{n+1}) \subseteq \overline{B}(x_n,r_n) \cap G_{n+1}\).

    因此\(\forall k \geq n, d(x_n,x_k)<1/n\),即\(\lim_{n \to \infty} x_n=x\)存在,且\(x \in \bigcap_{k=1}^\infty \overline{B}(x_k,r_k)\).
    
    因此\(x \in \bigcap_{k=1}^\infty G_k \supseteq \bigcap_{k=1}^\infty \overline{B}(x_k,r_k)\),证毕.
\end{proof}

\begin{comment}
    \begin{theorem}
        若有集合族\((X,\mathcal{S})\),则\(X \setminus \bigcup_{A \in \mathcal{S}} A=\bigcap_{A \in \mathcal{S}}(X \setminus A)\).
    \end{theorem}

    \begin{proof}
        考虑任意的\(x \in A \in \mathcal{S}\),则有
        \begin{align*}
            x \in X \setminus \bigcup_{A \in \mathcal{S}}A \Longleftrightarrow
            x \notin \bigcup_{A \in \mathcal{S}}A \Longleftrightarrow
            \forall A \in \mathcal{S}, x \in X \setminus A \Longleftrightarrow
            x \in \bigcap_{A \in \mathcal{S}} X \setminus A.
        \end{align*}
        于是\(X \setminus \bigcup_{A \in \mathcal{S}} A=\bigcap_{A \in \mathcal{S}}(X \setminus A)\).
    \end{proof}

    \begin{theorem}
        \(A\)是开集和\(A^c\)是闭集等价.
    \end{theorem}

    \begin{proof}
        若\(A\)是开集,那么对于\(A^c\)的任意极限点\(x\),都存在去心邻域\(N(x) \cap A^c \ne \varnothing\).

        于是\(N(x) \nsubseteq A\),也即\(x\)不是\(A\)的内点,于是\(x \notin A, x \in A^c\),\(A^c\)包含了它的所有极限点.

        若\(A^c\)是闭集,那么\(\forall x \in A, x\)不是\(A^c\)的极限点,因此存在\(x\)的去心邻域\(N(x) \cap A^c=\varnothing\).
        
        也即\(N(x) \subset A\),\(x\)是\(A\)的内点,所以\(A\)中的所有点都是\(A\)的内点.
    \end{proof}

    \begin{theorem}
        开集\(G\)对可数并封闭,对有限交封闭;闭集\(F\)对可数交封闭,对有限并封闭.    
    \end{theorem}

    \begin{proof}
        \(\bigcup_{G \in \mathcal{S}}G\)中的每一点\(x\)都是某个\(G_i\)的内点,于是存在\(N(x) \subset G_i \subset \bigcup_{G \in \mathcal{S}}G\).

        \(\bigcap_{i=1}^n G_i\)中的每一点\(x\)都存在一个邻域\(N_i(x) \subset G_i\),取\(N(x)=\bigcap_{i=1}^n N_i(x) \subset \bigcap_{i=1}^n G_i\).

        {\kaishu 对以上结论取补集并使用徳·摩根律即可证明闭集的结论.}
    \end{proof}

    \begin{theorem}
        \(A \subset \mathbb{R}\)是联通集当且仅当对于\(x<y \in A, (x,y) \subset A\).
    \end{theorem}

    \begin{proof}
        若存在\(z \in (x,y) \in \mathbb{R}\)且\(z \notin A\),令\(A_1=(-\infty,z) \cap A, A_2=(z,\infty) \cap A\).

        显然\(A_1,A_2\)是分离的,而\(A=A_1 \cup A_2\),那么\(A\)就不是联通的.
    \end{proof}

    \begin{theorem}
        开集\(G \subset \mathbb{R}\)可以被写成不相交的开区间\(I_1, I_2, \dots\)的并集\(\bigcup_{k=1}^\infty I_k\).
    \end{theorem}

    \begin{proof}
        定义\(x \sim y\)为存在\(G\)的联通子集\(C\)使得\(x,y \in C\),定义\([x]=\{y \in G: y \sim x\}\).

        若\([x] \cap [y] \ne \varnothing\),则\(\exists z \in G, x \sim z, y \sim z\),从而\(x \sim y\),则\([x]=[y]\).

        因此一个等价类\([x]\)规定一个最大联通子集\(C\),且它们互不相交,且\(G=\bigcup_{x \in G}[x]\).

        由有理数的稠密性,\([x]\)一定包含一个有理数\(q\),且\([x]=[q]\),那么\(G=\bigcup_{q \in G}[q]\).
    \end{proof}

    \begin{proof}
        定义\(I_x=(a_x,b_x), a_x=\inf\{a \in \mathbb{R}: (a,x] \subset G\}, b_x=\sup\{b \in \mathbb{R}: [x,b) \subset G\}\).

        由于\(G\)是开集,因而\(\forall x \in G, \exists N(x) \subset G\),于是\(\forall x \in G, x \in N(x) \subset I_x \subseteq \bigcup_{x \in G}I_x\).

        因此\(G \subseteq \bigcup_{x \in G}I_x\).另一方面,\(\forall x \in G, I_x \subset G, \bigcup_{x \in G}I_x \subseteq G\),故\(G=\bigcup_{x \in G}I_x\).

        若\(I_x \cap I_y \ne \varnothing\),则\(I_x=I_y\),由有理数的稠密性,可以为每个\(I_x\)选择任一有理数\(q_x\).

        若\(q_x \in I_y\),则\(I_x, I_y\)视为同一分支.由有理数集是可数的,于是总分支数是至多可数的.

        将所有涉及的有理数\(q_1, q_2, \dots\)所代表的区间记作\(I_1, I_2, \dots\),则\(G=\bigcup_{k=1}^\infty I_k\).
    \end{proof}

    \newpage

    \begin{theorem}[2.25]
        设\(\mathcal{S}\)是\(X\)上的\(\sigma-\)代数,那么对于\(A_1, \dots, A_2, \dots \in \mathcal{S}\),

        有\(A_1 \cap A_2 \in \mathcal{S}, A_1 \setminus A_2 \in \mathcal{S}, \bigcap_{i=1}^\infty A_i \in \mathcal{S}\).
    \end{theorem}

    \begin{proof}
        由\(X \setminus A_1, X \setminus A_2 \in \mathcal{S}\),于是\((X \setminus A_1) \cup (X \setminus A_2)=X \setminus (A_1 \cap A_2) \in \mathcal{S}\),即\(A_1 \cap A_2 \in \mathcal{S}\).

        由\(A_1 \setminus A_2=A_1 \cap (X \setminus A_2)\)且\(X \setminus A_2 \in \mathcal{S}\)推出\(A_1 \setminus A_2 \in \mathcal{S}\).

        由\(X \setminus A_i \in \mathcal{S}\)得到\(X \setminus \bigcap_{i=1}^\infty A_i=\bigcup_{i=1}^\infty (X \setminus A_i) \in \mathcal{S}\),于是\(\bigcap_{i=1}^\infty A_i \in \mathcal{S}\).
    \end{proof}

    \begin{theorem}[2.33]
        设\(f: X \to Y\)是一个函数,\((Y,\mathcal{A})\)确定一个集合族且\(A \subseteq \mathcal{A}\),那么
        
        a.\(f^{-1}(X \setminus A)=Y \setminus f^{-1}(A)\) \enspace
        b.\(f^{-1}(\bigcup_{A \in \mathcal{A}}A)=\bigcup_{A \in \mathcal{A}}f^{-1}(A)\) \enspace
        c.\(f^{-1}(\bigcap_{A \in \mathcal{A}}A)=\bigcap_{A \in \mathcal{A}}f^{-1}(A)\)
    \end{theorem}

    \begin{proof}
        考虑任意的\(x \in A \in \mathcal{A}\),则有
        \begin{align*}
            &x \in f^{-1}(Y \setminus A) \Longleftrightarrow 
            f(x) \in Y \setminus A \Longleftrightarrow
            f(x) \notin A \Longleftrightarrow x \in X \setminus f^{-1}(A) \\
            &x \in f^{-1}(\bigcup_{A \in \mathcal{A}}A) \Longleftrightarrow
            f(x) \in \bigcup_{A \in \mathcal{A}}A \Longleftrightarrow
            \exists A \in \mathcal{A}, f(x) \in A \Longleftrightarrow
            x \in \bigcup_{A \in \mathcal{A}}f^{-1}(A) \\
            &x \in f^{-1}(\bigcap_{A \in \mathcal{A}}A) \Longleftrightarrow
            f(x) \in \bigcap_{A \in \mathcal{A}}A \Longleftrightarrow
            \forall A \in \mathcal{A}, f(x) \in A \Longleftrightarrow
            x \in \bigcap_{A \in \mathcal{A}}f^{-1}(A) \qedhere
        \end{align*}
    \end{proof}

    \begin{theorem}[2.27]
        设\(\mathcal{A}\)是\(X\)上的集合族,那么所有包含\(\mathcal{A}\)的\(\sigma-\)代数之交也是\(\sigma-\)代数.
    \end{theorem}

    \begin{proof}
        令\(\mathcal{S}\)是所有包含\(\mathcal{A}\)的\(\sigma-\)代数之交.\(X\)上的所有\(\sigma-\)代数都包含空集.

        若\(A \in \mathcal{S}\),那么\(X \setminus A\)也在每一个包含\(A\)的\(\sigma-\)代数中,也即\(X \setminus A \in \mathcal{S}\).

        若\(A_1, A_2, \dots \in \mathcal{S}\),那么\(\bigcup_{i=1}^\infty A_i\)也在每一个包含\(A_i\)的\(\sigma-\)代数中,也即\(\bigcup_{i=1}^\infty A_i \in \mathcal{S}\).

        {\kaishu 因此对于任意集合族都存在包含之的最小\(\sigma-\)代数.}
    \end{proof}
\end{comment}