\section{Real Number System}

\begin{theorem}[0]\label{br1.1}
    实数集\(\mathbb{R}\)是完备的有序域,构造蓝图呈现如下.
    \begin{enumerate}
        \item 在集合\(\mathbb{F}\)上定义二元运算\(+: \mathbb{F} \times \mathbb{F} \to \mathbb{F}, \cdot: \mathbb{F} \times \mathbb{F} \to \mathbb{F}\),验证\((\mathbb{F}, +, \cdot)\)成域.
        \item 定义域\(\mathbb{F}\)的正部\(\mathbb{F}^+ \subseteq \mathbb{F}\),验证\(\mathbb{F}^+\)满足如下性质:
            \begin{enumerate}
                \item[2.1] 三分性:\(\forall x \in \mathbb{F}, x \in \mathbb{F}^+, -x \in \mathbb{F}^+, x=0\)三者有且仅有其一成立.
                \item[2.2] 反对称性:\(\forall x \in \mathbb{F}^+, -x \notin \mathbb{F}^+\).
                \item[2.3] 加乘闭合性:\(\forall x,y \in \mathbb{F}^+, x+y \in \mathbb{F}^+, xy \in \mathbb{F}^+\). 
            \end{enumerate}
        \item 定义\(x<y\)为\(y-x \in \mathbb{F}^+\),定义\(x>y\)为\(y<x\).现在\((\mathbb{F}, +, \cdot, <)\)成有序域.
        \item 记\(\mathbb{F}^*=\mathbb{F}^+ \cup \{0\}\).定义绝对值\(\abs*{\cdot}: \mathbb{F} \to \mathbb{F}\)为\(\forall x \in \mathbb{F}^*, \abs*{x}=x; \forall x \notin \mathbb{F}^*, \abs*{x}=-x\),
        
        可以证明有三角不等式\(\abs*{x+y} \leq \abs*{x}+\abs*{y}\)成立.
        \item 从绝对值运算中诱导度量函数\(d: \mathbb{F} \times \mathbb{F} \to \mathbb{F}^*\)为\(d(x,y)=\abs*{x-y}\).
        \item 定义\((\mathbb{F}, +, \cdot, <)\)中的基本序列\(\{x_n\}\)为\(\forall \varepsilon \in \mathbb{F}^+, \exists N \in \mathbb{Z}^+, \forall m,n>N, d(x_m,x_n)<\varepsilon\).
        
        定义\((\mathbb{F}, +, \cdot, <)\)中的序列\(\{x_n\}\)收敛至\(x\)为\(\forall \varepsilon \in \mathbb{F}^+, \exists N \in \mathbb{Z}^+, \forall n>N, d(x_n,x)<\varepsilon\),

        记作\(\lim_{n \to \infty} x_n=x\).序列\(\{x_n\}\)是基本序列是其收敛的必要(可能不充分)条件.
        \item 验证\((\mathbb{F}, +, \cdot, <)\)中的任意基本序列\(\{x_n\}\)收敛,并记\((\mathbb{F}, +, \cdot, <)\)是完备的有序域.
    \end{enumerate}
    对于\(\mathbb{F}\)的任意子集\(A\),若存在\(x \in \mathbb{F}\)使得\(\forall a \in A, a \leq x\),则定义\(x\)为\(A\)的一个上界.

    若\(x\)是\(A\)的上界,且\(\forall y<x, y\)都不是\(A\)的上界,则定义\(x\)是\(A\)的上确界.

    若\(\forall A \subseteq \mathbb{R}, A \ne \varnothing\)且\(A\)有上界能推出\(\sup A \in \mathbb{F}\),则称\(\mathbb{F}\)具有最小上界性.

    \newpage

    对于任意\(\mathbb{Q}\)中的基本序列\(\{a_n\},\{b_n\}\),定义\(\{a_n\} \sim \{b_n\}\)为\(\lim_{n \to \infty} \abs*{a_n-b_n}=0\).

    定义\([\{a_n\}]=\{\{b_n\}: \{b_n\} \sim \{a_n\}\}\).定义实数集\(\mathbb{R}\)为
    \begin{align*}
        \mathbb{R}=\{[\{a_n\}]: \{a_n\}\text{是}\mathbb{Q}\text{中的基本序列}\}.
    \end{align*}
    则\(\mathbb{R}\)是一个完备的有序域.由基本序列构造还需注意一些良定义性细节,如下.

    \begin{enumerate}
        \item 验证有理数基本序列等价关系的良定义性.
        \item 定义实数集上的加法和乘法运算\(+: \mathbb{R} \times \mathbb{R} \to \mathbb{R}, \cdot: \mathbb{R} \times \mathbb{R} \to \mathbb{R}\)为
            \begin{align*}
                \forall [\{a_n\}],[\{b_n\}] \in \mathbb{R}, [\{a_n\}]+[\{b_n\}]=[\{a_n+b_n\}], [\{a_n\}] \cdot [\{b_n\}]=[\{a_nb_n\}]
            \end{align*}
        验证加法和乘法运算,单位元和逆元对等价类的良定义性,随后验证\((\mathbb{R}, +, \cdot)\)成域.
        \item 定义实数的正部\(\mathbb{R}^+=\{[\{x_n\}] \in \mathbb{R}: \exists \varepsilon \in \mathbb{Q}^+, N \in \mathbb{Z}^+, \forall n>N, x_n>\varepsilon\}\).
        
        验证\(\mathbb{R}^+\)对等价类的良定义性,从中诱导序关系\(<\),绝对值\(\abs*{\cdot}\)和度量函数\(d\).

        验证序关系,绝对值和度量对等价类的良定义性,从此\((\mathbb{R}, +, \cdot, <)\)成有序域.
        \item 定义嵌入映射\(\varphi: \mathbb{Q} \to \mathbb{R}\)为\(\varphi(q)=[\{q,q,\dots\}]\).
        
        定义\(\widetilde{\mathbb{Q}} \subseteq \mathbb{R}\)为\(\{\varphi(q): q \in \mathbb{Q}\}\),证明\((\widetilde{\mathbb{Q}}, +, \cdot, <)\)构成有序域.
        \item 定义基本序列\(\{[\{x_n^k\}]\}_{k \in \mathbb{Z}^+}\)为\(\forall \varepsilon \in \mathbb{Q}^+, \exists N \in \mathbb{Z}^+, \forall m,n>N, d([\{x_m^k\}], [\{x_n^k\}])<\varphi(\varepsilon)\).
    
        定义\(\{[\{x_n^k\}]\}_{k \in \mathbb{Z}^+}\)收敛于\([\{x_n\}]\)为\(\forall \varepsilon \in \mathbb{Q}^+, \exists N \in \mathbb{Z}^+, \forall k>N, d([\{x_n^k\}], [\{x_n\}])<\varphi(\varepsilon)\).
    
        记作\(\lim_{k \to \infty} [\{x_n^k\}]=[\{x_n\}]\).证明\(\{[\{x_n^k\}]\}_{k \in \mathbb{Z}^+}\)是基本序列和其收敛于某\([\{x_n\}]\)等价.
    \end{enumerate}
    这里不使用\(\mathbb{R}^+\)中的小量来定义收敛性是因为基本序列构造的实数的序结构依赖于正有理数,因此若使用\(\mathbb{R}^+\)来定义实数,最后也将化归为对正有理数的无穷小讨论.
    
    这个定义要求看起来虽弱些,但最终我们可以证明有理数在实数中稠密,从而使用\(\mathbb{R}^+\)和\(\mathbb{Q}^+\)来定义收敛性是等价的.
\end{theorem}

\newpage

\begin{proof}[证明2]
    设\([\{x_n\}] \ne 0_\mathbb{R}\),即\(\exists \varepsilon_0 \in \mathbb{Q}^+, \forall N_1 \in \mathbb{Z}^+, \exists n_0>N_1, \abs*{x_{n_0}} \geq \varepsilon_0\).

    由于\(\{x_n\}\)是基本序列,故\(\forall \varepsilon \in \mathbb{Q}^+, \exists N_2 \in \mathbb{Z}^+, \forall m,n>N_2, \abs*{x_m-x_n}<\varepsilon/2\).

    取\(\varepsilon=\varepsilon_0, m=n_0, \forall n>\max\{N_1,N_2\}, \abs*{x_n}=\abs*{x_{n_0}-(x_{n_0}-x_n)} \geq \abs*{\abs*{x_{n_0}}-\abs*{x_{n_0}-x_n}}>\varepsilon_0/2\).

    定义序列\(\{y_n\}\)为\(y_n=0, n \leq N_2; y_n=x_n^{-1}, n>N_2\).下证\(\{y_n\}\)是基本序列.

    由于\(\{x_n\}\)是基本序列,故\(\forall \varepsilon \in \mathbb{Q}^+, \exists N_3 \in \mathbb{Z}^+, \forall m,n>N_3, \abs*{x_m-x_n}<\varepsilon \varepsilon_0^2/4\).
    \begin{align*}
        \forall m,n>\max\{N_2,N_3\}, \abs*{y_m-y_n}=\abs*{\frac{1}{x_m}-\frac{1}{x_n}}=\frac{\abs*{x_m-x_n}}{\abs*{x_mx_n}}
        <\frac{\varepsilon \varepsilon_0^2/4}{\varepsilon_0^2/4}=\varepsilon
    \end{align*}
    下证乘法逆元的良定义性.设\(\{x_n^1\} \sim \{x_n^2\}\),分别对应\(\{y_n^1\},\{y_n^2\}\),下证\(\{y_n^1\} \sim \{y_n^2\}\).

    如前所述,\(\exists \varepsilon_1, \varepsilon_2 \in \mathbb{Q}^+, N_1,N_2 \in \mathbb{Z}^+, \forall n>N_1, \abs*{x_n^1}>\varepsilon_1/2; \forall n>N_2, \abs*{x_n^2}>\varepsilon_2/2\).

    由于\(\{x_n^1\} \sim \{x_n^2\}\),故\(\forall \varepsilon \in \mathbb{Q}^+, \exists N_3 \in \mathbb{Z}^+, \forall n>N_3, \abs*{x_n^1-x_n^2}<\varepsilon \varepsilon_1 \varepsilon_2/4\).
    \begin{align*}
        \forall n>\max\{N_1,N_2,N_3\}, \abs*{y_n^1-y_n^2}=\abs*{\frac{1}{x_n^1}-\frac{1}{x_n^2}}=\frac{\abs*{x_n^1-x_n^2}}{\abs*{x_n^1x_n^2}}
        <\frac{\varepsilon \varepsilon_1 \varepsilon_2/4}{\varepsilon_1 \varepsilon_2/4}=\varepsilon
    \end{align*}
    由于\([\{x_n\}] \cdot [\{y_n\}]=[\{x_ny_n\}]=1_\mathbb{R}\),故乘法逆元是存在且良定义的.
\end{proof}

\begin{proof}[证明3.0]
    设\(\{x_n^1\} \sim \{x_n^2\}\)且\([\{x_n^1\}] \in \mathbb{R}^+\).有\(\exists \varepsilon_0 \in \mathbb{Q}^+, N_1 \in \mathbb{Z}^+, \forall n>N_1, x_n^1>\varepsilon_0\).

    由于\(\{x_n^1\} \sim \{x_n^2\}\),故\(\forall \varepsilon \in \mathbb{Q}^+, \exists N_2 \in \mathbb{Z}^+, \forall n>N_2, \abs*{x_n^1-x_n^2}<\varepsilon/2\).

    取\(\varepsilon=\varepsilon_0/2\),得到\(\forall n>\max\{N_1,N_2\}, x_n^2=x_n^1-(x_n^1-x_n^2) \geq x_n^1-\abs*{x_n^1-x_n^2}>\varepsilon_0/2\).

    因此\(\exists \varepsilon_0/2 \in \mathbb{Q}^+, \forall n>\max\{N_1,N_2\},  x_n^2>\varepsilon_0/2\),即\([\{x_n^2\}] \in \mathbb{R}^+\),良定义性证毕.
\end{proof}
    
\begin{proof}[证明3.1]
    先设\([\{x_n\}] \ne 0_\mathbb{R}\),即\(\exists \varepsilon_0 \in \mathbb{Q}^+, \forall N_1 \in \mathbb{Z}^+, \exists n_0>N_1, \abs*{x_{n_0}} \geq \varepsilon_0\).

    由于\(\{x_n\}\)是基本序列,故\(\forall \varepsilon \in \mathbb{Q}^+, \exists N_2 \in \mathbb{Z}^+, \forall m,n>N_2, \abs*{x_m-x_n}<\varepsilon_0/2\).

    取\(\varepsilon=\varepsilon_0, m=n_0, \forall n>\max\{N_1,N_2\}, \abs*{x_n}=\abs*{x_{n_0}-(x_{n_0}-x_n)} \geq \abs*{\abs*{x_{n_0}}-\abs*{x_{n_0}-x_n}}>\varepsilon_0/2\).

    若\(\exists m,n>\max\{N_1,N_2\}, x_m>\varepsilon_0/2, x_n<-\varepsilon_0/2\),则\(\abs*{x_m-x_n}>\varepsilon_0\),与基本序列矛盾.

    故\(\forall [\{x_n\}] \ne 0_\mathbb{R}, \exists \varepsilon_0 \in \mathbb{Q}^+, N=\max\{N_1,N_2\}, \forall n>N, x_n>\varepsilon_0/2\)或\(\forall n>N, -x_n>\varepsilon_0/2\).

    若为\(x_n>\varepsilon_0/2\),则\([\{x_n\}] \in \mathbb{R}^+\);否则\(-x_n>\varepsilon_0/2\),从而\(-[\{x_n\}]=[\{-x_n\}] \in \mathbb{R}^+\).

    {\kaishu 因此\([\{x_n\}] \in \mathbb{R}^+, -[\{x_n\}] \in \mathbb{R}^+, [\{x_n\}]=0_\mathbb{R}\)有且仅有其一成立,三歧性证毕.}
\end{proof}

\begin{proof}[证明3.2]
    设\([\{x_n\}], -[\{x_n\}] \in \mathbb{R}^+\).根据加法闭合性公理,\([\{x_n\}]+(-[\{x_n\}])=0_\mathbb{R} \in \mathbb{R}^+\).

    根据三歧性,\([\{x_n\}]+(-[\{x_n\}])=0_\mathbb{R}, [\{x_n\}]+(-[\{x_n\}]) \in \mathbb{R}^+\)不同时成立,矛盾.
\end{proof}

\begin{proof}[证明3.3]
    设\([\{x_n\}], [\{y_n\}] \in \mathbb{R}^+\),下证\([\{x_n\}]+[\{y_n\}] \in \mathbb{R}^+, [\{x_n\}] \cdot [\{y_n\}] \in \mathbb{R}^+\).
    
    有\(\exists \varepsilon_1, \varepsilon_2 \in \mathbb{Q}^+, N_1,N_2 \in \mathbb{Z}^+, \forall n>N_1, x_n>\varepsilon_1, \forall n>N_2, y_n>\varepsilon_2\).

    因此\(\exists \varepsilon_0=\min\{\varepsilon_1+\varepsilon_2, \varepsilon_1 \varepsilon_2\}, N=\max\{N_1,N_2\}, \forall n>N, x_n+y_n>\varepsilon_0, x_ny_n>\varepsilon_0\).

    得到\([\{x_n\}]+[\{y_n\}]=[\{x_n+y_n\}] \in \mathbb{R}^+, [\{x_n\}] \cdot [\{y_n\}]=[\{x_ny_n\}] \in \mathbb{R}^+\),闭合性证毕.
\end{proof}

\newpage

\begin{proof}[证明4]
    \((\widetilde{\mathbb{Q}}, +, \cdot, <)\)构成有序域的证明显然,下证\(\lim_{n \to \infty} \varphi(x_n)=[\{x_n\}]\).

    由\(\{x_n\}\)是基本序列,故\(\forall \varepsilon \in \mathbb{Q}^+, \exists N \in \mathbb{Z}^+, \forall m,n>N, \abs*{x_m-x_n}<\varepsilon\).

    因此\(\forall \varepsilon \in \mathbb{Q}^+, n>N, d(\varphi(x_n),[\{x_n\}])=[\{0, \abs*{x_{n+1}-x_n},\abs*{x_{n+2}-x_n},\dots\}]\).

    依次取\(m\)为\(n+k, k \in \mathbb{Z}^+\)得\(\forall \varepsilon>0, \exists N \in \mathbb{Z}^+, \forall n>N, k \in \mathbb{Z}^+, \abs*{x_{n+k}-x_n}<\varepsilon\),
    
    故\(d(\varphi(x_n),[\{x_n\}])=[\{0, \abs*{x_{n+1}-x_n},\abs*{x_{n+2}-x_n},\dots\}]<[\{\varepsilon,\varepsilon,\dots\}]=\varphi(\varepsilon)\).
\end{proof}



\begin{proof}[证明5]
    由于\(\forall k \in \mathbb{Z}^+, \lim_{n \to \infty} \varphi(x_n^k)=[\{x_n^k\}]\),故\(\exists N_k \in \mathbb{Z}^+, d(\varphi(x_{N_k}^k), [\{x_n^k\}])<\varphi(1/k)\).

    \(\{[\{x_n^k\}]\}_{k \in \mathbb{Z}^+}\)是基本序列,故\(\forall \varepsilon \in \mathbb{Q}^+, \exists M_1 \in \mathbb{Z}^+, \forall p,q>M_1, d([\{x_n^p\}],[\{x_n^q\}])<\varphi(\varepsilon/3)\).

    令\(c_n=x_{N_n}^n\).下证\(\{c_n\}\)是基本序列且\(\lim_{k \to \infty} [\{x_n^k\}]=[\{c_n\}]\).取\(\forall p,q>\max\{M_1,\left\lceil 3/\varepsilon \right\rceil\}\),
    \begin{align*}
        d(\varphi(c_p),\varphi(c_q)) &\leq d(\varphi(c_p), [\{x_n^p\}])+d([\{x_n^p\}], [\{x_n^q\}])+d(\varphi(c_q), [\{x_n^q\}]) \\
        &<\varphi(\varepsilon/3)+\varphi(\varepsilon/3)+\varphi(\varepsilon/3)=\varphi(\varepsilon)
    \end{align*}
    因此\(\varphi(\abs*{c_p-c_q})=d(\varphi(c_p),\varphi(c_q))<\varphi(\varepsilon)\)推出\(\abs*{c_p-c_q}<\varepsilon\),即\(\{c_n\}\)确为基本序列.
    
    由\(\lim_{n \to \infty} \varphi(c_n)=[\{c_n\}]\),故\(\forall \varepsilon \in \mathbb{Q}^+, \exists M_2 \in \mathbb{Z}^+, \forall k>M_2, d(\varphi(c_k), [\{c_n\}])<\varphi(\varepsilon/2)\).
    \begin{align*}
        \forall k>\max\{M_2,\left\lceil 2/\varepsilon \right\rceil\}, d([\{x_n^k\}], [\{c_n\}]) &\leq d([\{x_n^k\}], \varphi(c_k))+d(\varphi(c_k), [\{c_n\}]) \\
        &<\varphi(\varepsilon/2)+\varphi(\varepsilon/2)=\varphi(\varepsilon)
    \end{align*}
    因此\(\lim_{k \to \infty} [\{x_n^k\}]=[\{c_n\}]\),完备性证毕.
\end{proof}

\begin{theorem}[1.20.b]\label{br1.20.b}
    设\([\{x_n\}]<[\{y_n\}]\),则\(\exists q \in \mathbb{Q}, [\{x_n\}]<\varphi(q)<[\{y_n\}]\).
\end{theorem}

\begin{proof}
    由于\([\{x_n\}]<[\{y_n\}]\),故\(\exists \varepsilon_0 \in \mathbb{Q}^+, N_0 \in \mathbb{Z}^+, \forall n>N, y_n-x_n>\varepsilon_0\).

    由于\(\lim_{n \to \infty}(\varphi(x_n))=[\{x_n\}], \lim_{n \to \infty}(\varphi(y_n))=[\{y_n\}]\),
    
    故\(\exists N_1,N_2 \in \mathbb{Z}^+, \forall n>N_1, d(\varphi(x_n),[\{x_n\}])<\varphi(\varepsilon_0/4), \forall n>N_2, d(\varphi(y_n),[\{y_n\}])<\varphi(\varepsilon_0/4)\).

    令\(N=\max\{N_0,N_1,N_2\}+1, q=(x_N+y_N)/2\),下证\([\{x_n\}]<\varphi(q)<[\{y_n\}]\).

    由于\(N>N_1, d(\varphi(x_N),[\{x_n\}])<\varphi(\varepsilon_0/4)\),即\(\varphi(x_N)-[\{x_n\}]>-\varphi(\varepsilon_0/4)\).

    同时\(N>N_0, \varphi(q)-\varphi(x_N)=\varphi((x_N+y_N)/2-x_N)=\varphi((y_N-x_N)/2)>\varphi(\varepsilon_0/2)\).

    因此\(\varphi(q)-[\{x_n\}]=(\varphi(q)-\varphi(x_N))+(\varphi(x_N)-[\{x_n\}])>\varphi(\varepsilon_0/4)\),即\(\varphi(q)>[\{x_n\}]\).

    由于\(N>N_1, d(\varphi(y_N),[\{y_n\}])<\varphi(\varepsilon_0/4)\),即\([\{y_n\}]-\varphi(y_N)>-\varphi(\varepsilon_0/4)\).

    同时\(N>N_0, \varphi(y_N)-\varphi(q)=\varphi(y_N-(x_N+y_N)/2)=\varphi((y_N-x_N)/2)>\varphi(\varepsilon_0/2)\).

    因此\([\{y_n\}]-\varphi(q)=([\{y_n\}]-\varphi(y_N))+(\varphi(y_N)-\varphi(q))>\varphi(\varepsilon_0/4)\),即\([\{y_n\}]>\varphi(q)\).

    {\kaishu 稠密性得证后,以正实数定义的收敛和以正有理数定义的等价.}
\end{proof}

\newpage

\begin{theorem}[1.19]\label{br1.19}
    \(\mathbb{R}\)具有最小上界性.
\end{theorem}

\begin{proof}
    设\(A \ne \varnothing\)且有上界.令\(K_n=\{k \in \mathbb{Z}: k/n \text{是} A \text{的上界}\}\).记\(\min K_n=k_n\).
    
    令\(x_n=k_n/n, y_n=(k_n-1)/n\).因此\(\forall n \in \mathbb{Z}^+, \varphi(x_n)\)都是\(A\)的上界,\(\varphi(y_n)\)都不是\(A\)的上界.

    先证\(\{x_n\}\)是基本序列.\(\forall m,n \in \mathbb{Z}^+, \varphi(x_m-1/m)\)不是\(A\)的上界,\(\varphi(x_n)\)是\(A\)的上界.

    因此\(\exists a \in A, \varphi(x_m-1/m)<a<\varphi(x_n)\),即\(\forall m,n \in \mathbb{Z}^+, x_m-x_n<1/m\).

    同理\(\varphi(x_n-1/n)\)不是\(A\)的上界,\(\varphi(x_m)\)是\(A\)的上界,因此\(x_n-x_m<1/n\).
    
    因此\(\forall m,n \in \mathbb{Z}^+, \abs*{x_m-x_n}<\max\{1/m,1/n\}\).因此\(\forall m,n>\left\lceil 1/\varepsilon \right\rceil, \abs*{x_m-x_n}<\varepsilon\).

    先证\([\{x_n\}]\)是\(A\)的上界.下设\(\exists a \in A, a>[\{x_n\}]\).由于\(\lim_{n \to \infty} \varphi(x_n)=[\{x_n\}]\),
    
    故\(\exists N \in \mathbb{Z}^+, d(\varphi(x_N),[\{x_n\}])<d(a,[\{x_n\}])\),即\(\varphi(x_N)<a\).然而\(\varphi(x_N)\)是\(A\)的上界,矛盾.
    
    由于\(\lim_{n \to \infty} \varphi(y_n)=[\{x_n\}]\),故\(\exists N \in \mathbb{Z}^+, d(\varphi(y_N),[\{x_n\}])<d(x,[\{x_n\}])\),即\(\varphi(y_N)>x\).
    
    然而\(\varphi(y_N)\)并非\(A\)的上界,即\(\exists a \in A, x<\varphi(y_N)<a\).因此\(\forall x<[\{x_n\}]\)均非\(A\)的上界.
\end{proof}

\begin{theorem}[1.21]\label{br1.21}
    对于任意的\(x>0, q \in \mathbb{Z}^+\),存在唯一的\(y>0\)使得\(y^q=x\).
\end{theorem}

\begin{lemma}
    \(\forall 0<a<b<c, b^q-a^q<q(b-a)b^{q-1}<q(b-a)c^{q-1}\).
\end{lemma}

\begin{proof}[引理证明]
    由于\(b^q-a^q=(b-a)\sum_{k=0}^{q-1} b^{q-k-1} a^k\),结合\(\forall k \in \mathbb{Z}^+, a^k<b^k<c^k\),

    故\((b-a)\sum_{k=0}^{q-1} b^{q-k-1} a^k<(b-a)\sum_{k=0}^{q-1} b^{q-k-1} b^k=q(b-a)b^{q-1}<q(b-a)c^{q-1}\).
\end{proof}

\begin{proof}[定理证明]
    先证明\(\forall [\{y_n\}]>0, [\{y_n\}]^q=[\{y_n^q\}]\)是基本序列.由于\(\{y_n\}\)有界,
    
    故\(\exists M>0, \forall n \in \mathbb{Z}^+, \abs*{y_n}<M\).根据引理,\(\forall m,n \in \mathbb{Z}^+, \abs*{y_m^q-y_n^q}<q M^{q-1} \abs*{y_m-y_n}\).

    由于\(\{y_n\}\)是基本序列,故\(\forall \varepsilon>0, \exists N \in \mathbb{Z}^+, \forall m,n>N, \abs*{y_m-y_n}<\varepsilon/q M^{q-1}\).

    因此\(\forall \varepsilon>0, \exists N \in \mathbb{Z}^+, \forall m,n>N, \abs*{y_m^q-y_n^q}<\varepsilon\),即\(\{y_n^q\}\)确为基本序列.

    令\(S=\{t \in \mathbb{R}^+: t^q<x\}\),根据\cref{br1.21},\(S\)非空且有上界\(x+1\).令\(\sup S=y\),下证\(y^q=x\).

    先设\(y^q<x\).考虑\(\delta \in (0,1)\),根据引理,\((y+\delta)^q-y^q<q \delta (y+\delta)^{q-1}<q \delta (y+1)^{q-1}\).

    取\(\delta=\min\{1, (x-y^q)/(q(y+1)^{q-1})\}\)将有\((y+\delta)^n<x\).因此\(y+\delta \in S\),矛盾.

    再设\(y^q>x\).考虑\(\delta \in (0,y)\),根据引理,\(y^q-(y-\delta)^q<q \delta y^{q-1}\).

    取\(\delta=\min\{y/2, (y^q-x)/(qy^{q-1})\}\)将有\((y-\delta)^n>x\).因此\(y-\delta<y\)是\(S\)的上界,矛盾.

    根据三歧性,只能是\((\sup S)^q=x\).下证其唯一性,设\(y_1^q=y_2^q=x\)且\(y_1 \ne y_2\).

    下面归纳地证明若\(y_1<y_2\),则\(y_1^q<y_2^q\).设\(y_1<y_2\)能推出\(y_1^{q-1}<y_2^{q-1}\),

    那么\(y_1^q=y_1 y_1^{q-1}< y_1 y_2^{q-1}< y_2 y_2^{q-1}=y_2^q\).同理若\(y_1>y_2\)则\(y_1^q>y_2^q\),均矛盾.故\(y_1=y_2\).
\end{proof}

\newpage

\begin{problem}[6]\label{pb1.6}
    对于\(\forall a>1\),定义\(a^{[\{x_n\}]}=\lim_{n \to \infty} a^{x_n}\),证明该极限存在.
\end{problem}

\begin{lemma}
    \(\forall x>0, n \in \mathbb{Z}^+, (1+x)^n \geq 1+nx\).
\end{lemma}

\begin{proof}[引理证明]
    根据二项式定理有\((1+x)^n=\sum_{k=0}^n C_n^k x^k=1+nx+\dots\),取前两项即得.
\end{proof}

\begin{proof}[定理证明]
    由于\(\{x_n\}\)作为基本序列有界,故\(\exists M \in \mathbb{Z}^+, \forall n \in \mathbb{Z}^+, \abs*{x_n}<M\).

    因此\(\forall m,n \in \mathbb{Z}^+, \abs*{a^{x_m}-a^{x_n}}=\abs*{a^{x_n}}\abs*{a^{x_m-x_n}-1} \leq \abs*{a^M}\abs*{a^{x_m-x_n}-1}\).

    \(\forall q \in \mathbb{Z}^+, \exists N_1 \in \mathbb{Z}^+, \forall m,n>N_1, \abs*{x_m-x_n}<1/q\),从而\(\abs*{a^M}\abs*{a^{x_m-x_n}-1} \leq \abs*{a^M}\abs*{a^{1/q}-1}\).

    因此\(a=((a^{1/q}-1)+1)^q\).根据引理,\(a \geq 1+q(a^{1/q}-1)\),即\(a^{1/q}-1 \leq (a-1)/q\).

    因此\(\abs*{a^{x_m}-a^{x_n}} \leq \abs*{a^M}\abs*{a^{x_m-x_n}-1} \leq \abs*{a^M}\abs*{a^{1/q}-1} \leq a^M(a-1)/q\).

    最终\(\forall \varepsilon>0, \exists q \in \mathbb{Z}^+, a^M(a-1)/q<\varepsilon, \exists N_1 \in \mathbb{Z}^+, \forall m,n>N_1, \abs*{x_m-x_n}<1/q\),证毕.
\end{proof}

\begin{problem}[7]\label{pb1.7}
    对于\(\forall a>1, x>0\),存在唯一的实数\(y\)使得\(a^y=x\).
\end{problem}

\begin{lemma}
    \(\forall a>1, n \in \mathbb{Z}^+, a-1 \geq n(a^{1/n}-1)\).
\end{lemma}

\begin{proof}[引理证明]
    \(a^n-1=(a-1)\sum_{k=0}^{n-1} a^{n-k-1} \leq (a-1)\sum_{k=0}^{n-1} 1=n(a-1)\),代入\(a^{1/n}\)即可.
\end{proof}

\begin{proof}[定理证明]
    由\(a^{1/n} \leq 1+(a-1)/n\),因此\(\exists \delta>1, n>(a-1)/(\delta-1), a^{1/n}<\delta\).

    令\(S=\{t \in \mathbb{R}^+: a^t<x\}, y=\sup S\),下证\(y\)是唯一满足\(a^y=x\)的实数.

    先设\(a^y<x\),则\(x/a^y>1\).令\(\delta=x/a^y\),则\(\exists n \in \mathbb{Z}^+, n>(a-1)/(\delta-1), a^{1/n}<\delta\).

    考虑\(a^{y+(1/n)}=a^y a^{1/n}<a^y \delta=a^y (x/a^y)=x\),从而\(y+(1/n) \in S\),矛盾.

    再设\(a^y>x\),则\(a^y/x>1\).令\(\delta=a^y/x\),则\(\exists n \in \mathbb{Z}^+, n>(a-1)/(\delta-1), a^{1/n}<\delta\).

    考虑\(a^{y-(1/n)}=a^y (1/a^{1/n})>a^y (1/\delta)=x\),从而\(y-(1/n)<y\)是\(S\)的上界,矛盾.
    \begin{comment}
        设\(y_1,y_2\)满足\(a^{y_1}=a^{y_2}\),下证\(y_1<y_2\).设\(y_1=[\{c_n^1\}], y_2=[\{c_n^2\}]\),其中\(c_n^1, c_n^2 \in \mathbb{Q}\).

        由有理数的稠密性,\(\exists r_1,r_2 \in \mathbb{Q}, y_1<\varphi(r_1)<\varphi(r_2)<y_2\).

        根据收敛性,\(\exists N \in \mathbb{Z}^+, \forall n>N, \varphi(c_n^1) \leq r_1, a^{c_n^1} \leq a^{r_1}; \varphi(c_n^2) \leq r_2, a^{c_n^2} \geq a^{r_2}\).

        因此\(\lim_{n \to \infty} a^{c_n^1}=a^{y_1} \leq a^{r_1}, \lim_{n \to \infty} a^{c_n^2}=a^{y_2} \geq a^{r_2}\),即\(a^{y_1} \leq a^{r_1}<a^{r_2} \leq a^{y_2}\),矛盾.
    \end{comment}
\end{proof}

\begin{comment}
    \begin{proof}[证明7]
        先证\(\{\abs*{x_n-y_n}\}\)为基本序列.由\(\{x_n\},\{y_n\}\)都是基本序列,
        
        故\(\forall \varepsilon \in \mathbb{Q}^+, \exists N_1,N_2 \in \mathbb{Z}^+, \forall m,n>N_1, \abs*{x_m-x_n}<\varepsilon/2, \forall m,n>N_2, \abs*{y_m-y_n}<\varepsilon/2\).

        从而\(\forall \varepsilon>0, m,n>\max\{N_1,N_2\}\),有
            \begin{align*}
                \abs*{(x_m-y_m)-(x_n-y_n)}=\abs*{(x_m-x_n)-(y_m-y_n)} \leq \abs*{x_m-x_n}+\abs*{y_m-y_n}<\varepsilon
            \end{align*}
        下证\(d([\{a_n\}],[\{b_n\}])\)的良定义性.设\(\{x_n^1\} \sim \{x_n^2\}, \{y_n^1\} \sim \{y_n^2\}\).

        得到\(\forall \varepsilon \in \mathbb{Q}^+, \exists N_1,N_2 \in \mathbb{Z}^+, \forall n>N_1, \abs*{x_n^1-x_n^2}<\varepsilon/2, \forall n>N_2, \abs*{y_n^1-y_n^2}<\varepsilon/2\).

        从而\(\forall \varepsilon \in \mathbb{Q}^+, n>\max\{N_1,N_2\}\),有
            \begin{align*}
                \abs*{\abs*{x_n^1-y_n^1}-\abs*{x_n^2-y_n^2}} &\leq \abs*{(x_n^1-y_n^1)-(x_n^2-y_n^2)}=\abs*{(x_n^1-x_n^2)-(y_n^1-y_n^2)} \\
                &\leq \abs*{x_n^1-x_n^2}+\abs*{y_n^1-y_n^2}<\varepsilon/2+\varepsilon/2=\varepsilon
            \end{align*}
        因此\(d\)是合法的映射.显然\(d([\{x_n\}],[\{x_n\}])=0_\mathbb{R}\),下设\(d([\{x_n\}],[\{y_n\}])=0_\mathbb{R}\).

        则\([\{\abs*{x_n-y_n}\}]=0_\mathbb{R}, \lim_{n \to \infty}\abs*{x_n-y_n}=0\),故\(\{x_n\} \sim \{y_n\}, [\{x_n\}]=[\{y_n\}]\).
        
        因此\([\{x_n\}]=[\{y_n\}]\)等价于\(d([\{x_n\}],[\{y_n\}])=0_\mathbb{R}\),下设\(d([\{x_n\}],[\{y_n\}]) \ne 0_\mathbb{R}\).

        则根据\textit{证明6},\(\exists \varepsilon_0 \in \mathbb{Q}^+, N \in \mathbb{Z}^+, \forall n>N, \abs*{x_n-y_n}>\varepsilon_0/2\),即\([\{\abs*{x_n-y_n}\}]>0_\mathbb{R}\).
        \begin{align*}
            d([\{x_n\}],[\{z_n\}])&=[\{\abs*{x_n-z_n}\}]=[\{\abs*{(x_n-y_n)+(y_n-z_n)}\}] \leq [\{\abs*{x_n-y_n}+\abs*{y_n-z_n}\}] \\
            &=[\{\abs*{x_n-y_n}\}]+[\{\abs*{y_n-z_n}\}]=d([\{x_n\}],[\{y_n\}])+d([\{y_n\}],[\{z_n\}])
        \end{align*}
        显然\(d([\{x_n\}],[\{y_n\}])=[\{\abs*{x_n-y_n}\}]=[\{\abs*{y_n-x_n}\}]=d([\{y_n\}],[\{x_n\}])\).
    \end{proof}

    \begin{proof}[证明8]
        考虑\(\forall q_1,q_2 \in \mathbb{Q}\).下证\(\varphi(q_1)+\varphi(q_2)=\varphi(q_1+q_2), \varphi(q_1) \cdot \varphi(q_2)=\varphi(q_1q_2)\).
            \begin{align*}
                &\varphi(q_1)+\varphi(q_2)=[\{q_1,q_1,\dots\}]+[\{q_2,q_2,\dots\}]=[\{(q_1+q_2),(q_1+q_2),\dots\}]=\varphi(q_1+q_2) \\
                &\varphi(q_1) \cdot \varphi(q_2)=[\{q_1,q_1,\dots\}] \cdot [\{q_2,q_2,\dots\}]=[\{q_1q_2,q_1q_2,\dots\}]=\varphi(q_1q_2)
            \end{align*}
        同时\(q_2>q_1 \Longleftrightarrow \varphi(q_2-q_1)=[\{(q_2-q_1),(q_2-q_1),\dots\}]>0_\mathbb{R} \Longleftrightarrow \varphi(q_2)>\varphi(q_1)\).

        {\kaishu 因此我们可以将\(\mathbb{Q}\)视作\(\mathbb{R}\)的子域,于是可以定义实数的基本序列和实数的收敛如下.

        定义实数基本序列\(\{[\{x_n^k\}]\}_{k \in \mathbb{Z}^+}\)为\(\forall \varepsilon \in \mathbb{Q}^+, \exists N \in \mathbb{Z}^+, \forall m,n>N, d([\{x_m^k\}], [\{x_n^k\}])<\varphi(\varepsilon)\).

        定义\(\{[\{x_n^k\}]\}_{k \in \mathbb{Z}^+}\)收敛于\([\{x_n\}]\)为\(\forall \varepsilon \in \mathbb{Q}^+, \exists N \in \mathbb{Z}^+, \forall k>N, d([\{x_n^k\}], [\{x_n\}])<\varphi(\varepsilon)\).

        记作\(\lim_{k \to \infty} [\{x_n^k\}]=[\{x_n\}]\),且下证\(\lim_{n \to \infty} \varphi(x_n)=[\{x_n\}]\).\footnote{\kaishu 这其实就是“有理数收敛到实数”的确切说法.}}

        由\(\{x_n\}\)是基本序列,故\(\forall \varepsilon \in \mathbb{Q}^+, \exists N \in \mathbb{Z}^+, \forall m,n>N, \abs*{x_m-x_n}<\varepsilon\).

        因此\(\forall \varepsilon \in \mathbb{Q}^+, n>N, d(\varphi(x_n),[\{x_n\}])=[\{0, \abs*{x_{n+1}-x_n},\abs*{x_{n+2}-x_n},\dots\}]\).

        依次取\(m\)为\(n+k, k \in \mathbb{Z}^+\)得\(\forall \varepsilon>0, \exists N \in \mathbb{Z}^+, \forall n>N, k \in \mathbb{Z}^+, \abs*{x_{n+k}-x_n}<\varepsilon\),

        故\(d(\varphi(x_n),[\{x_n\}])=[\{0, \abs*{x_{n+1}-x_n},\abs*{x_{n+2}-x_n},\dots\}]<[\{\varepsilon,\varepsilon,\dots\}]=\varphi(\varepsilon)\).
    \end{proof}

    \begin{theorem}[1.19]\label{br1.19}
        考虑\(\mathcal{S} \subseteq \mathbb{R}, [\{u_n\}] \in \mathbb{R}\).若\(\forall [\{s_n\}] \in \mathcal{S}, [\{s_n\}] \leq [\{u_n\}]\),则\([\{u_n\}]\)是\(\mathcal{S}\)的上界.

        若\([\{u_n\}]\)是\(\mathcal{S}\)的上界且\(\forall [\{u'_n\}]<[\{u_n\}]\)不是\(\mathcal{S}\)的上界,则定义\([\{u_n\}]\)是\(\mathcal{S}\)的上确界.
        
        则设\(\mathcal{S} \subseteq \mathbb{R}\)非空且有上界,则\(\sup \mathcal{S} \in \mathbb{R}\).
    \end{theorem}

    \begin{proof}[完备性证明]
        考虑实数域中的基本序列\(\{x_j\}, x_j \in \mathbb{R}\),其中\(x_j=[\{a_{i,j}\}], a_{i,j} \in \mathbb{Q}\).

        由于\(\{x_j\}\)是基本序列,则\(\forall \varepsilon>0, \exists N_0 \in \mathbb{N}, \forall m,n>N_0, d(x_j,x_k)<\varepsilon/2\).

        由于\(\{a_{i,j}\}\)也是基本序列,则\(\forall n \in \mathbb{N}, \exists N_n \in \mathbb{N}, \forall m>N_n, d(a_{m,j},x_j)<1/n\).

        取\(\{a_{i,j}\}\)的收敛子列\(\{c_n\}\)满足\(c_n=a_{N_n,n}\),下证\(\{c_n\}\)是基本序列且\(\lim_{n \to \infty}x_n=[\{c_n\}]\).

        \(\forall \varepsilon>0, \exists M \in \mathbb{N}, \forall m,n>M, 1/m,1/n<\varepsilon/3\),于是\(\forall m,n>\max\{N_0,M\}\),有
            \begin{align*}
                \abs*{a_{N_m,m}-a_{N_n,n}} &\leq d(a_{N_m,m},x_m)+d(x_m,x_n)+d(a_{N_n,n},x_n) \\
                &<1/m+\varepsilon/3+1/n<\varepsilon/3+\varepsilon/3+\varepsilon/3=\varepsilon
            \end{align*}
        于是\(\{c_n\}\)是基本序列,下证\(\lim_{n \to \infty}x_n=[\{c_n\}]\).\(\forall \varepsilon>0, \exists M' \in \mathbb{N}, \forall n>M', 1/n<\varepsilon\).
            \begin{align*}
                \forall n>M', d(c_n,x_n)=d(a_{N_n,n},x_n)<1/n<\varepsilon 
            \end{align*}
        {\kaishu 因此任意实数域中的基本序列在域中的收敛,即实数域具有完备性.}
    \end{proof}

    \begin{theorem}\label{1.A.2} 区间套定理 \:
        考虑区间套\([a_1,b_1] \supseteq [a_2,b_2] \supseteq \dots\)满足\(\lim_{n \to \infty}(b_n-a_n)=0\).

        其中\(\forall n \in \mathbb{N}, a_n,b_n \in \mathbb{R}\).求证:\(\exists c \in \mathbb{R}\)为该区间序列的唯一公共点.
    \end{theorem}

    \begin{proof}
        先证明\(\{a_n\},\{b_n\}\)都是基本序列.由于\(\lim_{n \to \infty}(b_n-a_n)=0\),即
            \begin{align*}
                \forall \varepsilon>0, \exists N \in \mathbb{N}, 
                \forall m,n>N, \abs*{a_m-a_n}<b_n-a_n<\varepsilon, \abs*{b_m-b_n}<b_n-a_n<\varepsilon
            \end{align*}
        因此\(\{a_n\},\{b_n\}\)都是基本序列.由于实数域中基本序列必收敛和\(\lim_{n \to \infty}(b_n-a_n)=0\),

        故\(\{a_n\},\{b_n\}\)收敛于同一点\(c \in \mathbb{R}\),下证其唯一性.考虑\(\exists c' \ne c\)为第二公共点.

        若\(c'<c\),则\(\exists N \in \mathbb{N}, \forall n>N, a_n>c', b_n>c'\),即\(c' \notin [a_n,b_n]\).

        若\(c'>c\),则\(\exists N \in \mathbb{N}, \forall n>N, b_n<c', a_n<c'\),即\(c' \notin [a_n,b_n]\).

        {\kaishu 根据实数的全序性,不存在这样的\(c'\).}
    \end{proof}

    \begin{theorem}\label{1.A.3} 确界原理 \:
        任何非空有上界的集合\(A \subset R\)有上确界.
    \end{theorem}

    \begin{proof}
        由于\(A\)有上界,考虑\(a_1,b_1 \in \mathbb{Q}\).其中\(a_1 \in A\),\(b_1\)是\(A\)的上界.
        考虑\((a_n+b_n)/2 \in \mathbb{Q}\).

        若\((a_n+b_n)/2 \in A\),则令\(a_{n+1}=a_n, b_{n+1}=(a_n+b_n)/2\);

        若\((a_n+b_n)/2\)是\(A\)的上界,则令\(a_{n+1}=(a_n+b_n)/2, b_{n+1}=b_n\).

        因此\([a_1,b_1] \supseteq [a_2,b_2] \supseteq \dots\)且\(\lim_{n \to \infty}(b_n-a_n)=0\),
        则\(\lim_{n \to \infty}a_n=\lim_{n \to \infty}b_n=c \in \mathbb{R}\).

        下证\(c=\sup A\).若\(\exists a \in A, a>c\),那么\(\exists N \in \mathbb{N}, \forall n>N, b_n<a\),矛盾.

        若\(\exists b<c\)为\(A\)的上界,那么\(\exists N \in \mathbb{N}, \forall n>N, a_n<b\),矛盾,因此只能有\(c=\sup A\).
    \end{proof}

    \begin{theorem}\label{1.A.4}
        单调递增且有上界的实数列\(\{a_n\} \in \mathbb{R}\)一定收敛.
    \end{theorem}

    \begin{proof}
        根据确界原理,\(\{a_n\}\)有上确界\(\sup\{a_n\}=c\).下证\(\lim_{n \to \infty}a_n=c\).

        由于\(c\)为上确界,因此\(\forall \varepsilon>0, \exists N \in \mathbb{N}, \forall n>N, a_n>c-\varepsilon\).

        改写为\(\forall \varepsilon>0, \exists N \in \mathbb{N}, \forall n>N, \abs*{a_n-c}<\varepsilon\),
        这显然就是\(\lim_{n \to \infty}a_n=c\)的\(\varepsilon-N\)定义.
    \end{proof}

    \begin{theorem}\label{1.A.5} 波尔查诺定理 \:
        有界实数列\(\{c_n\}\)必有收敛子列.
    \end{theorem}

    \begin{proof}
        设\(\forall n \in \mathbb{N}, c_n \in [a_1,b_1]\),考虑区间\(A_n=[a_n,(a_n+b_n)/2]\)和\(B_n=[(a_n+b_n)/2,b_n]\).

        必有其中之一包含无限项\(\{c_n\}\),若为\(A_n\)则令\(a_{n+1}=(a_n+b_n)/2, b_{n+1}=b_n\),若为\(B_n\)则反之.

        得到区间序列\([a_1,b_1] \supseteq [a_2,b_2] \supseteq \dots\)且\(\lim_{n \to \infty}(b_n-a_n)=0\),
        \(\lim_{n \to \infty}a_n=\lim_{n \to \infty}b_n=c\).

        根据区间套定理,从每一\([a_i,b_i]\)中抽取\(\{c_{n_i}\}\),形成\(\{c_n\}\)的子列\(c_{n_i}\).
        于是\(\lim_{i \to \infty}c_{n_i}=c\).
    \end{proof}

    \begin{theorem}\label{1.A.6} 有限覆盖定理 \:
        每个有界闭区间都有一个有限子覆盖.
    \end{theorem}

    \begin{proof}
        用反证法.设\(\forall n \in \mathbb{N}, a_n \in [a_1,b_1]\),考虑\(A_n=[a_n,(a_n+b_n)/2]\)和\(B_n=[(a_n+b_n)/2,b_n]\).

        必有其中之一没有有限子覆盖,若为\(A_n\)则令\(a_{n+1}=(a_n+b_n)/2, b_{n+1}=b_n\),若为\(B_n\)则反之.

        得到区间序列\([a_1,b_1] \supseteq [a_2,b_2] \supseteq \dots\)且\(\lim_{n \to \infty}(b_n-a_n)=0\),
        \(\lim_{n \to \infty}a_n=\lim_{n \to \infty}b_n=c\).

        考虑开区间\(c \in (\alpha,\beta)\),则\(\exists N \in \mathbb{N}, \forall n>N, a_n>\alpha, b_n<\beta\).

        显然\([a_n,b_n] \subset (\alpha,\beta)\),即\([a_n,b_n]\)有一个有限子覆盖,矛盾.
    \end{proof}
\end{comment}