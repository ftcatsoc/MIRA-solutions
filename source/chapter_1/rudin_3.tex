\section{Numerical Sequences and Series}

\begin{problem}[13]\label{pb3.13}
    证明若\(\sum_{n=0}^\infty a_n, \sum_{n=0}^\infty b_n\)绝对收敛,则\(\sum_{n=0}^\infty c_n, c_n=\sum_{k=0}^n a_kb_{n-k}\)也绝对收敛.
\end{problem}

\begin{proof}
    \(\sum_{n=1}^N \abs*{c_n}=\sum_{n=1}^N \abs*{\sum_{k=0}^n a_kb_{n-k}} \leq \sum_{n=1}^N \sum_{k=0}^n \abs*{a_kb_{n-k}}\).
    \begin{align*}
        \sum_{n=0}^N \abs*{a_n} \cdot \sum_{n=0}^N \abs*{b_n}=\sum_{n=0}^{2N} \sum_{j+k=n}^{j,k \leq N} \abs*{a_jb_k}
        \geq \sum_{n=0}^N \sum_{j+k=n} \abs*{a_jb_k}=\sum_{n=0}^N \sum_{k=0}^n \abs*{a_kb_{n-k}} \geq \sum_{n=0}^N \abs*{c_n}
    \end{align*}
    由于\(\sum_{n=0}^\infty \abs*{a_n}, \sum_{n=0}^\infty \abs*{b_n}\)收敛,故\(\sum_{n=0}^N \abs*{c_n} \leq \sum_{n=0}^N \abs*{a_n} \cdot \sum_{n=0}^N \abs*{b_n}\)有界.

    结合\(\sum_{n=0}^N \abs*{c_n}\)的单调性,故\(\sum_{n=0}^\infty c_n\)绝对收敛.
\end{proof}

\begin{problem}[16]\label{pb3.16}
    设\(a>0, x_1>\sqrt{a}\),定义\(x_{n+1}=(x_n+a/x_n)/2\).证明\(\{x_n\}\)单调递减且收敛于\(\sqrt{a}\).
    
    定义误差\(\varepsilon_n=x_n-\sqrt{a}\),证明\(\varepsilon_{n+1}=\varepsilon_n^2/(2x_n)<\varepsilon_n^2/(2\sqrt{a})\).
\end{problem}

\begin{proof}
    设\(x_n>\sqrt{a}\),则\(x_{n+1}-\sqrt{a}=(x_n^2-2\sqrt{a}x_n+a)/(2x_n)=(x_n-\sqrt{a})^2/(2x_n)>0\).

    考虑\(x_{n+1}-x_n=(-x_n+a/x_n)/2\),显然\(a/x_n<\sqrt{a}, -x_n<-\sqrt{a}\),即\(x_{n+1}-x_n<0\).

    因此\(\lim_{n \to \infty} x_n=L\),故\(L=(L+a/L)/2\),即\(L=\sqrt{a}\).

    由于\(x_{n+1}-\sqrt{a}=\varepsilon_{n+1}, x_n-\sqrt{a}=\varepsilon_n\),代入上式即\(\varepsilon_{n+1}=\varepsilon_n^2/(2x_n)\).
\end{proof}

\begin{problem}[17]\label{pb3.17}
    设\(a>1, x_1>\sqrt{a}\),定义\(x_{n+1}=(a+x_n)/(1+x_n)\).证明\(\{x_n\}\)震荡收敛于\(\sqrt{a}\).
    
    定义误差\(\varepsilon_n=x_n-\sqrt{a}\),证明\(\varepsilon_{n+1}=((1-\sqrt{a})/(1+x_n))\varepsilon_n\).
\end{problem}

\begin{proof}
    下证\(\forall n \in \mathbb{Z}^+, \varepsilon_{n+1} \varepsilon_n<0\)且\(\abs*{\varepsilon_{n+1}}<\abs*{\varepsilon_n}\).
    \begin{align*}
        x_{n+1}-\sqrt{a}=\frac{a+x_n-\sqrt{a}-\sqrt{a} x_n}{1+x_n}
        =\frac{\sqrt{a}(1-\sqrt{a})+x_n(1-\sqrt{a})}{1+x_n}=\frac{1-\sqrt{a}}{1+x_n}(x_n-\sqrt{a})
    \end{align*}
    由于\(\abs*{\varepsilon_{n+1}/\varepsilon_n}=\abs*{(1-\sqrt{a})/(1+x_n)}<\abs*{(1-\sqrt{a})/(1+\sqrt{a})}<1\),
    
    故\(\lim_{n \to \infty} \varepsilon_n=0\),即\(\lim_{n \to \infty} x_n=a\).
\end{proof}

\begin{problem}[18]\label{pb3.18}
    设\(a>0, x_1>\sqrt[p]{a}\),定义\(x_{n+1}=\dfrac{p-1}{p} x_n+\dfrac{a}{p} x_n^{-p+1}\).
    
    证明\(\{x_n\}\)收敛于\(\sqrt{a}\),误差平方收敛.
\end{problem}

\begin{proof}
    考虑\(g(x)=\dfrac{p-1}{p} x+\dfrac{a}{p} x^{-p+1}=x, g'(x)=\dfrac{p-1}{p} (1-\dfrac{a}{x^p})\).

    令\(g(x)=x\),得到\(x_0=\sqrt[p]{a}, g'(x_0)=0\),因此\(\sqrt[p]{a}\)是\(g(x)\)的不动点.

    在\(\sqrt[p]{a}\)处展开\(x_{n+1}=g(x_n)=g(\sqrt[p]{a})+g'(\sqrt[p]{a})(x-\sqrt[p]{a})+O((x-\sqrt[p]{a})^2)\).

    由于\(g(\sqrt[p]{a})=\sqrt[p]{a}, g'(\sqrt[p]{a})=0\),故\(x_{n+1}-\sqrt[p]{a}=O((x_{n+1}-\sqrt[p]{a})^2)\).
\end{proof}

\newpage

\subsection*{Supplement from CMC problems}

\begin{problem}[1]\label{cmc1}
    设\(\forall n \in \mathbb{Z}^+, a_n \geq 0\)且\(\sum_{n=1}^\infty a_n\)发散.定义\(S_n=\sum_{k=1}^n a_n\).

    a.证明\(\forall m<n \in \mathbb{Z}^+, \sum_{k=m+1}^n a_k/S_k \geq 1-(S_m/S_n)\)且\(\sum_{n=1}^\infty a_n/S_n\)发散.

    b.证明\(\forall \alpha \in (1,\infty), \sum_{n=1}^\infty a_n/S_n^\alpha\)收敛.
\end{problem}

\begin{proof}[证明a]
    \(a_k/S_k \geq a_k/S_n\),故\(\sum_{k=m+1}^n a_k/S_k>\sum_{k=m}^n a_k/S_n=(S_n-S_m)/S_n=1-S_m/S_n\).

    由于\(\lim_{n \to \infty} S_n=\infty\),故\(\exists m_0, m_1, \dots \in \mathbb{Z}^+, \forall j \in \mathbb{Z}^+, S_{m_j} \geq 2S_{m_{j-1}}\),其中\(N_0=1\).
    \begin{align*}
        \sum_{k=1}^\infty \frac{a_n}{S_n}=\sum_{j=1}^\infty \sum_{k=m_{j-1}+1}^{m_j} \frac{a_n}{S_n} 
        \geq \sum_{j=1}^\infty 1-\frac{S_{m_j}}{S_{m_{j-1}}} \geq \sum_{j=1}^\infty 1-\frac{S_{m_{j-1}}}{2S_{m_{j-1}}}=\sum_{j=1}^\infty \frac{1}{2}=\infty
    \end{align*}
    因此\(\sum_{n=1}^\infty a_n/S_n\)发散.
\end{proof}

\begin{proof}[证明b]
    由于\(a_n/S_n^\alpha=(S_n-S_{n-1})/S_n^\alpha\),因此\(\forall x \in [S_{n-1},S_n], 1/x^\alpha \geq 1/S_n^\alpha\).
    \begin{align*}
        \frac{S_n-S_{n-1}}{S_n^\alpha}=\int_{S_{n-1}}^{S_n} \frac{dx}{S_n^\alpha} \leq \int_{S_{n-1}}^{S_n} \frac{dx}{x^\alpha}, \quad
        \sum_{n=1}^\infty \frac{a_n}{S_n^\alpha}=\int_{a_1}^\infty \frac{dx}{S_n^\alpha} \leq \int_{a_1}^\infty \frac{dx}{x^\alpha}<\infty
    \end{align*}
    因此\(\sum_{n=1}^\infty a_n/S_n^\alpha\)收敛.
\end{proof}

\begin{problem}[2]\label{cmc2}
    设\(\forall n \in \mathbb{Z}^+, a_n \geq 0\)且\(\sum_{n=1}^\infty a_n\)收敛.定义\(r_n=\sum_{k=n}^\infty a_k\).

    a.证明\(\forall m<n \in \mathbb{Z}^+, \sum_{k=m}^n a_k/r_k>1-(r_n/r_m)\)且\(\sum_{n=1}^\infty a_n/r_n\)发散.

    b.证明\(\forall \alpha \in (0,1), \sum_{n=1}^\infty a_n/r_n^\alpha\)收敛.
\end{problem}

\begin{proof}[证明a]
    \(a_k/r_k>a_k/r_m\),故\(\sum_{k=m}^n a_k/r_k>\sum_{k=m}^n a_k/r_m=(r_m-r_n)/r_m=1-r_n/r_m\).

    由于\(\sum_{n=1}^\infty a_n\)收敛,故\(\exists m_0, m_1, \dots \in \mathbb{Z}^+, \forall j \in \mathbb{Z}^+, r_{m_j} \leq r_{m_{j-1}}/2\),其中\(N_0=1\).
    \begin{align*}
        \sum_{n=1}^\infty \frac{a_n}{r_n}=\sum_{j=1}^\infty \sum_{k=m_{j-1}+1}^{m_j} \frac{a_n}{r_n} 
        >\sum_{j=1}^\infty 1-\frac{r_{m_j}}{r_{m_{j-1}}} \geq \sum_{j=1}^\infty 1-\frac{r_{m_{j-1}}}{2r_{m_{j-1}}}=\sum_{n=1}^\infty \frac{1}{2}=\infty
    \end{align*}
    因此\(\sum_{n=1}^\infty a_n/r_n\)发散.
\end{proof}

\begin{proof}[证明b]
    由于\(a_n/r_n^\alpha=(r_n-r_{n+1})/r_n^\alpha\),因此\(\forall x \in [r_{n+1},r_n], 1/x^\alpha \geq 1/S_n^\alpha\).
    \begin{align*}
        \frac{r_{n+1}-r_n}{r_n^\alpha}=\int_{r_{n+1}}^{r_n} \frac{dx}{r_n^\alpha} \leq \int_{r_{n+1}}^{r_n} \frac{dx}{x^\alpha}, \quad
        \sum_{n=1}^\infty \frac{a_n}{r_n^\alpha}=\int_0^{r_1} \frac{dx}{r_n^\alpha} \leq \int_0^{r_1} \frac{dx}{x^\alpha}<\infty
    \end{align*}
    因此\(\sum_{n=1}^\infty a_n/r_n^\alpha\)收敛.
\end{proof}

\begin{comment}
    \begin{problem}[4]\label{pb3.4}
        定义序列\(\{a_n\}\)为\(a_1=0, a_{2n}=a_{2n-1}/2, a_{2n+1}=1/2+a_{2n}\),求\(\{a_n\}\)的上下极限.
    \end{problem}

    \begin{proof}
        有\(a_{2n+1}=(a_{2n-1}+1)/2, a_{2n+1}-1=(a_{2n-1}-1)/2\).定义\(b_n=a_{2n-1}-1, b_{n+1}=b_n/2\).

        由于\(b_1=-1\),故\(b_n=-1/2^{n-1}, a_{2n-1}=1-1/2^{n-1}\),所以\(\lim_{n \to \infty} a_{2n-1}=1\).

        同时\(a_{2n}=a_{2n-1}/2=1/2-1/2^n\),所以\(\lim_{n \to \infty} a_{2n}=1/2\).

        最终\(\limsup_{n \to \infty} a_n=1, \liminf_{n \to \infty} a_n=1/2\).
    \end{proof}

    \begin{problem}[6]\label{pb3.6}
        考虑\(a_n=(\sqrt{n+1}-\sqrt{n})/n\)和\(a_n=(\sqrt[n]{n}-1)^n\)的敛散性.
    \end{problem}

    \begin{proof}
        \(a_n=(\sqrt{n+1}-\sqrt{n})/n\)收敛,注意到
        \begin{align*}
            a_n=\frac{\sqrt{n+1}-\sqrt{n}}{n} \leq \frac{\sqrt{n+1}}{n+1}-\frac{\sqrt{n}}{n}, 
            \sum_{n=1}^\infty a_n \leq \lim_{n \to \infty} \frac{\sqrt{n+1}}{n+1}=1
        \end{align*}
        \(a_n=(\sqrt[n]{n}-1)^n\)收敛,注意到\(\limsup_{n \to \infty} \sqrt[n]{a_n}=\limsup_{n \to \infty} (\sqrt[n]{n}-1)\).

        由于\(\sqrt[n]{n}-1=e^{\ln n/n}-1=O(\ln n/n)\),故\(\limsup_{n \to \infty} (\sqrt[n]{n}-1)=0\),级数收敛.
    \end{proof}

    \begin{problem}[7]\label{pb3.7}
        设\(\forall n \in \mathbb{Z}^+, a_n \geq 0\)且\(\sum_{n=1}^\infty a_n\)收敛,则\(\sum_{n=1}^\infty (\sqrt{a_n}/n)\)收敛.
    \end{problem}

    \begin{proof}
        由平均值不等式有\(\sqrt{a_n}/n=\sqrt{a_n \cdot 1/n^2} \leq a_n/2+1/2n^2\),
        
        故\(\sum_{n=1}^\infty (\sqrt{a_n}/n) \leq \sum_{n=1}^\infty a_n/2+\sum_{n=1}^\infty 1/2n^2\),且\(\sum_{n=1}^\infty a_n, \sum_{n=1}^\infty 1/n^2\)收敛.
    \end{proof}

    \begin{problem}[8]\label{pb3.8}
        设\(\sum_{n=1}^\infty a_n\)收敛且\(\{b_n\}\)单调有界,证明\(\sum_{n=1}^\infty a_nb_n\)收敛.
    \end{problem}

    \begin{proof}
        {\kaishu 不失一般性,设\(\{b_n\}\)单调增有上界}.设\(\sup b_n=b\),考虑\(c_n=b-b_n\).

        从而\(\sum_{n=1}^\infty a_nb_n=\sum_{n=1}^\infty a_n(b-c_n)=b \sum_{n=1}^\infty a_n-\sum_{n=1}^\infty a_nc_n\).

        显然\(\lim_{n \to \infty} c_n=0\)且\(\{c_n\}\)单调递减,故\(b \sum_{n=1}^\infty a_n, \sum_{n=1}^\infty a_nc_n\)均收敛.
    \end{proof}

    \begin{problem}[10]\label{pb3.10}
        幂级数\(\sum_{n=1}^\infty a_nx_n\)的系数\(a_n \in \mathbb{Z}\),且其中有无穷多个不是\(0\).

        证明\(\sum_{n=1}^\infty a_nx_n\)的收敛半径最大为\(1\).
    \end{problem}

    \begin{proof}
        由于\(\{a_n\}\)含有无穷多非零元素,故\(\forall N \in \mathbb{Z}^+, \exists n>N, \abs*{a_n} \ne 0\).

        令\(a_{n_k}\)为第\(k\)个不为零的系数,构造子列\(\{a_{n_k}\}\).由于\(a_{n_k} \in \mathbb{Z}^+\),故\(\abs*{a_{n_k}} \geq 1\).

        于是\(\lim_{k \to \infty} a_{n_k} \geq 1\),即\(\limsup_{n \to \infty} a_n \geq 1\),从而\(R=1/\limsup_{n \to \infty} a_n \leq 1\).
    \end{proof}

    \begin{problem}[14]\label{pb3.14}
        对于序列\(\{a_n\}_{n \in \mathbb{Z}^+}\)构造\(\{S_n\}_{n \in \mathbb{Z}^+}, S_n=\sum_{k=1}^n a_k/n\).

        a.若\(\lim_{n \to \infty} a_n=a\),证明\(\lim_{n \to \infty} S_n=a\).

        b.构造序列\(\{a_n\}_{n \in \mathbb{Z}^+}\)使得\(\lim_{n \to \infty} S_n=0\)但\(\{a_n\}\)不收敛.

        c.构造序列\(\{a_n\}_{n \in \mathbb{Z}^+}\)使得\(\lim_{n \to \infty} S_n=0\)但\(\forall n \in \mathbb{Z}^+, a_n>0, \limsup_{n \to \infty} a_n=\infty\).

        d.令\(r_n=a_{n+1}-a_n\),证明若\(\lim_{n \to \infty} nr_n=0\),则\(\lim_{n \to \infty} S_n=\lim_{n \to \infty} a_n\).
    \end{problem}

    \begin{proof}[证明a]
        由于\(\lim_{n \to \infty} a_n=a\),则\(\forall \varepsilon>0, \exists N_1 \in \mathbb{Z}^+, \forall n>N_1, \abs*{a_1-a}<\varepsilon/2\).
        \begin{align*}
            \forall n>N_1, \abs*{S_n-a}=\abs*{\sum_{k=1}^n \frac{a_k-a}{n}} 
            \leq \abs*{\sum_{k=1}^{N_1} \frac{a_k-a}{n}}+\abs*{\sum_{k=N_1+1}^{n} \frac{a_k-a}{n}}
        \end{align*}
        令\(C=\sum_{k=1}^{N_1} (a_k-a)\).选取\(N_2 \in \mathbb{Z}^+\)使得\(\forall n>N_2, C/n<\varepsilon/2\),即\(N_2=\left\lceil 2C/\varepsilon \right\rceil\).

        选取\(N=\max \{N_1,N_2\}\).因此\(\forall \varepsilon>0, \exists N \in \mathbb{Z}^+, \forall n>N,\)
        \begin{align*}
            \abs*{\sum_{k=1}^{N_1} \frac{a_k-a}{n}}=\frac{C}{n}<\frac{\varepsilon}{2},
            \abs*{\sum_{k=N_1+1}^n \frac{a_k-a}{n}}<\abs*{\sum_{k=N_1+1}^n \frac{\varepsilon}{2n}}<\frac{(n-N_1)\varepsilon}{2n}<\frac{\varepsilon}{2}
        \end{align*}
        因此\(\abs*{S_n-a} \leq \abs*{\sum_{k=1}^{N_1} (a_k-a)/n}+\abs*{\sum_{k=N_1+1}^{n} (a_k-a)/n}<\varepsilon/2+\varepsilon/2=\varepsilon\).
    \end{proof}

    \begin{proof}[证明b]
        令\(a_n=(-1)^{n-1}\),显然\(a_n\)不收敛,但\(\lim_{n \to \infty} S_n=\lim_{n \to \infty} 1/n=0\).
    \end{proof}

    \begin{proof}[证明c]
        令\(a_n=n, \exists k \in \mathbb{Z}^+, n=2^k\),否则\(a_n=1/n^2\).显然\(\limsup_{n \to \infty} a_n=\infty\).
        \begin{align*}
            \frac{\sum_{k=1}^n a_k}{n} \leq \frac{1}{n} \left(\frac{\left\lfloor \log_2 n\right\rfloor (\left\lfloor \log_2 n\right\rfloor+1)}{2}+\sum_{k=1}^\infty \frac{1}{k^2}\right)
        \end{align*}
        由于\(\sum_{k=1}^\infty 1/k^2=\pi^2/6\)为有限值且\(\left\lfloor \log_2 n \right\rfloor \leq \log_2 n\),故\(S_n \leq (\log_2 n)^2/n+o((\log_2 n)^2/n)\).

        由于\(\lim_{n \to \infty} (\log_2 n)^2/n=0\),故\(\lim_{n \to \infty} S_n=0\).
    \end{proof}

    \begin{proof}[证明d]
        先证明\(a_n-S_n=\sum_{k=1}^{n-1} kr_k/n\),因为
        \begin{align*}
            \sum_{k=1}^{n-1} kr_k=(n-1)a_n-\sum_{k=1}^{n-1} a_k=na_n-\sum_{k=1}^n a_n, 
            \frac{\sum_{k=1}^{n-1} kr_k}{n}=a_n-\frac{\sum_{k=1}^n a_k}{n}
        \end{align*}
        视\(p_n=\sum_{k=1}^{n-1} kr_k/(n-1)\)为\(nr_n\)的平均值序列,同证明1得到\(\lim_{n \to \infty} nr_n=\lim_{n \to \infty} p_n=0\).

        然而\(\sum_{k=1}^{n-1} kr_k/n \leq \sum_{k=1}^{n-1} kr_k/(n-1)=p_n\)且\(\sum_{k=1}^{n-1} kr_k/n \geq 0\),故\(\lim_{n \to \infty} \sum_{k=1}^{n-1} kr_k/n=0\).

        因此\(\lim_{n \to \infty} (a_n-S_n)=\lim_{n \to \infty} nr_n=0\),即\(\lim_{n \to \infty} a_n=\lim_{n \to \infty} S_n\).
    \end{proof}

    \begin{proof}[证明c]
        由于\(a_n/(1+n^2 a_n)=1/(1/a_n+n^2) \leq 1/n^2\),故\(\sum_{n=1}^\infty a_n/(1+n^2 a_n) \leq \sum_{n=1}^\infty 1/n^2\).

        令\(a_n=1/n\),则\(\sum_{n=1}^\infty a_n\)发散,且\(\sum_{n=1}^\infty 1/(1+na_n)=\sum_{n=1}^\infty 1/2\)发散.

        令\(A=\{n \in \mathbb{Z}^+: \exists k \in \mathbb{Z}^+, n=k^2\}\).令\(a_n=1, n \in A; a_n=1/n^2, n \notin A\).
        \begin{align*}
            \sum_{n=1}^\infty \frac{a_n}{1+na_n}=\sum_{n \in A} \frac{1}{1+n}+\sum_{n \notin A} \frac{1}{n^2+n}
            \leq \sum_{k=1}^\infty \frac{1}{k^2+1}+\sum_{n=1}^\infty \frac{1}{n^2+n}<2\sum_{n=1}^\infty \frac{1}{n^2}
        \end{align*}
        因此\(\sum_{n=1}^\infty 1/(1+na_n)\)收敛,然而\(\sum_{n=1}^\infty a_n \geq \sum_{n \in A} a_n=\infty\)发散.
        
        故\(\sum_{n=1}^\infty a_n/(1+n a_n)\)既可能发散也可能收敛.
    \end{proof}
\end{comment}