\section{Numerical Sequences and Series}

\begin{problem}[13]\label{pb3.13}
    证明若\(\sum_{n=0}^\infty a_n, \sum_{n=0}^\infty b_n\)绝对收敛,则\(\sum_{n=0}^\infty c_n, c_n=\sum_{k=0}^n a_kb_{n-k}\)也绝对收敛.
\end{problem}

\begin{proof}
    \(\sum_{n=1}^N \abs*{c_n}=\sum_{n=1}^N \abs*{\sum_{k=0}^n a_kb_{n-k}} \leq \sum_{n=1}^N \sum_{k=0}^n \abs*{a_kb_{n-k}}\).
    \begin{align*}
        \sum_{n=0}^N \abs*{a_n} \cdot \sum_{n=0}^N \abs*{b_n}=\sum_{n=0}^{2N} \sum_{j+k=n}^{j,k \leq N} \abs*{a_jb_k}
        \geq \sum_{n=0}^N \sum_{j+k=n} \abs*{a_jb_k}=\sum_{n=0}^N \sum_{k=0}^n \abs*{a_kb_{n-k}} \geq \sum_{n=0}^N \abs*{c_n}
    \end{align*}
    由于\(\sum_{n=0}^\infty \abs*{a_n}, \sum_{n=0}^\infty \abs*{b_n}\)收敛,故\(\sum_{n=0}^N \abs*{c_n} \leq \sum_{n=0}^N \abs*{a_n} \cdot \sum_{n=0}^N \abs*{b_n}\)有界.

    结合\(\sum_{n=0}^N \abs*{c_n}\)的单调性,故\(\sum_{n=0}^\infty c_n\)绝对收敛.
\end{proof}

\begin{comment}
    \begin{problem}[4]\label{pb3.4}
        定义序列\(\{a_n\}\)为\(a_1=0, a_{2n}=a_{2n-1}/2, a_{2n+1}=1/2+a_{2n}\),求\(\{a_n\}\)的上下极限.
    \end{problem}

    \begin{proof}
        有\(a_{2n+1}=(a_{2n-1}+1)/2, a_{2n+1}-1=(a_{2n-1}-1)/2\).定义\(b_n=a_{2n-1}-1, b_{n+1}=b_n/2\).

        由于\(b_1=-1\),故\(b_n=-1/2^{n-1}, a_{2n-1}=1-1/2^{n-1}\),所以\(\lim_{n \to \infty} a_{2n-1}=1\).

        同时\(a_{2n}=a_{2n-1}/2=1/2-1/2^n\),所以\(\lim_{n \to \infty} a_{2n}=1/2\).

        最终\(\limsup_{n \to \infty} a_n=1, \liminf_{n \to \infty} a_n=1/2\).
    \end{proof}

    \begin{problem}[6]\label{pb3.6}
        考虑\(a_n=(\sqrt{n+1}-\sqrt{n})/n\)和\(a_n=(\sqrt[n]{n}-1)^n\)的敛散性.
    \end{problem}

    \begin{proof}
        \(a_n=(\sqrt{n+1}-\sqrt{n})/n\)收敛,注意到
        \begin{align*}
            a_n=\frac{\sqrt{n+1}-\sqrt{n}}{n} \leq \frac{\sqrt{n+1}}{n+1}-\frac{\sqrt{n}}{n}, 
            \sum_{n=1}^\infty a_n \leq \lim_{n \to \infty} \frac{\sqrt{n+1}}{n+1}=1
        \end{align*}
        \(a_n=(\sqrt[n]{n}-1)^n\)收敛,注意到\(\limsup_{n \to \infty} \sqrt[n]{a_n}=\limsup_{n \to \infty} (\sqrt[n]{n}-1)\).

        由于\(\sqrt[n]{n}-1=e^{\ln n/n}-1=O(\ln n/n)\),故\(\limsup_{n \to \infty} (\sqrt[n]{n}-1)=0\),级数收敛.
    \end{proof}

    \begin{problem}[7]\label{pb3.7}
        设\(\forall n \in \mathbb{Z}^+, a_n \geq 0\)且\(\sum_{n=1}^\infty a_n\)收敛,则\(\sum_{n=1}^\infty (\sqrt{a_n}/n)\)收敛.
    \end{problem}

    \begin{proof}
        由平均值不等式有\(\sqrt{a_n}/n=\sqrt{a_n \cdot 1/n^2} \leq a_n/2+1/2n^2\),
        
        故\(\sum_{n=1}^\infty (\sqrt{a_n}/n) \leq \sum_{n=1}^\infty a_n/2+\sum_{n=1}^\infty 1/2n^2\),且\(\sum_{n=1}^\infty a_n, \sum_{n=1}^\infty 1/n^2\)收敛.
    \end{proof}

    \begin{problem}[8]\label{pb3.8}
        设\(\sum_{n=1}^\infty a_n\)收敛且\(\{b_n\}\)单调有界,证明\(\sum_{n=1}^\infty a_nb_n\)收敛.
    \end{problem}

    \begin{proof}
        {\kaishu 不失一般性,设\(\{b_n\}\)单调增有上界}.设\(\sup b_n=b\),考虑\(c_n=b-b_n\).

        从而\(\sum_{n=1}^\infty a_nb_n=\sum_{n=1}^\infty a_n(b-c_n)=b \sum_{n=1}^\infty a_n-\sum_{n=1}^\infty a_nc_n\).

        显然\(\lim_{n \to \infty} c_n=0\)且\(\{c_n\}\)单调递减,故\(b \sum_{n=1}^\infty a_n, \sum_{n=1}^\infty a_nc_n\)均收敛.
    \end{proof}

    \begin{problem}[10]\label{pb3.10}
        幂级数\(\sum_{n=1}^\infty a_nx_n\)的系数\(a_n \in \mathbb{Z}\),且其中有无穷多个不是\(0\).

        证明\(\sum_{n=1}^\infty a_nx_n\)的收敛半径最大为\(1\).
    \end{problem}

    \begin{proof}
        由于\(\{a_n\}\)含有无穷多非零元素,故\(\forall N \in \mathbb{Z}^+, \exists n>N, \abs*{a_n} \ne 0\).

        令\(a_{n_k}\)为第\(k\)个不为零的系数,构造子列\(\{a_{n_k}\}\).由于\(a_{n_k} \in \mathbb{Z}^+\),故\(\abs*{a_{n_k}} \geq 1\).

        于是\(\lim_{k \to \infty} a_{n_k} \geq 1\),即\(\limsup_{n \to \infty} a_n \geq 1\),从而\(R=1/\limsup_{n \to \infty} a_n \leq 1\).
    \end{proof}

    \begin{problem}[14]\label{pb3.14}
        对于序列\(\{a_n\}_{n \in \mathbb{Z}^+}\)构造\(\{S_n\}_{n \in \mathbb{Z}^+}, S_n=\sum_{k=1}^n a_k/n\).

        a.若\(\lim_{n \to \infty} a_n=a\),证明\(\lim_{n \to \infty} S_n=a\).

        b.构造序列\(\{a_n\}_{n \in \mathbb{Z}^+}\)使得\(\lim_{n \to \infty} S_n=0\)但\(\{a_n\}\)不收敛.

        c.构造序列\(\{a_n\}_{n \in \mathbb{Z}^+}\)使得\(\lim_{n \to \infty} S_n=0\)但\(\forall n \in \mathbb{Z}^+, a_n>0, \limsup_{n \to \infty} a_n=\infty\).

        d.令\(r_n=a_{n+1}-a_n\),证明若\(\lim_{n \to \infty} nr_n=0\),则\(\lim_{n \to \infty} S_n=\lim_{n \to \infty} a_n\).
    \end{problem}

    \begin{proof}[证明a]
        由于\(\lim_{n \to \infty} a_n=a\),则\(\forall \varepsilon>0, \exists N_1 \in \mathbb{Z}^+, \forall n>N_1, \abs*{a_1-a}<\varepsilon/2\).
        \begin{align*}
            \forall n>N_1, \abs*{S_n-a}=\abs*{\sum_{k=1}^n \frac{a_k-a}{n}} 
            \leq \abs*{\sum_{k=1}^{N_1} \frac{a_k-a}{n}}+\abs*{\sum_{k=N_1+1}^{n} \frac{a_k-a}{n}}
        \end{align*}
        令\(C=\sum_{k=1}^{N_1} (a_k-a)\).选取\(N_2 \in \mathbb{Z}^+\)使得\(\forall n>N_2, C/n<\varepsilon/2\),即\(N_2=\left\lceil 2C/\varepsilon \right\rceil\).

        选取\(N=\max \{N_1,N_2\}\).因此\(\forall \varepsilon>0, \exists N \in \mathbb{Z}^+, \forall n>N,\)
        \begin{align*}
            \abs*{\sum_{k=1}^{N_1} \frac{a_k-a}{n}}=\frac{C}{n}<\frac{\varepsilon}{2},
            \abs*{\sum_{k=N_1+1}^n \frac{a_k-a}{n}}<\abs*{\sum_{k=N_1+1}^n \frac{\varepsilon}{2n}}<\frac{(n-N_1)\varepsilon}{2n}<\frac{\varepsilon}{2}
        \end{align*}
        因此\(\abs*{S_n-a} \leq \abs*{\sum_{k=1}^{N_1} (a_k-a)/n}+\abs*{\sum_{k=N_1+1}^{n} (a_k-a)/n}<\varepsilon/2+\varepsilon/2=\varepsilon\).
    \end{proof}

    \begin{proof}[证明b]
        令\(a_n=(-1)^{n-1}\),显然\(a_n\)不收敛,但\(\lim_{n \to \infty} S_n=\lim_{n \to \infty} 1/n=0\).
    \end{proof}

    \begin{proof}[证明c]
        令\(a_n=n, \exists k \in \mathbb{Z}^+, n=2^k\),否则\(a_n=1/n^2\).显然\(\limsup_{n \to \infty} a_n=\infty\).
        \begin{align*}
            \frac{\sum_{k=1}^n a_k}{n} \leq \frac{1}{n} \left(\frac{\left\lfloor \log_2 n\right\rfloor (\left\lfloor \log_2 n\right\rfloor+1)}{2}+\sum_{k=1}^\infty \frac{1}{k^2}\right)
        \end{align*}
        由于\(\sum_{k=1}^\infty 1/k^2=\pi^2/6\)为有限值且\(\left\lfloor \log_2 n \right\rfloor \leq \log_2 n\),故\(S_n \leq (\log_2 n)^2/n+o((\log_2 n)^2/n)\).

        由于\(\lim_{n \to \infty} (\log_2 n)^2/n=0\),故\(\lim_{n \to \infty} S_n=0\).
    \end{proof}

    \begin{proof}[证明d]
        先证明\(a_n-S_n=\sum_{k=1}^{n-1} kr_k/n\),因为
        \begin{align*}
            \sum_{k=1}^{n-1} kr_k=(n-1)a_n-\sum_{k=1}^{n-1} a_k=na_n-\sum_{k=1}^n a_n, 
            \frac{\sum_{k=1}^{n-1} kr_k}{n}=a_n-\frac{\sum_{k=1}^n a_k}{n}
        \end{align*}
        视\(p_n=\sum_{k=1}^{n-1} kr_k/(n-1)\)为\(nr_n\)的平均值序列,同证明1得到\(\lim_{n \to \infty} nr_n=\lim_{n \to \infty} p_n=0\).

        然而\(\sum_{k=1}^{n-1} kr_k/n \leq \sum_{k=1}^{n-1} kr_k/(n-1)=p_n\)且\(\sum_{k=1}^{n-1} kr_k/n \geq 0\),故\(\lim_{n \to \infty} \sum_{k=1}^{n-1} kr_k/n=0\).

        因此\(\lim_{n \to \infty} (a_n-S_n)=\lim_{n \to \infty} nr_n=0\),即\(\lim_{n \to \infty} a_n=\lim_{n \to \infty} S_n\).
    \end{proof}

    \begin{proof}[证明c]
        由于\(a_n/(1+n^2 a_n)=1/(1/a_n+n^2) \leq 1/n^2\),故\(\sum_{n=1}^\infty a_n/(1+n^2 a_n) \leq \sum_{n=1}^\infty 1/n^2\).

        令\(a_n=1/n\),则\(\sum_{n=1}^\infty a_n\)发散,且\(\sum_{n=1}^\infty 1/(1+na_n)=\sum_{n=1}^\infty 1/2\)发散.

        令\(A=\{n \in \mathbb{Z}^+: \exists k \in \mathbb{Z}^+, n=k^2\}\).令\(a_n=1, n \in A; a_n=1/n^2, n \notin A\).
        \begin{align*}
            \sum_{n=1}^\infty \frac{a_n}{1+na_n}=\sum_{n \in A} \frac{1}{1+n}+\sum_{n \notin A} \frac{1}{n^2+n}
            \leq \sum_{k=1}^\infty \frac{1}{k^2+1}+\sum_{n=1}^\infty \frac{1}{n^2+n}<2\sum_{n=1}^\infty \frac{1}{n^2}
        \end{align*}
        因此\(\sum_{n=1}^\infty 1/(1+na_n)\)收敛,然而\(\sum_{n=1}^\infty a_n \geq \sum_{n \in A} a_n=\infty\)发散.
        
        故\(\sum_{n=1}^\infty a_n/(1+n a_n)\)既可能发散也可能收敛.
    \end{proof}
\end{comment}

\begin{problem}[16]\label{pb3.16}
    设\(a>0, x_1>\sqrt{a}\),定义\(x_{n+1}=(x_n+a/x_n)/2\).证明\(\{x_n\}\)单调递减且收敛于\(\sqrt{a}\).
    
    定义误差\(\varepsilon_n=x_n-\sqrt{a}\),证明\(\varepsilon_{n+1}=\varepsilon_n^2/(2x_n)<\varepsilon_n^2/(2\sqrt{a})\).
\end{problem}

\begin{proof}
    设\(x_n>\sqrt{a}\),则\(x_{n+1}-\sqrt{a}=(x_n^2-2\sqrt{a}x_n+a)/(2x_n)=(x_n-\sqrt{a})^2/(2x_n)>0\).

    考虑\(x_{n+1}-x_n=(-x_n+a/x_n)/2\),显然\(a/x_n<\sqrt{a}, -x_n<-\sqrt{a}\),即\(x_{n+1}-x_n<0\).

    因此\(\lim_{n \to \infty} x_n=L\),故\(L=(L+a/L)/2\),即\(L=\sqrt{a}\).

    由于\(x_{n+1}-\sqrt{a}=\varepsilon_{n+1}, x_n-\sqrt{a}=\varepsilon_n\),代入上式即\(\varepsilon_{n+1}=\varepsilon_n^2/(2x_n)\).
\end{proof}

\begin{problem}[17]\label{pb3.17}
    设\(a>1, x_1>\sqrt{a}\),定义\(x_{n+1}=(a+x_n)/(1+x_n)\).证明\(\lim_{x \to \infty}x_n=\sqrt{a}\).
    
    定义误差\(\varepsilon_n=x_n-\sqrt{a}\),证明\(\varepsilon_{n+1}=((1-\sqrt{a})/(1+x_n))\varepsilon_n\).
\end{problem}

\begin{proof}
    下证\(\forall n \in \mathbb{Z}^+, \varepsilon_{n+1} \varepsilon_n<0\)且\(\abs*{\varepsilon_{n+1}}<\abs*{\varepsilon_n}\).
    \begin{align*}
        x_{n+1}-\sqrt{a}=\frac{a+x_n-\sqrt{a}-\sqrt{a} x_n}{1+x_n}
        =\frac{\sqrt{a}(1-\sqrt{a})+x_n(1-\sqrt{a})}{1+x_n}=\frac{1-\sqrt{a}}{1+x_n}(x_n-\sqrt{a})
    \end{align*}
    由于\(\abs*{\varepsilon_{n+1}/\varepsilon_n}=\abs*{(1-\sqrt{a})/(1+x_n)}<\abs*{(1-\sqrt{a})/(1+\sqrt{a})}<1\),
    
    故\(\lim_{n \to \infty} \varepsilon_n=0\),即\(\lim_{n \to \infty} x_n=a\).
\end{proof}

\begin{problem}[18]\label{pb3.18}
    设\(a>0, x_1>\sqrt[p]{a}\),定义\(x_{n+1}=\dfrac{p-1}{p} x_n+\dfrac{a}{p} x_n^{-p+1}\).
    
    证明\(\lim_{x \to \infty}x_n=\sqrt[p]{a}\),定义\(\varepsilon_n=x_n-\sqrt[p]{a}\),证明\(\varepsilon_{n+1}=O(\varepsilon_n^2)\).
\end{problem}

\begin{proof}
    考虑\(g(x)=\dfrac{p-1}{p} x+\dfrac{a}{p} x^{-p+1}=x, g'(x)=\dfrac{p-1}{p} (1-\dfrac{a}{x^p})\).

    令\(g(x)=x\),得到\(x_0=\sqrt[p]{a}, g'(x_0)=0\),因此\(\sqrt[p]{a}\)是\(g(x)\)的不动点.

    在\(\sqrt[p]{a}\)处展开\(x_{n+1}=g(x_n)=g(\sqrt[p]{a})+g'(\sqrt[p]{a})(x-\sqrt[p]{a})+O((x-\sqrt[p]{a})^2)\).

    由于\(g(\sqrt[p]{a})=\sqrt[p]{a}, g'(\sqrt[p]{a})=0\),故\(x_{n+1}-\sqrt[p]{a}=O((x_{n+1}-\sqrt[p]{a})^2)\).
\end{proof}

\newpage

\subsection*{Supplement problems}

\begin{problem}[1]\label{sp1}
    设\(\forall n \in \mathbb{Z}^+, a_n \geq 0\)且\(\sum_{n=1}^\infty a_n\)发散.定义\(S_n=\sum_{k=1}^n a_n\).

    a.证明\(\sum_{n=1}^\infty a_n/S_n\)发散.

    b.证明\(\forall \alpha \in (1,\infty), \sum_{n=1}^\infty a_n/S_n^\alpha\)收敛.
\end{problem}

\begin{comment}
    \begin{proof}[证明a]
        \(a_k/S_k \geq a_k/S_n\),故\(\sum_{k=m+1}^n a_k/S_k>\sum_{k=m}^n a_k/S_n=(S_n-S_m)/S_n=1-S_m/S_n\).

        由于\(\lim_{n \to \infty} S_n=\infty\),故\(\exists m_0, m_1, \dots \in \mathbb{Z}^+, \forall j \in \mathbb{Z}^+, S_{m_j} \geq 2S_{m_{j-1}}\),其中\(N_0=1\).
        \begin{align*}
            \sum_{k=1}^\infty \frac{a_n}{S_n}=\sum_{j=1}^\infty \sum_{k=m_{j-1}+1}^{m_j} \frac{a_n}{S_n} 
            \geq \sum_{j=1}^\infty 1-\frac{S_{m_j}}{S_{m_{j-1}}} \geq \sum_{j=1}^\infty 1-\frac{S_{m_{j-1}}}{2S_{m_{j-1}}}=\sum_{j=1}^\infty \frac{1}{2}=\infty
        \end{align*}
        因此\(\sum_{n=1}^\infty a_n/S_n\)发散.
    \end{proof}
\end{comment}

\begin{proof}[证明a]
    由于\(a_n/S_n \in (0,1)\),因此\(a_n/S_n>-\ln (1-a_n/S_n)\).
    \begin{align*}
        \frac{a_n}{S_n}>-\ln (1-\frac{a_n}{S_n})=\ln \frac{S_n}{S_{n-1}}, 
        \quad \sum_{k=1}^n \frac{a_k}{S_k}>\sum_{k=2}^n \ln \frac{S_k}{S_{k-1}}=\ln S_n-\ln S_1
    \end{align*}
    由于\(\sum_{n=1}^\infty a_n\)发散,故\(S_n \to \infty, \ln S_n-\ln S_1 \to \infty\),证毕.
\end{proof}

\begin{proof}[证明b]
    由于\(a_n/S_n^\alpha=(S_n-S_{n-1})/S_n^\alpha\),因此\(\forall x \in [S_{n-1},S_n], 1/x^\alpha \geq 1/S_n^\alpha\).
    \begin{align*}
        \frac{S_n-S_{n-1}}{S_n^\alpha}=\int_{S_{n-1}}^{S_n} \frac{dx}{S_n^\alpha} \leq \int_{S_{n-1}}^{S_n} \frac{dx}{x^\alpha}, \quad
        \sum_{n=1}^\infty \frac{a_n}{S_n^\alpha}=\int_{a_1}^\infty \frac{dx}{S_n^\alpha} \leq \int_{a_1}^\infty \frac{dx}{x^\alpha}<\infty
    \end{align*}
    因此\(\sum_{n=1}^\infty a_n/S_n^\alpha\)收敛.
\end{proof}

\begin{problem}[2]\label{sp2}
    设\(\forall n \in \mathbb{Z}^+, a_n \geq 0\)且\(\sum_{n=1}^\infty a_n\)收敛.定义\(r_n=\sum_{k=n}^\infty a_k\).

    a.证明\(\sum_{n=1}^\infty a_n/r_n\)发散.

    b.证明\(\forall \alpha \in (0,1), \sum_{n=1}^\infty a_n/r_n^\alpha\)收敛.
\end{problem}

\begin{comment}
    \begin{proof}[证明a]
        \(a_k/r_k>a_k/r_m\),故\(\sum_{k=m}^n a_k/r_k>\sum_{k=m}^n a_k/r_m=(r_m-r_n)/r_m=1-r_n/r_m\).

        由于\(\sum_{n=1}^\infty a_n\)收敛,故\(\exists m_0, m_1, \dots \in \mathbb{Z}^+, \forall j \in \mathbb{Z}^+, r_{m_j} \leq r_{m_{j-1}}/2\),其中\(N_0=1\).
        \begin{align*}
            \sum_{n=1}^\infty \frac{a_n}{r_n}=\sum_{j=1}^\infty \sum_{k=m_{j-1}+1}^{m_j} \frac{a_n}{r_n} 
            >\sum_{j=1}^\infty 1-\frac{r_{m_j}}{r_{m_{j-1}}} \geq \sum_{j=1}^\infty 1-\frac{r_{m_{j-1}}}{2r_{m_{j-1}}}=\sum_{n=1}^\infty \frac{1}{2}=\infty
        \end{align*}
        因此\(\sum_{n=1}^\infty a_n/r_n\)发散.
    \end{proof}
\end{comment}

\begin{proof}[证明a]
    由于\(a_n/r_n \in (0,1)\),因此\(a_n/r_n>-\ln (1-a_n/r_n)\).
    \begin{align*}
        \frac{a_n}{r_n}>-\ln (1-\frac{a_n}{r_n})=\ln \frac{r_n}{r_{n+1}}, 
        \quad \sum_{k=1}^n \frac{a_k}{r_k}>\sum_{k=1}^{n-1} \ln \frac{r_k}{r_{k+1}}=\ln r_1-\ln r_n
    \end{align*}
    由于\(\sum_{n=1}^\infty a_n\)收敛,故\(r_n \to 0, \ln r_1-\ln r_n \to \infty\),证毕.
\end{proof}

\begin{proof}[证明b]
    由于\(a_n/r_n^\alpha=(r_n-r_{n+1})/r_n^\alpha\),因此\(\forall x \in [r_{n+1},r_n], 1/x^\alpha \geq 1/S_n^\alpha\).
    \begin{align*}
        \frac{r_{n+1}-r_n}{r_n^\alpha}=\int_{r_{n+1}}^{r_n} \frac{dx}{r_n^\alpha} \leq \int_{r_{n+1}}^{r_n} \frac{dx}{x^\alpha}, \quad
        \sum_{n=1}^\infty \frac{a_n}{r_n^\alpha}=\int_0^{r_1} \frac{dx}{r_n^\alpha} \leq \int_0^{r_1} \frac{dx}{x^\alpha}<\infty
    \end{align*}
    因此\(\sum_{n=1}^\infty a_n/r_n^\alpha\)收敛.
\end{proof}

\newpage

\begin{problem}[3]\label{sp3}
    给定序列\(\{u_n\}, \{S_n\}\),其中\(u_n>0, S_1=1, 2S_{n+1}=S_n+\sqrt{S_n^2+u_n}\).
    
    证明\(\sum_{n=1}^\infty u_n\)收敛等价于\(\lim_{n \to \infty} S_n<\infty\).
\end{problem}

\begin{proof}
    \(2S_{n+1}=S_n+\sqrt{S_n^2+u_n}\)等价于\(u_n=4S_{n+1}(S_{n+1}-S_n)\).
    
    先设\(\lim_{n \to \infty} S_n=S<\infty\).由于\(2S_{n+1}=S_n+\sqrt{S_n^2+u_n}>2S_n\),故\(S_n<S\).
    \begin{align*}
        u_n=4S_{n+1}(S_{n+1}-S_n)<4S(S_{n+1}-S_n), \sum_{n=1}^\infty u_n<4S(S-S_1)<\infty
    \end{align*}
    再设\(\sum_{n=1}^\infty u_n=U<\infty\).由于\(S_n \geq S_1\),从而
    \begin{align*}
        S_{n+1}-S_n=\frac{u_n}{4S_{n+1}} \leq \frac{u_n}{4}, 
        \sum_{n=1}^\infty (S_{n+1}-S_n) \leq \frac{U}{4}<\infty
    \end{align*}
    因此\(\lim_{n \to \infty} S_n=S_1+\sum_{n=1}^\infty (S_{n+1}-S_n)<\infty\),证毕.
\end{proof}

\begin{problem}[4]\label{sp4}
    设\(a>0, a_1>\sqrt{a}\),定义\(a_{n+1}=(a_n+a/a_n)/2\). 证明\(\sum_{n=1}^\infty ((a_n/a_{n+1})^2-1)\)收敛.
\end{problem}

\begin{proof}
    由于\(a_{n+1}=(a_n+a/a_n)/2 \geq \sqrt{a_n \cdot a/a_n}=\sqrt{a}\),因而
    \begin{align*}
        \left(\frac{a_n}{a_{n+1}}\right)^2-1=\frac{a_n^2-a_{n+1}^2}{a_{n+1}^2} \leq \frac{a_n^2-a_{n+1}^2}{a}, 
        \sum_{n=1}^\infty \left(\left(\frac{a_n}{a_{n+1}}\right)^2-1\right) \leq \lim_{n \to \infty} \frac{a_1^2-a_{n+1}^2}{a}
    \end{align*}
    由于\(\lim_{n \to \infty} (a_1^2-a_{n+1}^2)/a=a_1^2/a-1<\infty\),级数收敛.
\end{proof}

\begin{problem}[5]\label{sp5}
    设\(a>1, p_n>0\)且单调递增,证明\(\sum_{n=1}^\infty \dfrac{p_{n+1}-p_n}{p_{n+1}p_n^a}\)收敛.
\end{problem}

\begin{proof}
    由于\(p_n>0\)单调递增且\(a>1\),故\(p_n^{1-a}\)单调递减且有上界\(p_1^{1-a}\).
    \begin{align*}
        \sum_{n=1}^\infty \frac{p_{n+1}-p_n}{p_{n+1}p_n^a}=\sum_{n=1}^\infty \frac{p_{n+1}-p_n}{p_{n+1}p_n} p_n^{1-a}\leq p_1^{1-a} \sum_{n=1}^\infty \left(\frac{1}{p_n}-\frac{1}{p_{n+1}}\right)=p_1^{1-a} \lim_{n \to \infty} \left(\frac{1}{p_1}-\frac{1}{p_{n+1}}\right)
    \end{align*}
    由于\(1/p_n\)单调递减且有下界\(0\),故\(p_1^{1-a} \lim_{n \to \infty} (1/p_1-1/p_{n+1})\)存在,级数收敛.
\end{proof}

\begin{problem}[6]\label{sp6}
    若\(p \geq 1\),证明\(\sum_{n=1}^\infty \dfrac{1}{(n+1)\sqrt[p]{n}}<p\).
\end{problem}

\begin{proof}
    设\(f(x)=x^{1/p}, x \in [n,n+1]\),于是\(f'(x)=x^{1/p-1}/p\).
    \begin{align*}
        p\left(\frac{1}{\sqrt[p]{n}}-\frac{1}{\sqrt[p]{(n+1)}}\right)=p\frac{\sqrt[p]{n+1}-\sqrt[p]{n}}{\sqrt[p]{n}\sqrt[p]{n+1}}=\frac{pf'(\xi)}{\sqrt[p]{n}\sqrt[p]{n+1}}>\frac{(n+1)^{1/p-1}}{\sqrt[p]{n}\sqrt[p]{n+1}}=\frac{1}{\sqrt[p]{n}(n+1)}
    \end{align*}
    其中\(\xi \in (n,n+1)\),因此\(\sum_{n=1}^\infty \dfrac{1}{(n+1)\sqrt[p]{n}}<\sum_{n=1}^\infty p(n^{-1/p}-(n+1)^{-1/p})=p\),证毕.
\end{proof}

\newpage

\begin{problem}[7]\label{sp7}
    若正项级数\(\sum_{n=1}^\infty 1/p_n\)收敛,令\(S_n=\sum_{k=1}^n p_k\),证明\(\sum_{n=1}^\infty \dfrac{n^2p_n}{S_n^2}\)收敛.
\end{problem}

\begin{proof}
    设\(\sum_{n=1}^\infty 1/p_n=C_1, \sum_{n=1}^\infty 1/S_n=C_2\).由分部求和得到
    \begin{align*}
        \sum_{n=1}^\infty \frac{n^2p_n}{S_n^2} &=\sum_{n=1}^\infty \frac{n^2(S_n-S_{n-1})}{S_n^2} 
        \leq \frac{1}{p_1}+\sum_{n=2}^\infty \dfrac{n^2(S_n-S_{n-1})}{S_nS_{n-1}}
        =\frac{1}{p_1}+\sum_{n=2}^\infty \left(\frac{n^2}{S_{n-1}}-\frac{n^2}{S_n}\right) \\
        &=\frac{1}{p_1}+\frac{4}{p_1}+\sum_{n=2}^\infty \frac{1}{S_n}+\sum_{n=2}^\infty \frac{2n}{S_n}
        =\frac{2}{p_1}+C_2+2\sum_{n=1}^\infty \frac{n}{S_n}
    \end{align*}
    最后两步等式是因为\(1/p_1=1/S_1\).由柯西不等式得到
    \begin{align*}
        \left(\sum_{n=1}^\infty \frac{n}{S_n}\right)^2=\left(\sum_{n=1}^\infty \left(\frac{n \sqrt{p_n}}{S_n} \cdot \frac{1}{\sqrt{p_n}}\right)\right)^2 \leq \sum_{n=1}^\infty \frac{n^2p_n}{S_n^2} \cdot \sum_{n=1}^\infty \frac{1}{p_n}=C_1\sum_{n=1}^\infty \frac{n^2p_n}{S_n^2}
    \end{align*}
    注意到等号右侧出现了原级数,于是
    \begin{align*}
        \left(\sum_{n=1}^\infty \frac{n}{S_n}\right)^2 \leq C_1\sum_{n=1}^\infty \frac{n^2p_n}{S_n^2} \leq C_1\left(\frac{2}{p_1}+C_2+2\sum_{n=1}^\infty \frac{n}{S_n}\right)=\frac{2C_1}{p_1}+C_1C_2+2C_1\sum_{n=1}^\infty \frac{n}{S_n}
    \end{align*}
    于是可以解出\(\sum_{n=1}^\infty n/S_n\)的上界,解得
    \begin{align*}
        \sum_{n=1}^\infty \frac{n}{S_n} \leq \sqrt{C_1^2+C_1C_2+\frac{2C_1}{p_1}}+C_1 \leq 3C_1+C_2, \sum_{n=1}^\infty \frac{n^2p_n}{S_n^2} \leq \frac{2}{p_1}+8C_1+3C_2
    \end{align*}
    因此原级数有上界\(2/p_1+8C_1+3C_2\).
\end{proof}

\begin{problem}[8]\label{sp8}
    设\(\{p_n\}\)单调递增且恒正,证明\(\sum_{n=1}^\infty 1/p_n\)和\(\sum_{n=1}^\infty n/\sum_{k=1}^n p_k\)同敛散.
\end{problem}

\begin{proof}
    若\(\sum_{n=1}^\infty (n/\sum_{k=1}^n p_k)\)收敛,则\(n/\sum_{k=1}^n p_k \geq n/(np_n)=1/p_n\),故\(\sum_{n=1}^\infty 1/p_n\)收敛.

    若\(\sum_{n=1}^\infty 1/p_n\)收敛,令\(q_n=n/\sum_{k=1}^{2n} p_k\).下证\(q_{n+1} \leq q_n\).
    \begin{align*}
        (q_{n+1}-q_n)\sum_{k=1}^n p_k \sum_{k=1}^{n+1} p_k=np_{n+1}-\sum_{k=1}^n p_k=\sum_{k=1}^n (p_{n+1}-p_k) \geq 0
    \end{align*}
    由于\(2n/\sum_{k=1}^{2n} p_k \leq 2n\sum_{k=n+1}^{2n} p_k \leq 2n/(2np_n)=1/p_n\),故\(1/q_{2n} \leq 1/p_n\).

    由于\(q_{2n+1} \leq q_{2n} \leq 1/p_n\),故\(\sum_{k=1}^{2n} q_{2n}=\sum_{k=1}^{2n} k/\sum_{j=1}^k p_j \leq 2\sum_{k=1}^n 1/p_k\),
    
    故级数有上界\(2\sum_{n=1}^\infty 1/p_n\).(令\(p_n=1/n^\alpha, \alpha>1\),即可得到\(\sum_{n=1}^\infty n/\sum_{k=1}^n k^\alpha\)收敛.)
\end{proof}

\newpage

\begin{problem}[9]\label{sp9}
    证明\(\sum_{n=1}^\infty \dfrac{\sum_{k=1}^n 1/k}{(n+1)(n+2)}\)收敛并求其和.
\end{problem}

\begin{proof}
    令\(\sum_{k=1}^n 1/k=a_n\),则当\(n\)足够大时\(a_n<1+\int_1^n 1/x dx=1+\ln n<\sqrt{n}\),
    
    故\(\sum_{n=1}^\infty \dfrac{a_n}{(n+1)(n+2)}<\sum_{n=1}^\infty \dfrac{\sqrt{n}}{(n+1)(n+2)}<\dfrac{\sqrt{n}}{n^2}=n^{-3/2}\),该级数收敛.

    进一步,由于\(1/((n+1)(n+2))=1/(n+1)-1/(n+2)\),故
    \begin{align*}
        \sum_{n=1}^\infty \frac{a_n}{(n+1)(n+2)}=\sum_{n=1}^\infty \left(\frac{a_n}{n+1}-\frac{a_n}{n+2}\right)=\sum_{n=1}^\infty \frac{a_n-a_{n-1}}{n+1}=\sum_{n=1}^\infty \frac{1}{n(n+1)}=1
    \end{align*}
    其中最后一步是因为\(1/(n(n+1))=1/n-1/(n+1)\).
\end{proof}

\begin{problem}[10]\label{sp10}
    设\(\{a_n\}, \{b_n\}\)各项均为正且\(\exists c>0, \forall n \in \mathbb{Z}^+, b_{n+1}-b_n \geq c\).

    证明若\(\sum_{n=1}^\infty a_n\)收敛,则\(\sum_{n=1}^\infty \dfrac{n \sqrt[n]{\prod_{k=1}^n a_kb_k}}{b_{n+1}b_n}\)收敛.
\end{problem}

\begin{proof}
    由均值不等式\(\sqrt[n]{\prod_{k=1}^n a_kb_k} \leq \dfrac{\sum_{k=1}^n a_kb_k}{n}\),故\(\sum_{n=1}^\infty \dfrac{n \sqrt[n]{\prod_{k=1}^n a_kb_k}}{b_{n+1}b_n} \leq \sum_{n=1}^\infty \dfrac{\sum_{k=1}^n a_kb_k}{b_{n+1}b_n}\).

    令\(S_n=\sum_{k=1}^n a_kb_k\).由于\(\dfrac{1}{b_nb_{n+1}}=\dfrac{1}{b_{n+1}-b_n}\left(\dfrac{1}{b_n}-\dfrac{1}{b_{n+1}}\right) \leq \dfrac{1}{c}\left(\dfrac{1}{b_n}-\dfrac{1}{b_{n+1}}\right)\),故
    \begin{align*}
        \sum_{n=1}^\infty \dfrac{S_n}{b_{n+1}b_n} \leq \frac{1}{c} \sum_{n=1}^\infty \left(\dfrac{S_n}{b_n}-\dfrac{S_n}{b_{n+1}}\right)=\frac{1}{c} \sum_{n=1}^\infty \frac{S_n-S_{n-1}}{b_n}=\frac{1}{c} \sum_{n=1}^\infty \frac{a_nb_n}{b_n}=\frac{1}{c} \sum_{n=1}^\infty a_n
    \end{align*}
    因此\(\sum_{n=1}^\infty \dfrac{n \sqrt[n]{\prod_{k=1}^n a_kb_k}}{b_{n+1}b_n} \leq \sum_{n=1}^\infty \dfrac{\sum_{k=1}^n a_kb_k}{b_{n+1}b_n} \leq \dfrac{\sum_{n=1}^\infty a_n}{c}\),由于\(\sum_{n=1}^\infty a_n\)收敛,证毕.
\end{proof}

\begin{problem}[11]\label{sp11}
    设\(\{a_n\}, \{b_n\}\)满足\(a_n, b_n>0, a_1=b_1=1, b_n=a_nb_{n-1}-2\).

    设\(\{b_n\}\)是有界序列,证明\(\sum_{n=1}^\infty (\prod_{k=1}^n a_k)^{-1}\)收敛并求其和.
\end{problem}

\begin{proof}
    令\(S_n=\prod_{k=1}^n a_k\).由于\((a_nb_{n-1}-b_n)/2=1\),故
    \begin{align*}
        \sum_{n=1}^\infty \frac{1}{S_n}=\frac{1}{a_1}+\frac{1}{2}\sum_{n=2}^\infty \frac{a_nb_{n-1}-b_n}{S_n}=1+\frac{1}{2} \sum_{n=1}^\infty \left(\frac{b_{n-1}}{S_{n-1}}-\frac{b_n}{S_n}\right)=\frac{3}{2}-\frac{1}{2}\lim_{n \to \infty} \frac{b_n}{S_n}
    \end{align*}
    设\(b_n\)上界为\(M\),有
    \begin{align*}
        a_n=\left(1+\frac{2}{b_n}\right)\frac{b_n}{b_{n-1}}, \frac{b_n}{S_n}=\frac{1}{\prod_{k=2}^n (1+2/b_k)} \leq \frac{1}{\prod_{k=2}^n (1+2/M)}=\left(1+\frac{2}{M}\right)^{1-n}
    \end{align*}
    因此\(0 \leq \lim_{n \to \infty} b_n/S_n \leq \lim_{n \to \infty} (1+2/M)^{-n+1}=0\),得到\(\sum_{n=1}^\infty (\prod_{k=1}^n a_k)^{-1}=3/2\).
\end{proof}

\newpage

\begin{problem}[12]\label{sp12}
    定义\(B(n)\)为\(n\)的二进制展开中含有的\(1\)的个数.求和\(\sum_{n=1}^\infty \dfrac{B(n)}{n(n+1)}\).
\end{problem}

\begin{proof}
    首先\(B(2n)=B(n), B(2n+1)=B(n)+1\).由此将原级数拆分为奇数项和偶数项.
    \begin{align*}
        \sum_{n=1}^\infty \frac{B(n)}{n(n+1)}&=\sum_{n=0}^\infty \frac{B(2n+1)}{(2n+1)(2n+2)}+\sum_{n=1}^\infty \frac{B(2n)}{2n(2n+1)} \\
        &=\frac{1}{2}+\sum_{n=1}^\infty \frac{B(n)+1}{(2n+1)(2n+2)}+\sum_{n=1}^\infty \frac{B(n)}{2n(2n+1)} \\
        &=\frac{1}{2}+\sum_{n=1}^\infty \frac{1}{(2n+1)(2n+2)}+\sum_{n=1}^\infty B(n)\left(\dfrac{1}{(2n+1)(2n+2)}+\frac{1}{2n(2n+1)}\right) \\
        &=\sum_{n=0}^\infty \left(\frac{1}{2n+1}-\frac{1}{2n+2}\right)+\sum_{n=1}^\infty B(n)\left(\frac{1}{2n}-\frac{1}{2n+2}\right) \\
        &=\sum_{n=1}^\infty \frac{(-1)^{n-1}}{n}+\frac{1}{2}\sum_{n=1}^\infty \frac{B(n)}{n(n+1)}
    \end{align*}
    由于\(\ln (1+x)=\sum_{n=1}^\infty \dfrac{(-1)^{n-1} x}{n}\),故\(\sum_{n=1}^\infty \dfrac{B(n)}{n(n+1)}=2\ln 2=\ln 4\).
\end{proof}

\begin{problem}[13]\label{sp13}
    设\(\{a_n\}\)满足\(\forall k \in [n,2n], a_k \in [0,Ma_n]\),且\(\sum_{n=1}^\infty a_n\)收敛,证明\(\lim_{n \to \infty} na_n=0\).
\end{problem}

\begin{proof}
    由于\(\sum_{n=1}^\infty a_n\)收敛,令\(S_n=\sum_{k=1}^n a_n\),有\(\lim_{n \to \infty} (S_{2n}-S_n)=0\).

    由于\(\forall k=n+1, \dots, 2n, a_n \leq Ma_k\),故\(S_{2n}-S_n=\sum_{k=n+1}^{2n} a_k \geq Mna_n\).

    因此\(0 \leq \lim_{n \to \infty} na_n \leq \lim_{n \to \infty} (S_{2n}-S_n)/M=0\),即\(\lim_{n \to \infty} na_n=0\).
\end{proof}

\begin{problem}[14]\label{sp14}
    若正项级数\(\sum_{n=1}^\infty a_n\)收敛,证明\(\sum_{n=1}^\infty a_n^{n/(n+1)}\)收敛.
\end{problem}

\begin{proof}
    令\(A=\{n \in \mathbb{Z}^+: a_n^{n/(n+1)} \leq 2a_n\}\),若\(n \notin A\),则\(a_n^{n/(n+1)}<1/2^n\).
    \begin{align*}
        \sum_{n=1}^\infty a_n^{n/(n+1)}=\sum_{n \in A} a_n^{n/(n+1)}+\sum_{n \notin A} a_n^{n/(n+1)} \leq 2\sum_{n=1}^\infty a_n+\sum_{n=1}^\infty \frac{1}{2^n}=2\sum_{n=1}^\infty a_n+1
    \end{align*}
    因此原级数有上界\(2\sum_{n=1}^\infty a_n+1\),证毕.
\end{proof}

\begin{problem}[15]\label{sp15}
    求\(\sum_{n=1}^\infty \arctan(2/n^2)\).
\end{problem}

\begin{proof}
    由于\(\arctan(x)-\arctan(y)=\arctan\left(\dfrac{x-y}{1+xy}\right)\),故
    \begin{align*}
        \arctan\left(\frac{2}{n^2}\right)=\arctan\left(\frac{(n+1)-(n-1)}{1+(n+1)(n-1)}\right)=\arctan(n+1)-\arctan(n-1)
    \end{align*}
    \(\sum_{n=1}^\infty \arctan(2/n^2)=\lim_{n \to \infty} \arctan(n+1)-(\arctan 1+\arctan 0)=\pi/2-\pi/4=\pi/4\).
\end{proof}

\newpage

\begin{problem}[16]\label{sp16}
    设\(u_n=\dfrac{1}{\left(\sqrt{n+1}+\sqrt{n}\right)^p}\ln\left(\dfrac{n+1}{n-1}\right), p>0\),证明\(\sum_{n=2}^\infty u_n\)收敛.
\end{problem}

\begin{proof}
    考虑解析函数\(f(x)=\sqrt{1+x}-1, g(x)=\ln\left(\dfrac{1+x}{1-x}\right)\).
    
    由于\(f,g\)在\(x=0\)处解析,即存在\(\delta>0\)使得\(\forall x \in B(0,\delta)\)有
    \begin{align*}
        f \in O(x), \quad g \in (x+O(x^3))-(-x+O(x^3))=2x+O(x^3)
    \end{align*}
    取\(x=1/n\),由于\(1/n \to 0 \in B(0, \delta)\),故\(\forall n>\left\lceil 1/\delta \right\rceil\),有
    \begin{align*}
        u_n&=\left(\sqrt{n+1}+\sqrt{n}\right)^{-p}\ln\left(\dfrac{1+1/n}{1-1/n}\right)=\left(\sqrt{n+1}-\sqrt{n}\right)^p \ln\left(\dfrac{1+1/n}{1-1/n}\right) \\
        &=n^{p/2} \cdot \left(f\left(\frac{1}{n}\right)\right)^p \cdot g\left(\frac{1}{n}\right)=n^{p/2} \cdot O\left(\frac{1}{n^p}\right) \cdot \left(\frac{2}{n}+O\left(\frac{1}{n^3}\right)\right)=O\left(\frac{1}{n^{p/2+1}}\right)
    \end{align*}
    由于\(\sum_{n=1}^\infty n^{-p/2-1}\)绝对收敛,故级数收敛.
\end{proof}

\begin{problem}[17]\label{sp17}
    讨论\(\sum_{n=1}^\infty u_n, u_n=\dfrac{1}{n^p} \left(1-\dfrac{x \ln n}{n}\right)^n\)的收敛性和\(p,x\)的关系.
\end{problem}

\begin{proof}
    考虑解析函数\(f(x)=\ln(1-x), g(x)=e^x\),则\(\exists \delta>0, \forall x \in B(0,\delta)\)有
    \begin{align*}
        f(x)=-x+O(x^2), \quad g(x)=1+O(x)
    \end{align*}
    由于\(\left(1-\dfrac{x \ln n}{n}\right)^n=g\left(nf\left(\dfrac{x \ln n}{n}\right)\right)\),故
    \begin{align*}
        u_n=\frac{1}{n^p} g\left(nf\left(\dfrac{x \ln n}{n}\right)\right)=\frac{1}{n^p} g\left(-x \ln n+O\left(\frac{\ln^2 n}{n}\right)\right)=\frac{1}{n^{p+x}}\left(1+O\left(\frac{\ln^2 n}{n}\right)\right)
    \end{align*}
    \(\exists N \in \mathbb{Z}^+\)使得\(\forall n>N, \max\left(\dfrac{x \ln n}{n}, \dfrac{\ln^2 n}{n}\right)<\delta, \dfrac{1}{n^p} \left(1-\dfrac{x \ln n}{n}\right)^n=\dfrac{1}{n^{p+x}}+O\left(\dfrac{\ln^2 n}{n^{p+x+1}}\right)\).
    
    因此级数收敛的充要条件是\(p+x>1\).
\end{proof}

\newpage

\begin{problem}[18]\label{sp18}
    设\(u_n=\int_0^1 1/(1+t^4)^n dt\). 
    
    a.证明\(\lim_{n \to \infty} u_n\)存在. b.证明\(\sum_{n=1}^\infty u_n\)发散. c.证明\(\sum_{n=1}^\infty u_n/n\)收敛并求其和.
\end{problem}

\begin{proof}[证明a]
    将积分限\([0,1]\)划分为\([0,n^{-1/8}]\)和\([n^{-1/8},1]\).
    \begin{align*}
        \lim_{n \to \infty} \int_{n^{-8}}^1 \frac{dt}{(1+t^4)^n} \leq \lim_{n \to \infty} \frac{1}{(1+1/\sqrt{n})^n}=0, \lim_{n \to \infty} \int_0^{n^{-8}} \frac{dt}{(1+t^4)^n} \leq \lim_{n \to \infty} \frac{1}{\sqrt[8]{n}}=0
    \end{align*}
    而\(\forall n \in \mathbb{Z}^+, u_n \geq 0\),因此\(\lim_{n \to \infty} u_n=0\).
\end{proof}

\begin{proof}[证明b]
    \(u_n=\int_0^1 1/(1+t^4)^n dt \geq \int_0^1 1/(1+t)^n dt=(1-2^{n-1})/(n-1)\),

    由于\(\forall n \geq 2, (1-2^{n-1})/(n-1) \geq 1/2(n-1)\),故\(\sum_{n=1}^\infty u_n \geq 1/2\sum_{n=2}^\infty 1/n\)发散.
\end{proof}

\begin{proof}[证明c]
    使用分部积分法.
    \begin{align*}
        \int_0^1 \frac{dt}{(1+t^4)^n}&=\left.\frac{t}{(1+t^4)^n}\right|_0^1-\int_0^1 t d\frac{1}{(1+t^4)^n}=\frac{1}{2^n}+4n \int_0^1 \frac{t^4}{(1+t^4)^{n+1}} dt \\
        &=\frac{1}{2^n}+4n \left(\int_0^1 \frac{1}{(1+t^4)^n} dt-\int_0^1 \frac{1}{(1+t^4)^{n+1}} dt\right)=\frac{1}{2^n}+4n(u_n-u_{n+1}), \\
        \sum_{n=1}^\infty \frac{u_n}{n}&=\sum_{n=1}^\infty \frac{1}{n2^n}+\sum_{n=1}^\infty (u_n-u_{n+1})=\sum_{n=1}^\infty \frac{1}{n2^n}+\int_0^1 \frac{1}{1+t^4} dt
    \end{align*}
    由于\(\ln(1+x)=\sum_{n=1}^\infty (-1)^{n-1} x^n/n\),取\(x=-1/2\)得到\(\sum_{n=1}^\infty 1/(n2^n)=-\ln(1/2)=\ln 2\).

    由于\(1+t^4=(t^2+\sqrt{2}x+1)(t^2-\sqrt{2}x+1)\),故
    \begin{align*}
        \int_0^1 \frac{dt}{1+x^4}&=\frac{1}{2\sqrt{2}} \int_0^1 \left(\frac{-x+\sqrt{2}}{x^2-\sqrt{2}x+1}+\frac{x+\sqrt{2}}{x^2+\sqrt{2}x+1}\right) dt \\
        &=\left[\frac{1}{4\sqrt{2}}\ln \abs*{\frac{x^2+\sqrt{2}x+1}{x^2-\sqrt{2}x+1}}+\frac{1}{2\sqrt{2}} \arctan\left(\frac{\sqrt{2}x}{1-x^2}\right)\right]_0^1=\frac{\sqrt{2}}{4} \ln(\sqrt{2}+1)+\frac{\sqrt{2} \pi}{8}
    \end{align*}
    因此\(\sum_{n=1}^\infty u_n/n=\ln 2+\dfrac{\sqrt{2}}{4} \ln(\sqrt{2}+1)+\dfrac{\sqrt{2} \pi}{8}\).
\end{proof}

\begin{problem}[19]\label{sp19}
    证明\(\sum_{n=1}^\infty \int_0^{1/n} \sqrt{x}/(1+x^4) dx\)收敛.
\end{problem}

\begin{proof}
    由于\(\exists \delta>0, \forall x \in B(0,\delta), 1/(1+x^4)=1+O(x^4)\),故\(\forall n>\left\lceil \delta \right\rceil, 1/n<\delta\),有
    \begin{align*}
        \sum_{n=1}^\infty \int_0^{1/n} \frac{\sqrt{x}}{1+x^4} dx=\sum_{n=1}^\infty \int_0^{1/n} \left(x^{1/2}+O(x^{9/2})\right) dx=\sum_{n=1}^\infty \frac{2}{3n^{3/2}}+\sum_{n=1}^\infty O\left(\frac{1}{n^{11/2}}\right)
    \end{align*}
    主项和余项都绝对收敛,原级数收敛.
\end{proof}

\newpage

\begin{problem}[20]\label{sp20}
    证明\(\sum_{n=1}^\infty \int_0^{\pi/4} \cos^n x dx\)发散.
\end{problem}

\begin{proof}
    由于存在\(\int_0^{\pi/4} \cos^n x dx \leq \int_0^{\pi/4} 1 dx=\pi/4<\infty\),故根据控制收敛定理有
    \begin{align*}
        \sum_{n=1}^\infty \int_0^{\pi/4} \cos^n x dx&=\int_0^{\pi/4} \sum_{n=1}^\infty \cos^n x dx=\int_0^{\pi/4} \frac{\cos x}{1-\cos x} dx \\
        &=-\frac{\pi}{4}+\int_0^{\pi/4} \frac{1}{2\sin^2 (x/2)} dx=-\frac{\pi}{4}-\left.\cot x\right|_0^{\pi/8}=\infty \qedhere
    \end{align*}
\end{proof}

\begin{comment}
    \begin{proof}[方法2]
        将积分限\([0,\pi/4]\)划分为\([0,n^{-1/3}]\)和\([n^{-1/3},\pi/4]\).
        \begin{align*}
            \int_{n^{-1/3}}^{\pi/4} \cos^n x dx &\leq \frac{\pi}{4} \cos^n \left(n^{-1/3}\right)=\frac{\pi}{4} \exp\left(n \ln\left(\cos \left(n^{-1/3}\right)\right)\right) \\
            &=\frac{\pi}{4} \exp\left(n \ln\left(1-\frac{n^{-2/3}}{2}+O\left(n^{-4/3}\right)\right)\right)=\frac{\pi}{4}\exp\left(-\frac{n^{1/3}}{2}+O\left(n^{-1/3}\right)\right) \\
            \int_0^{n^{-1/3}} \cos^n x dx &=\int_0^{n^{-1/3}} \exp(n \ln(\cos x)) dx=\int_0^{n^{-1/3}} \exp\left(n \ln\left(1-\frac{x^2}{2}+O(x^4)\right)\right) dx \\
            &=\int_0^{n^{-1/3}} \exp\left(-n\left(\frac{x^2}{2}+O(x^4)\right)\right) dx
        \end{align*}
        令\(u=\sqrt{n} x\),积分限变为\([0,n^{1/6}]\),得到
        \begin{align*}
            \int_0^{n^{-1/3}} \exp\left(-n\left(\frac{x^2}{2}+O(x^4)\right)\right) dx=\frac{1}{\sqrt{n}} \int_0^{n^{1/6}} \exp\left(-\frac{u^2}{2}+O\left(\frac{u^4}{n}\right)\right) du
        \end{align*}
    \end{proof}
\end{comment}

\begin{problem}[21]\label{sp21}
    设\(a_n=\sum_{k=2}^{n-1} 1/(n-k)\ln k\),给出\(\lim_{n \to \infty} a_n\).
\end{problem}

\begin{proof}
    {\kaishu 和式在头尾各有一个奇点,故将求和各项平分为前后两段.}

    考虑\(\sum_{k=2}^{\left\lfloor n/2 \right\rfloor} 1/(n-k)\ln k\),则\(\forall k=2, \dots, \left\lfloor n/2 \right\rfloor\)都有
    \begin{align*}
        \sum_{k=2}^{\left\lfloor n/2 \right\rfloor} \frac{1}{(n-k) \ln k} &\leq \sum_{k=2}^{\left\lfloor n/2 \right\rfloor} \frac{2}{n \ln k}=\frac{2}{n} \sum_{k=2}^{\left\lfloor n/2 \right\rfloor} \frac{1}{\ln k}
        =\frac{2}{n} \sum_{k=2}^{\left\lfloor \sqrt{n/2} \right\rfloor} \frac{1}{\ln k}+\frac{2}{n} \sum_{k=\left\lfloor \sqrt{n/2} \right\rfloor+1}^{n/2} \frac{1}{\ln k} \\ 
        &\leq \frac{2}{n} \left(\frac{\sqrt{n/2}}{\ln 2}+\frac{n/2}{\ln \sqrt{n}}\right)=\frac{\sqrt{2}}{\sqrt{n}\ln 2}+\frac{2}{\ln n} \to 0
    \end{align*}
    现在考虑\(\sum_{k=\left\lfloor n/2 \right\rfloor+1}^{n-1} 1/(n-k)\ln k\).将其倒序重排,得到\(\sum_{k=1}^{\left\lfloor n/2 \right\rfloor} 1/k\ln(n-k)\).
    \begin{align*}
        \sum_{k=1}^{\left\lfloor n/2 \right\rfloor} \frac{1}{k\ln(n-k)} \leq \sum_{k=1}^{\left\lfloor n/2 \right\rfloor} \frac{1}{k\ln(n/2)}=\frac{1}{\ln(n/2)} \sum_{k=1}^{\left\lfloor n/2 \right\rfloor} \frac{1}{k} \leq \frac{1+\ln(n/2)}{\ln(n/2)} \to 1
    \end{align*}
    因此\(\lim_{n \to \infty} a_n \leq 1\),且下证\(\lim_{n \to \infty} a_n \geq 1\).
    \begin{align*}
        \sum_{k=2}^{n-1} \frac{1}{(n-k) \ln k}=\sum_{k=2}^{n-1} \frac{1}{k \ln(n-k)} \geq \sum_{k=2}^{n-1} \frac{1}{k \ln n}=\frac{1}{\ln n} \sum_{k=2}^{n-1} \frac{1}{k} \geq \frac{1}{\ln n} \int_1^n \frac{1}{k} dk=1
    \end{align*}
    因此得到\(\lim_{n \to \infty} a_n=1\).
\end{proof}
