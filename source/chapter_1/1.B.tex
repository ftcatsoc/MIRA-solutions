\section{1.B Basic Topology}

\begin{theorem}\label{br2.30}
    设\(Y \subseteq X\),则\(E \subseteq Y\)是\(Y\)的开子集等价于存在\(X\)的开子集\(G\)使得\(E=Y \cap G\).
\end{theorem}

\begin{proof}
    若\(E \subseteq Y\)是\(Y\)的开子集,则\(\forall p \in E, \exists r_p>0, \forall q \in Y, d(p,q)<r_p, q \in E\).

    令\(G=\bigcup_{p \in E} B(p,r_p)\),从而\(G\)是开集,下证\(E=Y \cap G\).

    由于\(\forall p \in E, p \in B(p,r_p) \subseteq G, p \in E \subseteq Y\),故\(E \subseteq Y \cap G\).

    由于\(\forall q \in B(p,r_p), q \in Y, d(p,q)<r_p, q \in E\),则\(\forall p \in E, B(p,r_p) \cap Y \subseteq E\),

    即\(\bigcup_{p \in E} (B(p,r_p) \cap Y)=\bigcup_{p \in E} B(p,r_p) \cap Y=G \cap Y \subseteq E\),充分性证毕.

    若\(E=Y \cap G\),则\(\forall p \in E, \exists r_p>0, \forall q \in Y, d(p,q)<r_p, q \in E\),即\(E\)是\(Y\)的开子集,证毕.
\end{proof}

\begin{theorem}\label{br2.36}
    若\(\{F_\alpha\}_{\mathcal{A}}\)是紧集族,且满足任意有限集\(\mathcal{S} \subseteq \mathcal{A}\)都有\(\bigcap_{\alpha \in \mathcal{S}} F_\alpha \ne \varnothing\),则\(\bigcap_{\alpha \in \mathcal{A}} F_\alpha \ne \varnothing\).
\end{theorem}

\begin{proof}
    设\(K \in \{F_\alpha\}_{\mathcal{A}}\).若\(K\)中不存在同时属于所有\(F_\alpha\)的点,则\(\{F_\alpha^c\}_{\mathcal{A}}\)是\(K\)的开覆盖.

    由于\(K\)是紧集,可以抽取一组有限子覆盖,即存在有限集\(\mathcal{S} \subseteq \mathcal{A}\)使得\(K \subseteq \bigcup_{\alpha \in \mathcal{S}} F_\alpha^c\).

    从而\(\bigcap_{\alpha \in \mathcal{S}} F_\alpha \subseteq K^c\),即\(K \cap \bigcap_{\alpha \in \mathcal{S}} F_\alpha=\varnothing\),与假设矛盾.
\end{proof}

\begin{comment}
    \begin{theorem}\label{br2.37}
        紧集\(K\)的无限子集\(E\)在\(K\)中必有极限点.
    \end{theorem}

    \begin{proof}
        设\(E\)在\(K\)中没有极限点,故\(\forall x \in E, \exists r_x>0, B(x,r_x) \cap E=\{x\}\).

        选择一个开集\(G\)满足\(K \setminus \bigcup_{x \in E} B(x,r_x) \subseteq G\),从而\(\{B(x,r_x)\}_{x \in E}, G\)构成一组开覆盖.

        由于\(\forall x \in E, \exists r_x>0, B(x,r_x) \cap E=\{x\}\),故\(x_0 \notin \bigcup_{x \in E \setminus \{x_0\}} B(x,r_x) \cup G\).

        因此\(K\)没有一个\(\{B(x,r_x)\}_{x \in E}, G\)的有限子覆盖,与\(K\)的紧性矛盾.
    \end{proof}
\end{comment}


\begin{theorem}\label{br2.43}
    若\(P \subseteq \mathbb{R}\)是非空完全集,那么\(P\)是不可数集.
\end{theorem}

\begin{proof}
    设\(P\)是可数集,考虑\(x_1 \in P\)的邻域\(B(x_1,r_1)\),下面{\kaishu 归纳地构造\(B(x_k,r_k)\)序列}.
    
    假设\(B(x_k,r_k)\)已经被构造.由\(x_k\)是\(P\)的极限点,故\(\exists x_{k+1} \in P, d(x_k,x_{k+1})<r_k\).

    选取\(r_{k+1} \in (0,\min\{d(x_k,x_{k+1}), r-d(x_k,x_{k+1})\})\),构造\(B(x_{k+1},r_{k+1})\).
    
    因此\(\overline{B}(x_{k+1},r_{k+1}) \subseteq B(x_k,r_k), x_n \notin \overline{B}(x_{k+1},r_{k+1})\).令\(F_k=B(x_k,r_k) \cap P\).

    由于\(\forall k \in \mathbb{Z}^+, F_k \ne \varnothing, F_{k+1} \subseteq F_k\),故\(\bigcap_{k=1}^\infty F_k \ne \varnothing\).

    然而由于\(F_k \subseteq P\)且\(x_k \notin F_{k+1}\),故\(\bigcap_{k=1}^\infty F_k=\varnothing\),矛盾.
\end{proof}

\begin{problem}[2]\label{1.B.2}
    设存在不全为零的整数\(a_0, \dots, a_n\),且复数\(z\)满足\(\sum_{k=0}^n a_n z^{n-k}=0\).

    这样的\(z\)被称为代数数.证明:所有代数数构成可数集.
\end{problem}

\begin{proof}
    由于\(a_0, \dots, a_n\)不全为零,故\(\sum_{k=0}^n \abs*{a_k}>0\).固定次数\(n\),设\(\sum_{k=0}^n \abs*{a_k}=N\).

    对于固定的\(n, N \in \mathbb{Z}^+\),只有有限个\(a_0, \dots, a_n\)的取值可能.记\(p_{n,N}, A_{n,N}\)为
    \begin{align*}
        p_{n,N}=\left\{p(z)=\sum_{k=0}^n a_k z^{n-k}: \sum_{k=0}^n \abs*{a_k}=N \right\},
        A_{n,N}=\bigcup_{p \in p_{n,N}}\left\{z \in \mathbb{C}: p(z)=0\right\}
    \end{align*}
    对于任意的\(p \in p_{n,N}\),最多只有\(n\)个复数\(z\)满足\(p(z)=0\),因而\(p_{n,N}, A_{n,N}\)是有限集.

    于是所有代数数的集合是\(\bigcup_{n=1}^\infty \bigcup_{N=1}^\infty A_{n,N}\),证毕.
\end{proof}

\newpage

\begin{problem}[12]\label{1.B.12}
    设\(K=\{0\} \cup \{1/n: n \in \mathbb{Z}^+\}\).证明\(K\)是紧集.
\end{problem}

\begin{proof}
    考虑\(K\)的可数开覆盖序列\(\{G_\alpha\}\),从中挑出任一包含\(0\)的开集\(G_0\).

    由\(0\)为\(G_0\)的内点,得\(\exists \delta>0, B(0,\delta) \subseteq G_0\),故\(\exists N \in \mathbb{Z}^+, \forall n>N, 1/n<\delta\).

    对于\(1, \dots, 1/N\),分别抽取\(G_1, \dots, G_N\)包含之,故\(G_0, G_1, \dots, G_N\)是一个有限子覆盖.
\end{proof}

\begin{problem}[13]\label{1.B.13}
    构造一个\((0,1)\)的可数开覆盖,但它没有有限子覆盖.
\end{problem}

\begin{proof}
    令\(G_k=(1/k, 1-1/k), k=3, 4, \dots\),则\(\{G_k\}_{k \in \mathbb{Z}^+, k \geq 3}\)是一个可数开覆盖.

    抽取有限子列\(G_{k_1}, \dots, G_{k_n}\),则令\(\max\{k_1, \dots, k_r\}=k_0\),得到\(\bigcup_{r=1}^n G_{k_r}=G_{k_0} \subseteq (0,1)\).
\end{proof}

\begin{problem}[16]\label{1.B.16}
    \(E=\{q \in \mathbb{Q}: q>0, q^2 \in (2,3)\}\).证明\(E\)在\(\mathbb{Q}\)中是有界闭集、开集但不是紧集.
\end{problem}

\begin{proof}
    显然\(-2,2\)分别是\(E\)的下界和上界,故\(E\)是有界集.下证\(E\)是闭集.

    假设存在\(q \in \mathbb{Q}\)使得\(q \notin E\)且存在\(q_1, q_2, \dots \in E\)满足\(\lim_{k \to \infty} q_k=q\).令
    \begin{align*}
        r(q)_1=\frac{q^2-2}{q+2}, r(q)_2=\frac{q^2-3}{q+3}, r(q)=\min\{r(q)_1, r(q)_2\}
    \end{align*}
    因此\(\forall k \in \mathbb{Z}^+, q_k \notin B(q,r(q))\),故\(\lim_{k \to \infty} q_k \ne q\),故\(E\)是闭集.

    考虑\(\forall q \in E, \exists r(q)>0, B(q,r(q)) \subseteq E\),故\(E\)是开集.

    然而考虑开覆盖序列\(\{G_k\}_{k \in \mathbb{Z}^+}, G_k=(q_k,2)\).其中\(q_0=3/2, q_k=(2q_{k-1}+2)/(q_{k-1}+2)\).

    显然\(E \subseteq \bigcup_{k=1}^\infty G_k\),但是若抽取有限子列\(G_{k_1}, \dots, G_{k_n}\),令\(\max\{k_1, \dots, k_r\}=k_0\),

    得到\(\bigcup_{r=1}^n G_{k_r}=G_{k_0}\),而\(\exists q_{k_0+1} \notin G_{k_0}\)且\(q_{k_0+1} \in E\),故\(E\)不是紧集.
\end{proof}

\begin{problem}[17]\label{1.B.17}
    令\(E \subseteq [0,1]\),其中所有实数的十进制展开只含\(4,7\).
    
    证明\(E\)在\([0,1]\)中不是稠密集,但却是紧集和完全集.
\end{problem}

\begin{proof}
    显然\(\sup E=7/9<1\),故\(1\)不是\(E\)中元素的极限点,故\(E\)不在\([0,1]\)中稠密.

    显然\(E\)是有界集,下证\(E\)是闭集.设\(\forall x=0.d_1d_2\dots \in [0,1] \setminus E, d_n \ne 4, d_n \ne 7\).

    显然选择\(r(n)=10^{-n-1}\)即可,\(B(x, 10^{-n-1}) \subseteq [0,1] \setminus E\),即\(E\)的补集是开集.

    设\(\forall \varepsilon>0, \exists n \in \mathbb{Z}^+, 10^{-n+1}<\varepsilon\).设\(x=0.d_1 \dots d_n \dots, x_n=0.d_1 \dots d_n' \dots\).

    令\(d_n\)和\(d_n'\)分别取不同值,从而\(\abs*{x-x_n}<10^{-n+1}<\varepsilon\),因此\(\{x_n\}_{n \in \mathbb{Z}^+}, x_n \in E\)收敛于\(x\).
\end{proof}

\newpage

\begin{problem}[19]\label{1.B.19}
    a.设\(A,B\)是度量空间\(X\)的不交闭集,证明它们是分离的.

    b.设\(A,B\)是度量空间\(X\)的不交开集,证明它们是分离的.

    c.固定\(p \in X, \delta>0\).令\(A=\{q \in X: d(p,q)<\delta\}, B=\{q \in X: d(p,q)>\delta\}\).证明它们分离.

    d.证明至少含有两个点的联通度量空间是不可数的.
\end{problem}

\begin{proof}[证明a]
    显然\(\overline{A} \cap B=A \cap \overline{B}=A \cap B=\varnothing\),即\(A,B\)是分离的.
\end{proof}

\begin{proof}[证明b]
    设\(\exists a_1, a_2, \dots \in A\)使得\(\lim_{k \to \infty}a_k=a \in B\).由\(B\)为开集,设\(\exists r>0, B(a,r) \subseteq B\).

    显然\(\exists n \in \mathbb{Z}^+, \forall k>n, a_k \in B(a,r) \subseteq A\),与\(A \cap B=\varnothing\)矛盾.
\end{proof}

\begin{proof}[证明c]
    显然\(A,B\)都是开集,因此它们是分离的.
\end{proof}

\begin{proof}[证明d]
    设存在一个{\kaishu 含可数元素的联通度量空间}\(X\),\(D=\{d(x,y): x,y \in X\}\)是可数集.

    由于\(\mathbb{R}\)是不可数集,故\(\exists \delta \in (\inf D, \sup D), \delta \notin D\),固定\(p \in X\).

    令\(A=\{q \in X: d(p,q)<\delta\}, B=\{q \in X: d(p,q)>\delta\}\).
    
    由于不存在\(q \in X, d(p,q)=\delta\),故\(X=A \cup B\),但\(A,B\)是分离集,矛盾.
\end{proof}

\begin{problem}[21]\label{1.B.21}
    设\(A,B\)是\(\mathbb{R}^n\)中的分离集,令\(a \in A, b \in B, t \in \mathbb{R}, p(t)=(1-t)a+tb\).

    设\(A_0=p^{-1}(A), B_0=p^{-1}(B)\).a.证明\(A_0,B_0\)是\(\mathbb{R}\)中的分离集.

    b.证明存在\(t_0 \in (0,1)\)使得\(p(t_0) \notin A \cup B\).

    c.证明\(\mathbb{R}^n\)的凸子集是联通集.
\end{problem}

\begin{proof}[证明a]
    设存在\(t_1, t_2, \dots \in A_0, \lim_{k \to \infty} t_k=t \in B_0\),则\(\forall k \in \mathbb{Z}^+, p(t_k) \in A\).

    然而\(\lim_{k \to \infty} p(t_k)=p(t) \in B\),则\(\overline{A} \cap B \ne \varnothing\),与\(A,B\)分离矛盾.
\end{proof}

\begin{proof}[证明b]
    设\(\forall t_0 \in (0,1), p^{-1}(t_0) \in A_0 \cup B_0\),令\(A'=A_0 \cap [0,1], B'=B_0 \cap [0,1]\).

    从而\((0,1)=A' \cup B'\)是两个不空分离集的并,\((0,1)\)不连通,显然矛盾.
\end{proof}

\begin{proof}[证明c]
    设\(a,b \in \mathbb{R}^n\).令\(A=\{p(t): t \in [0,1]\}\).若存在分离集\(V,W\)使得\(A=V \cup W\),

    那么\(p^{-1}(V) \cup p^{-1}(W)=[0,1]\),而\(p^{-1}(V), p^{-1}(W)\)分离,故\([0,1]\)不连通.
\end{proof}

\begin{problem}[22]\label{1.B.22}
    含有可数稠密子集的的度量空间是可分度量空间.证明\(\mathbb{R}^n\)是可分度量空间.
\end{problem}

\begin{proof}
    令\(Q=\{(q_1, \dots, q_n): \forall k=1, \dots, n, q_k \in \mathbb{Q}\}\),下证\(Q\)在\(\mathbb{R}^n\)中稠密.

    设\(x=(x_1, \dots, x_n) \in \mathbb{R}^n\),则\(\forall k=1, \dots, n, x_k\)有柯西序列\(\{q_k^i\}, q_k^i \in \mathbb{Q}\).

    令\(q_i=(q_1^i, \dots, q_n^i)\),从而\(\lim_{i \to \infty} q_i=(q_1^i, \dots, q_n^i)=(x_1, \dots, x_n)=x\).
\end{proof}

\newpage

\begin{problem}[23]\label{1.B.23}
    证明可分度量空间\(X\)有一组可数基.
\end{problem}

\begin{proof}
    设\(X\)的可数稠密子集是\(U=\{u_1, u_2, \dots\}\).令\(\mathcal{U}=\{B(u_k,q_j): u_k \in U, q_j \in \mathbb{Q}^+\}\).

    显然\(\mathcal{U}\)是可数集,下证\(\forall x \in V \subseteq X, \exists j,k \in \mathbb{Z}^+, x \in B(u_k,q_j) \subseteq V\),其中\(V\)是开集.

    由\(V\)是开集,\(\exists r>0, B(x,r) \subseteq V\).由\(U\)的稠密性,选择\(u_k \in U, d(u_k,x)<r/2\).

    由有理数的稠密性,选择\(q_j \in (d(u_k,x), r/2)\).构造\(B(u_k,q_j)\),下证\(x \in B(u_k,q_j) \subseteq V\).

    \(\forall y \in B(u_k,q_j), d(x,y) \leq d(x,u_k)+d(u_k,y)<r/2+q_j<r/2+r/2=r\),故\(y \in B(x,r)\).

    由于\(d(u_k,x)<q_j\),故\(x \in B(u_k,q_j)\).结合两者,得到\(x \in B(u_k,q_j) \subseteq V\).
\end{proof}

\begin{problem}[24]\label{1.B.24}
    设度量空间\(X\)满足其中任一无限子集都有极限点,证明它是可分度量空间.
\end{problem}

\begin{proof}
    固定\(x_1 \in X, \delta>0\).选取\(x_2, x_3, \dots \in X\)使得\(\forall k \in \mathbb{Z}^+, j<k \in \mathbb{Z}^+, d(x_j,x_k)>\delta\).

    若该过程可无限持续,则\(\forall k \in \mathbb{Z}^+, j>k \in \mathbb{Z}^+, d(x_j,x_k)>\delta\).
    
    从而\(\{x_k\}_{k \in \mathbb{Z}^+}\)不是柯西序列,与{\kaishu 任一无限子集都有极限点}矛盾.
    
    依次取\(\delta=1/n, n \in \mathbb{Z}^+\),得\(\forall n \in \mathbb{Z}^+, \exists x_1^n, \dots, x_{m_n}^n \in X, X=\bigcup_{k=1}^{m_n} B(x_k,1/n)\).
    
    构造\(U=\bigcup_{n=1}^\infty \bigcup_{k=1}^{m_n} \{x_k\}\).\(\forall x \in X, n \in \mathbb{Z}^+, \exists k_n \in \{1, \dots, m_n\}, d(x_{k_n}^n,x)<1/n\).

    于是\(x_{k_1}^1, x_{k_2}^2, \dots\)是\(x\)的一个逼近序列,即\(U\)在\(X\)中稠密.
\end{proof}

\begin{problem}[25]\label{1.B.25}
    证明紧度量空间\(X\)必有一组可数基.
\end{problem}

\begin{proof}
    由\(X\)为紧度量空间,设\(\forall n \in \mathbb{Z}^+, \exists x_1^n, \dots, x_{m_n}^n \in X, \bigcup_{k=1}^{m_n} B(x_k^n, 1/n)=X\).

    下证\(\bigcup_{n=1}^\infty \bigcup_{k=1}^{m_n} B(x_k^n, 1/n)\)是一组可数基.考虑开集\(V \subseteq X, x \in V\).

    存在\(r>0, B(x,r) \subseteq V\).由有理数的稠密性,选择\(N \in \mathbb{Z}^+, 1/N<r/2\).
    
    由\(X=\bigcup_{k=1}^{m_N} B(x_k^N, 1/n)\),于是\(\exists k \in \{1, \dots, m_N\}, x \in B(x_k^N, 1/N) \subseteq B(x,r) \subseteq V\).
\end{proof}

\begin{problem}[26]\label{1.B.26}
    设度量空间\(X\)满足其中任一无限子集都有极限点,证明它是紧度量空间.
\end{problem}

\begin{proof}
    \(X\)是{\kaishu 可分度量空间},故{\kaishu \(X\)的开覆盖总存在一组可数子覆盖}\(\{G_k\}_{k \in \mathbb{Z}^+}\).

    若\(X\)无有限子覆盖,那么\(\forall n \in \mathbb{Z}^+, X \setminus \bigcup_{k=1}^n G_k \ne \varnothing\),令\(F_n=X \setminus \bigcup_{k=1}^n G_k\).

    从每一\(F_n\)中抽取\(x_n\).令\(E=\{x_n: x \in \mathbb{Z}^+\}\).若\(E\)是有限集,则\(\exists x_t, x_t \in \bigcap_{n=1}^\infty F_n=\varnothing\),矛盾.

    若\(E\)是无限集,序列\(x_1, x_2, \dots\)应有极限点\(x\)且\(\exists n \in \mathbb{Z}^+, x \in \bigcup_{k=1}^n G_k\),即\(x \notin F_n\).

    由于\(F_n\)是闭集且\(x_n, x_{n+1}, \dots \in F_n\),然而其极限点\(x \notin F_n\),矛盾.
\end{proof}

\newpage

\begin{problem}[27]\label{1.B.27}
    设\(E \subseteq \mathbb{R}^n\),\(P\)是\(E\)的所有凝点的集,证明\(P\)是完全集,
    
    且\(E\)中最多有可数个点不在\(P\)中.
\end{problem}

\begin{proof}
    设\(\mathcal{S}=\{V_k\}_{k \in \mathbb{Z}^+}\)是\(\mathbb{R}^n\)的可数基.令\(\mathcal{U}=\{V_k: V_k \cap E \text{\kaishu 至多为可数集}\}\).

    考虑\(x \in \bigcup_{V_k \in \mathcal{U}} V_k\),即\(\exists k \in \mathbb{Z}^+, x \in V_k, V_k \cap E\)至多可数,即\(x \in \mathbb{R}^n \setminus P\).

    若\(x \in \mathbb{R}^n \setminus P\),则\(\exists r>0, B(x,r) \cap E\)至多可数.\(\exists \mathcal{F}_x \subseteq \mathcal{S}, B(x,r)=\bigcup_{V_k \in \mathcal{F}_x} V_k\).

    显然\(\forall V_k \in \mathcal{F}, V_k \cap E\)至多可数,即\(\mathcal{F}_x \subseteq \mathcal{U}, x \in \bigcup_{V_k \in \mathcal{U}} V_k\).因而\(\mathbb{R}^n \setminus P=\bigcup_{V_k \in \mathcal{U}} V_k\).

    {\kaishu 至此\(P\)是闭集,且至多只有可数点在\(E\)中,下证\(P\)是完全集.}

    取\(p \in P\),则\(\forall r>0, B(p,r) \cap E\)是不可数集.考虑\(B(p,r) \cap E\)的凝点集\(P'\).

    从\(P'\)中选择\(p' \ne p\),从而\(\forall r'>0, B(p',r') \cap B(p,r) \cap E\)也是不可数集,
    
    因此\(p'\)也是\(E\)的凝点,即\(\forall p \in P, r>0, \exists p' \in P' \subseteq P, d(p,p')<r\).\(P\)是完全集.

    {\kaishu 因此可分度量空间\(X\)的闭子集\(F\)都是一个完全集\(P\)和一个至多可数集的并.}
\end{proof}

\begin{comment}
\begin{problem}[28]\label{1.B.28}
    证明可分度量空间\(X\)的闭子集\(F\)都是一个完全集\(P\)和一个至多可数集的并.
\end{problem}

\begin{proof}
    令\(P\)是\(F\)所有凝点的集合,由于\(F\)是闭集,故\(P \subseteq F\).

    延续\cref{1.B.27}的操作,即可证明\(P\)是完全集且\(F \setminus P\)是至多可数集.
\end{proof}
\end{comment}

\begin{problem}[29]\label{1.B.29}
    开集\(G \subset \mathbb{R}\)可以被写成至多可数不相交的开区间的并集.
\end{problem}

\begin{proof}
    给出\(\mathbb{R}\)的可数基\(\mathcal{S}=\{(p,q): p,q \in \mathbb{Q}\}\).从而\(\forall G \subseteq \mathbb{R}, \exists \mathcal{U}_G \subseteq \mathcal{S}, G=\bigcup_{(p,q) \in \mathcal{U}_G}(p,q)\).

    定义\((p,q) \sim (p',q')\)当且仅当存在\((p_1,q_1), \dots, (p_n,q_n) \in \mathcal{S}\),
    
    满足\((p_1,q_1)=(p,q), (p_n,q_n)=(p',q'), \forall k=1, \dots, n-1, (p_k,q_k) \cap (p_{k+1}, q_{k+1}) \ne \varnothing\).

    定义等价类\(\mathcal{U}_{(p,q)}=\{(p',q') \in \mathcal{S}: (p,q) \sim (p',q')\}, [(p,q)]=\bigcup_{(p',q') \in \mathcal{U}_{(p,q)}}(p',q')\).

    故\(G=\bigcup_{(p,q) \in \mathcal{U}} [(p,q)]\),且\([(p_1,q_1)] \cap [(p_2,q_2)] \ne \varnothing\)和\([(p_1,q_1)]=[(p_2,q_2)]\)等价.

    下证若\(G\)是{\kaishu 联通开集},有\(\forall x,y \in G, (x,y) \subseteq G, G=[(x,y)]\).
    
    考虑紧集\([x,y]\).由\(\mathcal{S}\)为可数基,则\(\exists \mathcal{U}_{[x,y]} \subseteq \mathcal{S}, [x,y] \subseteq \bigcup_{(p,q) \in \mathcal{U}_{[x,y]}} (p,q)\).

    由\([x,y]\)为紧集,挑选有限子覆盖\((p_1,q_1), \dots, (p_n,q_n)\).从而\(\forall I_x \ni x, I_y \ni y, I_x \sim I_y\).

    {\kaishu 故一方面一个联通集包含于一个等价类,另一方面两个不相等的等价类是不连通的.}
\end{proof}

\begin{problem}[30]\label{1.B.30}
    若\(\forall k \in \mathbb{Z}^+, G_k\)是\(\mathbb{R}^n\)的稠密开子集,则\(\bigcap_{k=1}^\infty G_k \ne \varnothing\).
\end{problem}

\begin{proof}
    由\(G_1\)为开集,\(\exists x_1 \in G_1, r>0, B(x_1,r) \subseteq G_1\).下面{\kaishu 归纳地构造\(B(x_k,r_k)\)序列}.

    假设\(B(x_k,r_k)\)已经被构造.由\(G_{k+1}\)是\(\mathbb{R}^n\)的稠密子集,故\(\exists x_{k+1} \in G_{k+1}, d(x_k,x_{k+1})<r/2^{k+1}\).

    由\(G_{k+1}\)是开集,故\(\exists r_{k+1}'>0, B(x_{k+1},r_{k+1}') \subseteq G_{k+1}\).取\(r_{k+1} \in (0, \min\{r/2^{k+1}, r_{k+1}'\})\).

    于是\(B(x_{k+1},r_{k+1})\)被构造,考虑\(\forall x \in \overline{B}(x_{k+1},r_{k+1})\),有
    \begin{align*}
        d(x,x_k) \leq d(x,x_{k+1})+d(x_{k+1},x_k) \leq r_{k+1}+r/2^{k+1}<r/2^k \leq r_k
    \end{align*}
    这即表明\(\overline{B}(x_{k+1},r_{k+1}) \subseteq B(x_k,r_k)\).由于\(\forall k \in \mathbb{Z}^+, \overline{B}(x_k,r_k) \subseteq G_k\),考虑\(\bigcap_{k=1}^\infty \overline{B}(x_k,r_k)\).

    显然\(\bigcap_{k=1}^\infty G_k \supseteq \bigcap_{k=1}^\infty \overline{B}(x_k,r_k) \ne \varnothing\),证毕.
\end{proof}

\begin{comment}
    \begin{theorem}
        若有集合族\((X,\mathcal{S})\),则\(X \setminus \bigcup_{A \in \mathcal{S}} A=\bigcap_{A \in \mathcal{S}}(X \setminus A)\).
    \end{theorem}

    \begin{proof}
        考虑任意的\(x \in A \in \mathcal{S}\),则有
        \begin{align*}
            x \in X \setminus \bigcup_{A \in \mathcal{S}}A \Longleftrightarrow
            x \notin \bigcup_{A \in \mathcal{S}}A \Longleftrightarrow
            \forall A \in \mathcal{S}, x \in X \setminus A \Longleftrightarrow
            x \in \bigcap_{A \in \mathcal{S}} X \setminus A.
        \end{align*}
        于是\(X \setminus \bigcup_{A \in \mathcal{S}} A=\bigcap_{A \in \mathcal{S}}(X \setminus A)\).
    \end{proof}

    \begin{theorem}
        \(A\)是开集和\(A^c\)是闭集等价.
    \end{theorem}

    \begin{proof}
        若\(A\)是开集,那么对于\(A^c\)的任意极限点\(x\),都存在去心邻域\(N(x) \cap A^c \ne \varnothing\).

        于是\(N(x) \nsubseteq A\),也即\(x\)不是\(A\)的内点,于是\(x \notin A, x \in A^c\),\(A^c\)包含了它的所有极限点.

        若\(A^c\)是闭集,那么\(\forall x \in A, x\)不是\(A^c\)的极限点,因此存在\(x\)的去心邻域\(N(x) \cap A^c=\varnothing\).
        
        也即\(N(x) \subset A\),\(x\)是\(A\)的内点,所以\(A\)中的所有点都是\(A\)的内点.
    \end{proof}

    \begin{theorem}
        开集\(G\)对可数并封闭,对有限交封闭;闭集\(F\)对可数交封闭,对有限并封闭.    
    \end{theorem}

    \begin{proof}
        \(\bigcup_{G \in \mathcal{S}}G\)中的每一点\(x\)都是某个\(G_i\)的内点,于是存在\(N(x) \subset G_i \subset \bigcup_{G \in \mathcal{S}}G\).

        \(\bigcap_{i=1}^n G_i\)中的每一点\(x\)都存在一个邻域\(N_i(x) \subset G_i\),取\(N(x)=\bigcap_{i=1}^n N_i(x) \subset \bigcap_{i=1}^n G_i\).

        {\kaishu 对以上结论取补集并使用徳·摩根律即可证明闭集的结论.}
    \end{proof}

    \begin{theorem}
        \(A \subset \mathbb{R}\)是联通集当且仅当对于\(x<y \in A, (x,y) \subset A\).
    \end{theorem}

    \begin{proof}
        若存在\(z \in (x,y) \in \mathbb{R}\)且\(z \notin A\),令\(A_1=(-\infty,z) \cap A, A_2=(z,\infty) \cap A\).

        显然\(A_1,A_2\)是分离的,而\(A=A_1 \cup A_2\),那么\(A\)就不是联通的.
    \end{proof}

    \begin{theorem}
        开集\(G \subset \mathbb{R}\)可以被写成不相交的开区间\(I_1, I_2, \dots\)的并集\(\bigcup_{k=1}^\infty I_k\).
    \end{theorem}

    \begin{proof}
        定义\(x \sim y\)为存在\(G\)的联通子集\(C\)使得\(x,y \in C\),定义\([x]=\{y \in G: y \sim x\}\).

        若\([x] \cap [y] \ne \varnothing\),则\(\exists z \in G, x \sim z, y \sim z\),从而\(x \sim y\),则\([x]=[y]\).

        因此一个等价类\([x]\)规定一个最大联通子集\(C\),且它们互不相交,且\(G=\bigcup_{x \in G}[x]\).

        由有理数的稠密性,\([x]\)一定包含一个有理数\(q\),且\([x]=[q]\),那么\(G=\bigcup_{q \in G}[q]\).
    \end{proof}

    \begin{proof}
        定义\(I_x=(a_x,b_x), a_x=\inf\{a \in \mathbb{R}: (a,x] \subset G\}, b_x=\sup\{b \in \mathbb{R}: [x,b) \subset G\}\).

        由于\(G\)是开集,因而\(\forall x \in G, \exists N(x) \subset G\),于是\(\forall x \in G, x \in N(x) \subset I_x \subseteq \bigcup_{x \in G}I_x\).

        因此\(G \subseteq \bigcup_{x \in G}I_x\).另一方面,\(\forall x \in G, I_x \subset G, \bigcup_{x \in G}I_x \subseteq G\),故\(G=\bigcup_{x \in G}I_x\).

        若\(I_x \cap I_y \ne \varnothing\),则\(I_x=I_y\),由有理数的稠密性,可以为每个\(I_x\)选择任一有理数\(q_x\).

        若\(q_x \in I_y\),则\(I_x, I_y\)视为同一分支.由有理数集是可数的,于是总分支数是至多可数的.

        将所有涉及的有理数\(q_1, q_2, \dots\)所代表的区间记作\(I_1, I_2, \dots\),则\(G=\bigcup_{k=1}^\infty I_k\).
    \end{proof}

    \newpage

    \begin{theorem}[2.25]
        设\(\mathcal{S}\)是\(X\)上的\(\sigma-\)代数,那么对于\(A_1, \dots, A_2, \dots \in \mathcal{S}\),

        有\(A_1 \cap A_2 \in \mathcal{S}, A_1 \setminus A_2 \in \mathcal{S}, \bigcap_{i=1}^\infty A_i \in \mathcal{S}\).
    \end{theorem}

    \begin{proof}
        由\(X \setminus A_1, X \setminus A_2 \in \mathcal{S}\),于是\((X \setminus A_1) \cup (X \setminus A_2)=X \setminus (A_1 \cap A_2) \in \mathcal{S}\),即\(A_1 \cap A_2 \in \mathcal{S}\).

        由\(A_1 \setminus A_2=A_1 \cap (X \setminus A_2)\)且\(X \setminus A_2 \in \mathcal{S}\)推出\(A_1 \setminus A_2 \in \mathcal{S}\).

        由\(X \setminus A_i \in \mathcal{S}\)得到\(X \setminus \bigcap_{i=1}^\infty A_i=\bigcup_{i=1}^\infty (X \setminus A_i) \in \mathcal{S}\),于是\(\bigcap_{i=1}^\infty A_i \in \mathcal{S}\).
    \end{proof}

    \begin{theorem}[2.33]
        设\(f: X \to Y\)是一个函数,\((Y,\mathcal{A})\)确定一个集合族且\(A \subseteq \mathcal{A}\),那么
        
        a.\(f^{-1}(X \setminus A)=Y \setminus f^{-1}(A)\) \enspace
        b.\(f^{-1}(\bigcup_{A \in \mathcal{A}}A)=\bigcup_{A \in \mathcal{A}}f^{-1}(A)\) \enspace
        c.\(f^{-1}(\bigcap_{A \in \mathcal{A}}A)=\bigcap_{A \in \mathcal{A}}f^{-1}(A)\)
    \end{theorem}

    \begin{proof}
        考虑任意的\(x \in A \in \mathcal{A}\),则有
        \begin{align*}
            &x \in f^{-1}(Y \setminus A) \Longleftrightarrow 
            f(x) \in Y \setminus A \Longleftrightarrow
            f(x) \notin A \Longleftrightarrow x \in X \setminus f^{-1}(A) \\
            &x \in f^{-1}(\bigcup_{A \in \mathcal{A}}A) \Longleftrightarrow
            f(x) \in \bigcup_{A \in \mathcal{A}}A \Longleftrightarrow
            \exists A \in \mathcal{A}, f(x) \in A \Longleftrightarrow
            x \in \bigcup_{A \in \mathcal{A}}f^{-1}(A) \\
            &x \in f^{-1}(\bigcap_{A \in \mathcal{A}}A) \Longleftrightarrow
            f(x) \in \bigcap_{A \in \mathcal{A}}A \Longleftrightarrow
            \forall A \in \mathcal{A}, f(x) \in A \Longleftrightarrow
            x \in \bigcap_{A \in \mathcal{A}}f^{-1}(A) \qedhere
        \end{align*}
    \end{proof}

    \begin{theorem}[2.27]
        设\(\mathcal{A}\)是\(X\)上的集合族,那么所有包含\(\mathcal{A}\)的\(\sigma-\)代数之交也是\(\sigma-\)代数.
    \end{theorem}

    \begin{proof}
        令\(\mathcal{S}\)是所有包含\(\mathcal{A}\)的\(\sigma-\)代数之交.\(X\)上的所有\(\sigma-\)代数都包含空集.

        若\(A \in \mathcal{S}\),那么\(X \setminus A\)也在每一个包含\(A\)的\(\sigma-\)代数中,也即\(X \setminus A \in \mathcal{S}\).

        若\(A_1, A_2, \dots \in \mathcal{S}\),那么\(\bigcup_{i=1}^\infty A_i\)也在每一个包含\(A_i\)的\(\sigma-\)代数中,也即\(\bigcup_{i=1}^\infty A_i \in \mathcal{S}\).

        {\kaishu 因此对于任意集合族都存在包含之的最小\(\sigma-\)代数.}
    \end{proof}
\end{comment}