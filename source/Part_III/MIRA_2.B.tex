\section{Part III. Continuity and Measurability}

\subsection{MIRA 2.B Measurable Spaces and Functions}

\begin{theorem}[2.39]\label{2.39}
    设\((X,\mathcal{S})\)是测度空间且\(f:X \to \mathbb{R}\).\(O\)是任意开集,且\(f^{-1}(O) \in \mathcal{S}\),

    那么\(f\)是\(\mathcal{S}-\)可测函数.
\end{theorem}

\begin{proof}
    令\(\mathcal{R}=\{A \subseteq \mathbb{R}:f^{-1}(A) \in \mathcal{S}\}\),下证\(\mathcal{R}\)是\(\sigma-\)代数.取\(A=\varnothing\)得\(\varnothing \in \mathcal{R}\).
    
    设\(A \in \mathcal{R}\),则\(f^{-1}(\mathbb{R} \setminus A)=X \setminus f^{-1}(A) \in \mathcal{S}\),即\(\mathbb{R} \setminus A \in \mathcal{R}\).

    设\(A_1, A_2, \dots \in \mathcal{R}\),则\(f^{-1}(\bigcup_{k=1}^\infty A_k)=\bigcup_{k=1}^\infty f^{-1}(A_k) \in \mathcal{S}\),即\(\bigcup_{k=1}^\infty A_k \in \mathcal{R}\).

    于是\(\mathcal{R}\)是\(\sigma\)-代数,且包含所有开集\(O\),则它是\textit{borel}代数.

    下证\(\{(a,\infty): a \in \mathbb{R}\}\)生成的\(\sigma-\)代数\(\mathcal{S}\)包含所有\textit{borel}集.

    由\(\forall a \in \mathbb{R}, (a,\infty) \in \mathcal{S}\),取补得\(\forall b \in \mathbb{R}, (-\infty,b] \in \mathcal{S}\),取并得\(\forall a \in \mathbb{R}, b \in \mathbb{R}, (a,b] \in \mathcal{S}\).

    由\((a,b)=\bigcup_{k=1}^\infty (a,b-1/k], (-\infty,b)=\bigcup_{k=1}^\infty (-k,b-1/k]\)得\(\forall a,b \in \mathbb{R}, (a,b) \in \mathcal{S}, (-\infty,b) \in \mathcal{S}\).

    于是\(\forall a,b \in \mathbb{R}, (-\infty,b), (a,b), (a, \infty) \in \mathcal{S}\),故\(\mathcal{S}\)包含了所有开集,即证毕.
\end{proof}

\begin{theorem}[2.46]\label{2.46}
    若\(X,\mathcal{S}\)是测度空间且\(f,g:X \to \mathbb{R}\)都是\(\mathcal{S}-\)可测函数,

    那么\(f+g, fg, f/g\)都是\(\mathcal{S}-\)可测函数,前提是\(\forall x \in X, g(x) \ne 0\).
\end{theorem}

\begin{comment}
    \begin{proof}
        考虑任意的\(a \in \mathbb{R}, r \in \mathbb{Q}\),下面将证明
        \begin{align*}
            (f+g)^{-1}((a,\infty))=\bigcup_{r \in \mathbb{Q}}\left(f^{-1}((r,\infty)) \cap g^{-1}((a-r,\infty))\right)
        \end{align*}
        设\(x \in \bigcup_{r \in \mathbb{Q}}\left(f^{-1}((r,\infty)) \cap g^{-1}((a-r,\infty))\right)\),即\(f(x)>r, g(x)>a-r, f(x)+g(x)>a\).

        所以\(x \in (f+g)^{-1}((a,\infty))\),得到\(\bigcup_{r \in \mathbb{Q}}\left(f^{-1}((r,\infty)) \cap g^{-1}((a-r,\infty))\right) \subseteq (f+g)^{-1}((a,\infty))\).

        另一方面,设\(x \in (f+g)^{-1}((a,\infty))\),即\(f(x)+g(x)>a\),因此\((a-g(x),f(x)) \ne \varnothing\).

        由有理数的稠密性,总存在\(r \in \mathbb{Q}, r \in (a-g(x),f(x))\),同时满足\(f(x)>r, g(x)>a-r\).
        
        从而\(x \in f^{-1}((r,\infty))\)且\(g^{-1}((a-r,\infty))\),等价于\(x \in f^{-1}((r,\infty)) \cap g^{-1}((a-r,\infty))\).
        
        于是\((f+g)^{-1}((a,\infty)) \subseteq \bigcup_{r \in \mathbb{Q}}\left(f^{-1}((r,\infty)) \cap g^{-1}((a-r,\infty))\right)\),证毕.

        可得\(\forall r \in \mathbb{Q}, f^{-1}((r,\infty)) \in \mathcal{S}, g^{-1}((a-r,\infty)) \in \mathcal{S}\),
        因而\(\forall a \in \mathbb{R}, (f+g)^{-1}((a,\infty)) \in \mathcal{S}\).

        根据\thmref{2.39},\(f+g\)是\(\mathcal{S}-\)可测函数.

        由于\(fg=(f+g)^2-(f^2+g^2)\),根据定理2.44,可测函数具有复合封闭性.
        
        那么\((f+g)^2, f^2, g^2\)都是\(\mathcal{S}-\)可测函数,从而\((f+g)^2-(f^2+g^2)=fg\)是\(\mathcal{S}-\)可测函数.

        同理,\(1/g\)也为\(\mathcal{S}-\)可测函数,于是\(f(1/g)=f/g\)是可测函数.
    \end{proof}
\end{comment}

\begin{proof}
    下证\(\forall x \in X, a \in \mathbb{R}, f(x)+g(x)>a \Longleftrightarrow \exists r \in \mathbb{Q}, f(x)>r, g(x)>a-r\).

    一边是显然的.对于另一边,\(f(x)>a-g(x)\)表明\((a-g(x),f(x)) \ne \varnothing\).

    据有理数的稠密性,\(\exists r \in \mathbb{Q}, r \in (a-g(x),f(x))\),于是\(f(x)>r, g(x)>a-r\).

    定义\(A_{a,r}=f^{-1}((r,\infty)) \cap g^{-1}((a-r,\infty))\).由\(\sigma-\)代数的交并封闭性,\(A_{a,r} \in \mathcal{S}\).

    于是\(\forall a \in \mathbb{R}, (f+g)^{-1}(a,\infty)=\bigcup_{r \in \mathbb{Q}}A_{a,r} \in \mathcal{S}\).根据\cref{2.39},\(f+g\)是\(\mathcal{S}-\)可测函数.

    由于\(fg=[(f+g)^2-(f-g)^2]/4\),从而\(fg\)是\(\mathcal{S}-\)可测函数.

    同理,\(1/g\)也为\(\mathcal{S}-\)可测函数,于是\(f(1/g)=f/g\)是可测函数.
\end{proof}

\begin{theorem}[2.48]\label{2.48}
    若\(X,\mathcal{S}\)是测度空间且\(\mathcal{S}-\)可测函数列\(f_k:X \to \mathbb{R}\)逐点收敛于函数\(f\),
    
    即\(\lim_{k \to \infty}f_k(x)=f(x)\),那么\(f\)是\(\mathcal{S}-\)可测函数.
\end{theorem}

\begin{comment}
    \begin{proof}
        考虑任意的\(a \in \mathbb{R}, n \in \mathbb{N}, m \in \mathbb{N}\),下面将证明
        \begin{align*}
            f^{-1}((a,\infty))=\bigcup_{n=1}^\infty \bigcup_{m=1}^\infty \bigcap_{k=m}^\infty f_k^{-1}\left(\left(a+\frac{1}{n},\infty\right)\right)
        \end{align*}
        设\(x \in f^{-1}((a,\infty)), f(x)>a\),开集的所有点都是内点,因此\(\exists n \in \mathbb{N}, f(x)>a+\frac{1}{n}\).

        由于\(\lim_{k \to \infty}f_k(x)=f(x)\),因而\(\exists m \in \mathbb{N}, \forall k \geq m, f_k(x)>a+\frac{1}{n}\).

        因此\(\forall k \geq m, x \in f_k^{-1}((a+\frac{1}{n}))\),
        则\(f^{-1}((a,\infty)) \subseteq \bigcup_{n=1}^\infty \bigcup_{m=1}^\infty \bigcap_{k=m}^\infty f_k^{-1}((a+\frac{1}{n},\infty))\).

        设\(x \in \bigcup_{n=1}^\infty \bigcup_{m=1}^\infty \bigcap_{k=m}^\infty f_k^{-1}((a+\frac{1}{n},\infty))\),
        即\(\exists m,n \in \mathbb{N}, \forall k \geq m, f_k(x)>a+\frac{1}{n}\).

        取\(k \to \infty\),得到\(f(x)=\lim_{k \to \infty}f_k(x) \geq a+\frac{1}{n}>a, x \in f^{-1}((a,\infty))\).

        由于\(\forall n,k \in \mathbb{N}, f_k^{-1}((a+\frac{1}{n},\infty)) \in \mathcal{S}\),故\(f^{-1}((a,\infty)) \in \mathcal{S}\),证毕.
    \end{proof}
\end{comment}

\begin{proof}
    \(\forall a \in \mathbb{R}, x \in X\)满足\(f(x)>a\),那么\((a,f(x)) \ne \varnothing, \exists r \in \mathbb{Q}, r \in (a,f(x))\).

    由\(f_k\)逐点收敛至\(f\),固定\(x \in X\),那么\(\exists m \in \mathbb{N}, \forall k \geq m, \abs*{f_k(x)-f(x)}<a-r, f_k(x)>r\).

    定义\(\forall a \in \mathbb{R}, r \in \mathbb{Q}, A_r=\bigcup_{m=1}^\infty \bigcap_{k=m}^\infty f_k^{-1}((r,\infty)), A_a=\bigcup_{r>a}A_r\),那么\(A_a \in \mathcal{S}\).

    下证\(f^{-1}((a,\infty))=A_a\).若\(f(x)>a\),则\( \exists r \in (f(x),a), x \in A_r\),一边得证.

    若\(x \in A_a\),那么\(\exists r>a, m \in \mathbb{N}, \forall k \geq m, f_k(x)>r, f(x) \geq r>a\),从而\(x \in f^{-1}((a,\infty))\).

    于是\(f^{-1}((a,\infty))=\bigcup_{r>a} \bigcup_{m=1}^\infty \bigcap_{k=m}^\infty f_k^{-1}((r,\infty)) \in \mathcal{S}\),根据\cref{2.39},\(f\)是\(\mathcal{S}-\)可测函数.
\end{proof}

\newpage

\begin{problem}[1]\label{2.B.1}
    证明\(\mathcal{S}=\{\bigcup_{n \in K}(n,n+1]: K \subseteq \mathbb{Z}\}\)是\(\mathbb{R}\)上的\(\sigma-\)代数.
\end{problem}

\begin{proof}
    若\(\bigcup_{n \in K}(n,n+1] \in \mathcal{S}\),那么\(\mathbb{R} \setminus \bigcup_{n \in K}(n,n+1]=\bigcup_{n \in \mathbb{Z} \setminus K}(n,n+1] \in \mathcal{S}\).

    若\(\bigcup_{n \in K_1}(n,n+1], \bigcup_{n \in K_2}(n,n+1], \dots \in \mathcal{S}\)则\(\bigcup_{k=1}^\infty \bigcup_{n \in K_k}(n,n+1]=\bigcup_{n \in \bigcup_{k=1}^\infty K_k}(n,n+1] \in \mathcal{S}\).

    取\(K=\varnothing\)得到\(\varnothing \in \mathcal{S}\).综上,\(\mathcal{S}=\{\bigcup_{n \in K}(n,n+1]: K \subseteq \mathbb{Z}\}\)是\(\mathbb{R}\)上的\(\sigma-\)代数.
\end{proof}

\begin{problem}[4]\label{2.B.4}
    证明:所有形如\((r,n], r \in \mathbb{Q}, n \in \mathbb{N}\)的集合生成的\(\sigma-\)代数\(\mathcal{S}\)包含所有borel集.
\end{problem}

\begin{proof}
    下证\(\forall x \in \mathbb{R}, x>a \Longleftrightarrow \exists r>a \in \mathbb{Q}, x \in (r, \left\lceil r \right\rceil]\),一边是显然的.

    另一边,由\((a,x) \cap [\left\lfloor x \right\rfloor, \infty) \ne \varnothing\),故\(\exists r \in \mathbb{Q}, a<r<x, x \geq \left\lfloor r \right\rfloor, \left\lceil r \right\rceil \geq x\).

    那么\(\forall a \in \mathbb{R}, (a,\infty)=\bigcup_{r>a}(r, \left\lceil r \right\rceil]\).注意到\((r, \left\lceil r \right\rceil]\)生成的\(\sigma-\)代数是\((r,n]\)的子集.

    于是根据\cref{2.B.4},\(\mathcal{S}\)包含了所有开集,证毕.
\end{proof}

\begin{problem}[5]\label{2.B.5}
    证明:所有形如\((r,r+1), r \in \mathbb{Q}\)的集合生成的\(\sigma-\)代数\(\mathcal{S}\)包含所有borel集.
\end{problem}

\begin{proof}
    下证\(\forall x \in \mathbb{R}, x>a \Longleftrightarrow \exists r>a \in \mathbb{Q}, x \in (r,r+1)\),一边是显然的.

    另一边,由\((a,x) \cap (x-1, \infty) \ne \varnothing\),故\(\exists r \in \mathbb{Q}, a<r<x, r>x-1, x \in (r,r+1)\).

    那么\(\forall a \in \mathbb{R}, (a,\infty)=\bigcup_{r>a}(r,r+1)\),于是根据\cref{2.B.4},\(\mathcal{S}\)包含了所有开集.
\end{proof}

\begin{problem}[7]\label{2.B.7}
    证明:若\(B \subseteq \mathbb{R}\)是borel集,那么\(\forall t \in \mathbb{R}, t+B\)是borel集.
\end{problem}

\begin{proof}
    定义\(\mathcal{A}=\{B: \forall t \in \mathbb{R}, t+B \text{\kaishu 是\textit{borel}集}\}\),下证\(\mathcal{A}\)是一个\(\sigma\)-代数.

    首先\(\varnothing \in \mathcal{A}\).考虑\(B \in \mathcal{A}\),那么\(t+(\mathbb{R} \setminus B)=\mathbb{R} \setminus (t+B) \in \mathcal{A}\),即\(\mathcal{A}\)对补集封闭.

    设\(B_1, B_2, \dots \in \mathcal{A}\),那么\(t+\bigcup_{k=1}^\infty B_k=\bigcup_{k=1}^\infty (t+B_k) \in \mathcal{A}\),因此\(\mathcal{A}\)对可数并封闭.

    综上\(\mathcal{A}\)是\(\sigma-\)代数.而对于开集\(O=\bigcup_{k=1}^\infty I_k, t+O=t+\bigcup_{k=1}^\infty I_k=\bigcup_{k=1}^\infty (t+I_k) \in \mathcal{A}\).

    因此\(\mathcal{A}\)包含所有开集.而\(\mathcal{A}\)又是\(\sigma\)-代数,故\(\mathcal{A}\)包含所有\textit{borel}集,证毕.
\end{proof}

\begin{problem}[9]\label{2.B.9}
    设\((X,\mathcal{S})\)是可测集,给出函数\(f: X \to \mathbb{R}\)满足\(\abs*{f}\)是\(\mathcal{S}-\)可测的但\(f\)不可测.
\end{problem}

\begin{proof}
    令\(X=\mathbb{R}, \mathcal{S}=\{\varnothing, X\}, f(x)=\chi_{[0,\infty)}(x)-\chi_{(-\infty,0)}(x)\).

    显然\(\abs*{f}=1, \abs*{f}^{-1}(\{1\})=X\),因此\(\abs*{f}\)是\(\mathcal{S}-\)可测的,但是\(f^{-1}(\{1\})=[0,\infty) \notin \mathcal{S}\).
\end{proof}

\begin{comment}
    \begin{problem}[10]\label{2.B.10}
        证明:所有在十进制表示下的有无穷位是\(5\)的实数构成了一个borel集合.
    \end{problem}

    \begin{proof}
        给出实数\(x\)的十进制表示\([a_N \dots a_0. a_{-1} \dots], x=\sum_{i=-\infty}^N a_i 10^i\).

        此时题设集合\(A=\{x=[a_N \dots a_0. a_{-1} \dots]: \limsup_{i \to -\infty}\chi_{a_i=5}=1\}\).

        据上极限的定义,设\(B_i=\{x=[a_N \dots a_0. a_{-1} \dots]: a_i=5\}\),则\(A=\bigcap_{n=1}^\infty \bigcup_{i=n}^\infty B_i\).
        \begin{align*}
            B_i=\bigcup_{a_N, \dots, a_{i-1}} \left[\sum_{k=i-1}^N a_k 10^k +5 \cdot 10^i,\right. \left.\sum_{k=i-1}^N a_k 10^k +6 \cdot 10^i\right)
        \end{align*}
        作为区间的可数并,\(B_i\)是\textit{borel}集,因此\(A=\bigcap_{n=1}^\infty \bigcup_{i=n}^\infty B_i\)也是\textit{borel}集.
    \end{proof}
\end{comment}

\begin{problem}[11]\label{2.B.11}
    设\(\mathcal{R}\)是\(Y\)上的\(\sigma-\)代数,且\(X \in \mathcal{R}\).设\(\mathcal{S}=\{E \in \mathcal{R}: E \subseteq X\}\).

    a.证明\(\mathcal{S}=\{F \cap X: F \in \mathcal{R}\}\). \quad b.证明\(\mathcal{S}\)是\(X\)上的\(\sigma-\)代数.
\end{problem}

\begin{proof}[证明a]
    \(\forall E \subseteq X, E \in \mathcal{R} \Longleftrightarrow \exists F \in \mathcal{R}, E=F \cap X\).
\end{proof}

\begin{proof}[证明b]
    令\(F=\varnothing\)得到\(\varnothing \in \mathcal{S}\).若\(F \in \mathcal{R}, Y \setminus F \in \mathcal{R}\),则\(X \setminus (F \cap X)= (Y \setminus F) \cap X \in \mathcal{S}\).

    若\(F_1, F_2, \dots \in \mathcal{R}, \bigcup_{k=1}^\infty F_k \in \mathcal{R}\),那么\(\bigcup_{k=1}^\infty (F_k \cap X)=X \cap \bigcup_{k=1}^\infty F_k \in \mathcal{S}\).

    综上,\(\mathcal{S}\)是\(X\)上的\(\sigma-\)代数.
\end{proof}

\newpage

\begin{comment}
    \begin{problem}[12]\label{2.B.12}
        令\(B(a,\delta)=(a-\delta, a+\delta)\).设函数\(f: \mathbb{R} \to \mathbb{R}\).对于正整数\(k \in \mathbb{N}\),令
        \begin{align*}
            G_k=\{a: \exists \delta>0, \forall a_1,a_2 \in B(a,\delta), \abs*{f(a_1)-f(a_2)}<1/k\}
        \end{align*}
        a.证明\(G_k\)是开集. \quad b.证明函数所有连续点的集合为\(\bigcap_{k=1}^\infty G_k\).
    \end{problem}

    \begin{proof}[证明a]
        设\(a_0 \in G_k\),则\(\exists \delta_0>0, \forall a_1, a_2 \in B(a_0,\delta_0), \abs*{f(a_1)-f(a_2)}<1/k\).
        
        设\(a \in B(a_0,\delta_0)\).令\(\delta=\delta_0-\abs*{a_0-a}>0\),那么\(B(a,\delta) \subseteq B(a_0,\delta_0)\).

        于是\(\forall a_1, a_2 \in B(a,\delta), \abs*{f(a_1)-f(a_2)}<1/k\),故\(a \in G_k\),因而\(B(a_0,\delta_0) \subseteq G_k\).
    \end{proof}

    \begin{proof}[证明b]
        若\(x \in \bigcap_{k=1}^\infty G_k\),下证\(f\)在\(x\)处连续.\(\forall \varepsilon>0, \exists k \in \mathbb{N}, 1/k<\varepsilon, \exists \delta>0,\)
        \begin{align*}
            \sup \{\abs*{f(x)-f(a)}: a \in B(x,\delta)\} 
            \leq \sup \{\abs*{f(a_1)-f(a_2)}: a_1, a_2 \in B(x,\delta)\}<1/k<\varepsilon
        \end{align*}
        若\(f\)在\(x\)处连续,则\(\forall \varepsilon>0, \exists \delta>0, \forall a \in B(x,\delta), \abs*{f(x)-f(a)}<\varepsilon/2\).

        那么\(\forall a_1, a_2 \in B(x,\delta), \abs*{f(a_1)-f(a_2)} \leq \abs*{f(x)-f(a_1)}+\abs*{f(x)-f(a_2)}<\varepsilon/2+\varepsilon/2=\varepsilon\).

        因此\(\forall \varepsilon>0, \exists k \in \mathbb{N}, 1/k<\varepsilon, x \in \bigcap_{k=1}^\infty G_k\).
    \end{proof}
\end{comment}

\begin{problem}[13]\label{2.B.13}
    设\((X, \mathcal{S})\)是测度空间,\(E_1, \dots, E_n \subseteq X\)互不相交,\(c_1, \dots, c_n\)非零且互异.

    证明:\(f=\sum_{k=1}^n c_k \chi_{E_k}\)是\(\mathcal{S}\)可测函数等价于\(E_1, \dots, E_n \in \mathcal{S}\).
\end{problem}

\begin{proof}
    \(\sum_{k=1}^n c_k \chi_{E_k}(X)=\{c_1, \dots, c_n, 0\}\).若\(f\)是\(\mathcal{S}\)可测函数,

    那么\(\forall k=1, \dots, n, \sum_{k=1}^n f^{-1}(\{c_k\})=E_k \in \mathcal{S}\),从而\(E_1, \dots, E_n \in \mathcal{S}\).

    若\(E_1, \dots, E_n \in \mathcal{S}\),那么\(X \setminus \bigcup_{k=1}^n E_k \in \mathcal{S}\).因此\(f^{-1}(\{0\})=X \setminus \bigcup_{k=1}^n E_k \in \mathcal{S}\).

    结合\(\forall k=1, \dots, n, f^{-1}(\{c_k\})=E_k \in \mathcal{S}\),从而\(\sum_{k=1}^n c_k \chi_{E_k}\)是\(\mathcal{S}\)可测函数.
\end{proof}

\begin{comment}
    \begin{problem}[14]\label{2.B.14}
        考虑函数列\(f_1, f_2, \dots: X \to \mathbb{R}\),证明:
        \begin{align*}
            \{x \in X: \lim_{k \to \infty} f_k(x)\text{存在}\}
            =\bigcap_{n=1}^\infty \bigcup_{j=1}^\infty \bigcap_{k=j}^\infty (f_k-f_j)^{-1}((-1/n, 1/n))
        \end{align*}
    \end{problem}

    \begin{proof}
        若\(x \in \bigcap_{n=1}^\infty \bigcup_{j=1}^\infty \bigcap_{k=j}^\infty (f_k-f_j)^{-1}((-1/n, 1/n))\),给定\(\forall \varepsilon>0\),那么\(\exists n \in \mathbb{N}, 1/n<\varepsilon\).

        则\(\exists n,j \in \mathbb{N}, \forall k \geq j, \abs*{f_k(x)-f_j(x)}<1/n<\varepsilon\),于是\(f_1(x), f_2(x), \dots\)是柯西序列.

        根据{\kaishu 实数的完备性},\(f_1(x), f_2(x), \dots\)在\(\mathbb{R}\)中必收敛,也即\(\lim_{k \to \infty}f_k(x)\)存在.

        设\(\lim_{k \to \infty}f_k(x)=c\),那么\(\forall n \in \mathbb{N}, \exists j \in \mathbb{N}, \forall k \geq j, \abs*{f_k(x)-c}<1/2n\).

        则\(\forall n \in \mathbb{N}, \exists j,k \in \mathbb{N}, \abs*{f_k(x)-f_j(x)} \leq \abs*{f_k(x)-c}+\abs*{f_j(x)-c}< 1/2n+1/2n<1/n\).
        
        即\(x \in \bigcap_{n=1}^\infty \bigcup_{j=1}^\infty \bigcap_{k=j}^\infty (f_k-f_j)^{-1}((-1/n, 1/n))\).
    \end{proof}
\end{comment}

\begin{problem}[15]\label{2.B.15}
    设\(E_1, E_2, \dots \subseteq X\)互不相交且\(\bigcup_{k=1}^\infty E_k=X\).令\(\mathcal{S}=\{\bigcup_{k \in K} E_k: K \subseteq \mathbb{Z}^+\}\).

    a.证明\(\mathcal{S}\)是\(\sigma-\)代数. \quad b.证明\(f: X \to \mathbb{R}\)是\(\mathcal{S}\)可测函数等价于\(\forall k \in \mathbb{Z}^+, \left.f\right|_{E_k}\)是常数.
\end{problem}

\begin{proof}[证明a]
    令\(K=\varnothing\),则\(\varnothing \in \mathcal{S}\).考虑\(\bigcup_{k \in K} E_k \in \mathcal{S}\),则\(X \setminus \bigcup_{k \in K} E_k=\bigcup_{k \in \mathbb{Z}^+ \setminus K} E_k \in \mathcal{S}\).

    若\(\bigcup_{k \in K_1} E_k, \bigcup_{k \in K_2} E_k, \dots \in \mathcal{S}\),那么\(\bigcup_{n=1}^\infty \bigcup_{k \in K_n} E_k=\bigcup_{k \in \bigcup_{n=1}^\infty K_n} E_k \in \mathcal{S}\).
\end{proof}

\begin{proof}[证明b]
    若\(\left.f\right|_{E_k}=c_k\),那么\(f(X)=\{c_1, c_2, \dots\}\),于是\(\forall k \in \mathbb{Z}^+, f^{-1}(\{c_k\})=E_k \in \mathcal{S}\).

    若\(f\)是\(\mathcal{S}\)可测函数,那么\(\forall B \subseteq \mathbb{R}, k \in \mathbb{N}\)必有\(f^{-1}(B) \cap E_k=\varnothing\)或\(E_k \subseteq f^{-1}(B)\).

    若\(\left.f\right|_{E_k}\)不是常值函数,即存在\(k_1, k_2 \in E_k, f(k_1)=c_1 \ne c_2=f(k_2)\).

    那么\(f^{-1}(\{c_1\}) \cap E_k \ne \varnothing\)且\(E_k \nsubseteq f^{-1}(\{c_1\})\),因此\(f\)不是\(\mathcal{S}\)可测函数,矛盾.
\end{proof}

\begin{problem}[16]\label{2.B.16}
    设\(\mathcal{S}\)是\(X\)上的\(\sigma-\)代数且\(A \subseteq X\).令\(\mathcal{S}_A=\{E \in \mathcal{S}: A \subseteq E \text{或} A \cap E=\varnothing\}\).

    a.证明\(\mathcal{S}_A\)是\(\sigma\)代数. \quad b.证明\(f: X \to \mathbb{R}\)是\(\mathcal{S}_A\)可测函数等价于\(f\)是\(\mathcal{S}\)可测函数且\(\left.f\right|_A\)是常函数.
\end{problem}

\begin{proof}[证明a]
    若\(E \in \mathcal{S}\),则\(E \cap A=\varnothing \Longleftrightarrow A \subseteq X \setminus E\).因此\(E \in \mathcal{S}_A \Longleftrightarrow X \setminus E \in \mathcal{S}_A\).
    
    若\(E_1, E_2, \dots \in \mathcal{S}_A\),那么若\(\exists k \in \mathbb{N}, A \subseteq E_k\),则\(A \subseteq \bigcup_{k=1}^\infty E_k, \bigcup_{k=1}^\infty E_k \in \mathcal{S}_A\).

    若\(\forall k \in \mathbb{N}, E_k \nsubseteq A\),那么\(E_k \cap A=\varnothing, \bigcup_{k=1}^\infty E_k \cap A=\varnothing\),因此\(\bigcup_{k=1}^\infty E_k \in \mathcal{S}_A\).

    因此二者之一总是成立,即\(\bigcup_{k=1}^\infty E_k \in \mathcal{S}_A\),结合\(\varnothing \in \mathcal{S}_A\),从而\(\mathcal{S}_A\)是\(\sigma\)代数.
\end{proof}

\begin{proof}[证明b]
    若\(f\)是\(\mathcal{S}-\)可测函数,那么\(\exists B \subseteq \mathbb{R}, f^{-1}(B) \notin \mathcal{S}_A \subseteq \mathcal{S}\),从而\(f\)是\(\mathcal{S}-\)可测函数.

    若\(\left.f\right|_A\)不是常函数,即存在\(k_1, k_2 \in E_k, f(k_1)=c_1 \ne c_2=f(k_2)\).

    那么\(f^{-1}(\{c_1\}) \cap A \ne \varnothing\)且\(A \nsubseteq f^{-1}(\{c_1\})\),因此\(f^{-1}(\{c_1\}) \notin \mathcal{S}_A\),从而\(f\)不是\(\mathcal{S}-\)可测函数.

    {\kaishu 因此只有\(f\)是\(\mathcal{S}-\)可测函数且\(\left.f\right|_A\)是常函数,\(f\)才为\(\mathcal{S}_A-\)可测函数}.

    设\(\left.f\right|_A=c\).考虑\(\forall B \subseteq \mathbb{R}, f^{-1}(B) \in \mathcal{S}\).若\(c \in B\),则\(A \subseteq f^{-1}(B)\).

    若\(c \notin B\),则\(A \cap B=\varnothing\),从而\(f^{-1}(B) \in \mathcal{S}_A\),则\(f\)是\(\mathcal{S}_A-\)可测函数.
\end{proof}

\begin{comment}
    \begin{problem}[17]\label{2.B.17}
        设\(f: X \to \mathbb{R}\)满足\(D=\{x \in X: f(x) \text{在} x \text{处不连续}\}\)是可数集.

        其中\(X \subseteq \mathbb{R}\)是borel集.证明:\(f\)是borel可测函数.
    \end{problem}

    \begin{proof}
        \(f^{-1}((a,\infty))=(\left.f\right|_{D^c})^{-1}((a,\infty)) \cup (\left.f\right|_{D})^{-1}((a,\infty))\),则\(\left.f\right|_{D^c}\)是连续函数.
        
        于是存在开集\(G\)使得\(A=(\left.f\right|_{D^c})^{-1}((a,\infty))=G \cap D^c\),从而\(A\)是\textit{borel}集.
        
        \(B=(\left.f\right|_{D})^{-1}((a,\infty))\)是可数集,也是\textit{borel}集,那么\(A \cup B\)是\textit{borel}集.
    \end{proof}
\end{comment}


\begin{problem}[18]\label{2.B.18}
    设\(f: \mathbb{R} \to \mathbb{R}\)是可微函数,证明\(f'\)是borel可测函数.
\end{problem}

\begin{proof}
    构造\(g_n(x)=\dfrac{f(x+1/n)-f(x)}{1/n}\),那么\(f'(x)=\lim_{n \to \infty}g_n(x)\).

    \(f\)作为可微函数一定是连续的,于是\(g_n=n(f(x+1/n)-f(x))\)是\textit{borel}可测的.

    根据\cref{2.48},\(f'\)是\textit{borel}可测的.
\end{proof}

\newpage

\begin{problem}[19]\label{2.B.19}
    设\(X \subseteq \mathbb{R}\)是非空集合,\(\mathcal{S}\)是\(X\)上的可数-余可数\(\sigma-\)代数.

    证明:\(f: X \to \mathbb{R}\)是\(\mathcal{S}-\)可测函数等价于存在常数\(c\)使得\(X \setminus f^{-1}(\{c\})\)是可数集.
\end{problem}

\begin{proof}
    若\(X\)是可数的,那么任何\(f: X \to \mathbb{R}\)都是\(\mathcal{S}-\)可测函数且\(X \setminus f^{-1}(\{c\})\)可数.

    假设\(X\)不可数.若存在常数\(c\)使得\(X \setminus f^{-1}(\{c\})\)是可数集,考虑\(\forall B \in \sigma(\mathcal{B})\).

    若\(c \in B\),那么\(f^{-1}(B)\)是余可数集,若\(c \notin B\),那么\(f^{-1}(B)\)是可数集,从而\(B \in \mathcal{S}\).

    设\(f\)是\(\mathcal{S}-\)可测函数.若\(f^{-1}((a_1, \infty))\)是可数集,则\(\forall a_2 \geq a_1, f^{-1}((a_2, \infty))\)是可数集;

    若\(f^{-1}((a_1, \infty))\)是余可数集,则\(\forall a_2 \leq a_1, f^{-1}((a_2, \infty))\)是余可数集.

    设\(S=\{a \in \mathbb{R}: f^{-1}((a, \infty)) \text{是可数集}\}\),令\(\inf S=c\),下证\(c \in \mathbb{R}\).

    若\(c=+\infty\),那么\(\forall a \in \mathbb{R}, f^{-1}((a, \infty))\)是不可数集,即\(\forall a \in \mathbb{R}, f^{-1}((-\infty, a])\)可数.

    于是\(f^{-1}((-\infty, a)) \subseteq f^{-1}((-\infty, a])\)是可数集且\(X=\bigcup_{n=1}^\infty f^{-1}((-\infty, n))\)可数,矛盾.

    若\(c=-\infty\),那么\(\forall a \in \mathbb{R}, f^{-1}((a, \infty))\)是可数集,于是\(X=\bigcup_{n=1}^\infty f^{-1}((n, \infty))\)可数,矛盾.

    于是\(c \in \mathbb{R}\).设\(\mathbb{Q}_1=\mathbb{Q} \cap (-\infty, c), \mathbb{Q}_2=\mathbb{Q} \cap (c, \infty)\),则\(\mathbb{Q}_1, \mathbb{Q}_2\)均可数.
    
    令\(C_1=\bigcup_{q \in \mathbb{Q}_1} f^{-1}((-\infty, q)), C_2=\bigcup_{q \in \mathbb{Q}_2} f^{-1}((q, \infty))\),则\(C_1 \cup C_2\)是可数集.
   
    注意到\(X \setminus f^{-1}(\{c\})=C_1 \cup C_2\),因此\(\inf S=c\)即为所求.
\end{proof}

\begin{comment}
    \begin{problem}[20]\label{2.B.20}
        设\((X, \mathcal{S})\)是测度空间,\(f,g: X \to \mathbb{R}\)都是\(\mathcal{S}-\)可测函数.

        \(\forall x \in X, f(x)>0\).求证:\(f^g\)也是\(\mathcal{S}-\)可测函数.
    \end{problem}

    \begin{proof}
        \(f^g=e^{g \ln f}\),根据\cref{2.33}显然\(\ln f, g \ln f, e^{g \ln f}\)都是\(\mathcal{S}-\)可测函数.
    \end{proof}
\end{comment}

\begin{comment}
    \begin{problem}[22]\label{2.B.22}
        设\(X \subseteq \mathbb{R}, f: X \to \mathbb{R}\)是增函数.证明:除可数点外\(f\)在\(X\)上连续.
    \end{problem}

    \begin{proof}
        由于\(f\)是增函数,故对于不连续点\(\forall c_1<c_2, f(c_1^-)<f(c_1^+) \leq f(c_2^-)<f(c_2^+)\).

        若\(c_1 \ne c_2\),那么\((f(c_1^-), f(c_1^+)) \cap (f(c_2^-), f(c_2^+))=\varnothing\).令\(D\)是所有不连续点的集合.

        {\kaishu 由于各不连续点的变差区间是互不相交的且含有至少一个有理数,故\(D\)是可数的.}
    \end{proof}
\end{comment}

\begin{comment}
    \begin{problem}[23]\label{2.B.23}
        设\(f: \mathbb{R} \to \mathbb{R}\)是严格增函数,证明\(f^{-1}: f(\mathbb{R}) \to \mathbb{R}\)是连续函数.
    \end{problem}

    \begin{proof}
        \(\forall y_0 \in f(\mathbb{R}), \exists x_0 \in \mathbb{R}, f(x_0)=y_0\).令\(\delta_1=f(x_0+\varepsilon)-y_0, \delta_2=y_0-f(x_0-\varepsilon)\).

        取\(\delta=\min\{\delta_1, \delta_2\}\).从而\(\forall \abs*{y-y_0}<\delta, y<y_0+\delta \leq f(x+\varepsilon), y>y_0-\delta \geq f(x-\varepsilon)\).

        得\(y \in (f(x_0-\varepsilon), f(x_0+\varepsilon))\),由单调性\(f^{-1}(y) \in (x_0-\varepsilon, x_0+\varepsilon), \abs*{f^{-1}(y)-f^{-1}(y_0)}<\varepsilon\).

        因此\(\forall \varepsilon>0, \exists \delta=\min\{\delta_1, \delta_2\}, \abs*{f^{-1}(y)-f^{-1}(y_0)}<\varepsilon\).
    \end{proof}
\end{comment}

\begin{comment}
    \begin{problem}[25]\label{2.B.25}
        设\(X \subseteq \mathbb{R}, f: X \to \mathbb{R}\)是增函数.
        
        证明:存在严格增函数序列\(f_1, f_2, \dots: X \to \mathbb{R}\)使得\(\lim_{n \to \infty}f_n(x)=f(x)\).
    \end{problem}

    \begin{proof}
        令\(f_k(x)=f(x)+x/n\),易证\(f_k'(x)=f'(x)+1/n>0\)且\(\lim_{n \to \infty}f_n(x)=f(x)\).
    \end{proof}
\end{comment}

\begin{problem}[26]\label{2.B.26}
    设\(X \subseteq \mathbb{R}, f: X \to \mathbb{R}\)是有界增函数.
    
    证明:存在函数\(g: \mathbb{R} \to \mathbb{R}\)使得\(\forall x \in X, g(x)=f(x)\).
\end{problem}

\begin{proof}
    定义\(g: \mathbb{R} \to \mathbb{R}\)为
    \begin{align*}
        g(x)=
        \begin{cases}
            \sup \{f(t): t \in X, t \leq x\}, t \in (-\infty, x) \cap X \\
            \inf \{f(t): t \in X\}, t \notin (-\infty, x) \cap X
        \end{cases}
    \end{align*}
    由于\(f\)是有界增函数,因此\(\forall x \in X, g(x)=\sup \{f(t): t \in X, t \leq x\}=f(x)\).
\end{proof}

\begin{problem}[27]\label{2.B.27}
    设\((X, \mathcal{S})\)是测度空间,证明:存在\(f: X \to [-\infty, \infty]\),
    
    满足\(\forall a \in \mathbb{R}, f^{-1}((a, \infty)) \in \mathcal{S}\)但是\(f\)不是\(\mathcal{S}-\)可测函数.
\end{problem}

\begin{proof}
    {\kaishu 本构造在函数只能取到一端无穷时失效.} 设\(V\)是\textit{Vitali}集.
    \begin{align*}
        f(x)=
        \begin{cases}
            +\infty, x \in V \\
            -\infty, x \in X \setminus V 
        \end{cases}
    \end{align*}
    则\(\forall a \in \mathbb{R}, f^{-1}((a,\infty))=\varnothing \in \mathcal{S}\),但\(f^{-1}(\{\infty\})=V \notin \mathcal{S}\).
\end{proof}

\newpage

\begin{problem}[28]\label{2.B.28}
    设\(X \subseteq \mathbb{R}, f: X \to \mathbb{R}\)是borel可测函数.证明:\(g: \mathbb{R} \to \mathbb{R}\)
    \begin{align*}
        g(x)=
        \begin{cases}
            f(x), x \in X \\
            0, x \in X^c
        \end{cases}
    \end{align*}
    是borel可测函数.
\end{problem}

\begin{proof}
    \(f^{-1}(\mathbb{R})=X \in \sigma(\mathcal{B})\).因此\(\forall a \geq 0, g^{-1}((a,\infty))=f^{-1}((a,\infty)) \in \sigma(\mathcal{B})\).

    同时\(\forall a<0, g^{-1}((a,\infty))=f^{-1}((a,\infty)) \cup X^c \in \sigma(\mathcal{B})\).因此\(\forall a \in \mathbb{R}, g^{-1}((a,\infty)) \in \sigma(\mathcal{B})\).
\end{proof}

\begin{problem}[29]\label{2.B.29}
    设\((X, \mathcal{S})\)是测度空间,\(\{f_t\}_{t \in \mathbb{R}}\)是\(\mathcal{S}-\)可测函数族,其中\(f_t: X \to [0,1]\).

    定义\(f: X \to [0,1]\)为\(f(x)=\sup \{f_t(x): t \in \mathbb{R}\}\).证明:存在\(\{f_t\}_{t \in \mathbb{R}}\)使得\(f\)不是\(\mathcal{S}-\)可测函数.
\end{problem}

\begin{proof}
    令\textit{Vitali}集为\(V\).定义\(\forall t \in V, f_t=\chi_{\{t\}}\),于是\(\chi_{\{t\}}\)是\(\mathcal{S}-\)可测函数.

    然而\(f=\sup \{f_t(x): t \in \mathbb{R}\}=\chi_V\)不是\(\mathcal{S}-\)可测函数.
\end{proof}

\begin{problem}[30]\label{2.B.30}
    证明\(\forall x \in \mathbb{R}, \lim_{j \to \infty}(\lim_{k \to \infty} (\cos(j! \pi x))^{2k})=\chi_{\mathbb{Q}}(x)\).
\end{problem}

\begin{proof}
    若\(x\)是有理数,则存在互质的\(p,q \in \mathbb{Z}, x=p/q\).\(\forall j \geq q, j! x=p(j!/q) \in \mathbb{Z}\).

    从而\(\cos(j! \pi x)=\pm 1, \lim_{j \to \infty}(\cos(j! \pi x))^2=1\).

    若\(x\)是无理数,那么\(\forall j \in \mathbb{N}, j! x \notin \mathbb{Z}\),从而\(\abs*{\cos(j! \pi x)}<1, \lim_{k \to \infty}(\cos(j! \pi x))^{2k}=0\).
\end{proof}

\begin{comment}
    {\newpage \kaishu
        关于\cref{2.B.25}有一个在可数区间上可行的废案.
        
        为严格增区间和常值区间依次标号,索引为\(k\).\(n\)是函数列的索引号,令\(\delta_{n,k}=1/(n+k)\).

        在严格增区间\((a_k, b_k)\)内,定义函数\(f_n(x)=f(x)-\delta_{n,k}\).

        在常值区间\((a_l, b_l)\)内,则定义一次函数\(f_n(x)\)经过\(f_n(a_l)=f(a_l)-\delta_{l,k}, f_n(b_l)=f(b_l)-\delta_{l,k+1}\).

        容易验证\(\forall n \in \mathbb{N}, f_n\)是严格增函数且\(f_1, f_2, \dots\)逐点收敛于\(f\).
    }
\end{comment}

