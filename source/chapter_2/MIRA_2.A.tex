\section{2.A Outer Measure on R}

\begin{problem}[1]\label{2.A.1}
    证明:对于任意的\(A,B \subseteq \mathbb{R}\)满足\(\abs*{B}=0\),\(\abs*{A \cup B}=\abs*{A}\).
\end{problem}

\begin{proof}
    由外测度的次可加性,有\(\abs*{A \cup B} \leq \abs*{A}+\abs*{B}=\abs*{A}\).

    由外测度的单调性,有\(A \cup B \supseteq A, \abs*{A \cup B} \geq \abs*{A}\).于是\(\abs*{A \cup B}=\abs*{A}\).
\end{proof}

\begin{problem}[2]\label{2.A.2}
    设\(A \subseteq \mathbb{R}, t \in \mathbb{R}\).令\(tA=\{ta:a \in A\}\).求证:\(\abs*{tA}=\abs*{t}\abs*{A}\).
\end{problem}

\begin{proof}
    设\(\{I_k\}\)是\(A\)的开覆盖序列,那么\(\{tI_k\}\)是\(tA\)的开覆盖序列.

    对于任意的开区间\(I\),有\(\abs*{tI}=\abs*{t}\abs*{I}\),若\(t \ne 0\),则\(\abs*{tA} \leq \sum_{k=1}^\infty \ell(tI_k)=\abs*{t} \sum_{k=1}^\infty \ell(I_k)\).

    取下确界,于是\(\abs*{tA} \leq \abs*{t}\abs*{A}\).同理\(\abs*{A}=\abs*{(1/t)(tA)} \leq (1/t)\abs*{tA}\),那么\(\abs*{tA}=\abs*{t}\abs*{A}\).

    当\(t=0\)时,由于\(0 \cdot \infty=0\),故\(\abs*{tA}=\abs*{t}\abs*{A}=0\),证毕.
\end{proof}

\begin{problem}[3]\label{2.A.3}
    证明:对于任意的\(A,B \subseteq \mathbb{R}\)满足\(\abs*{A}<\infty\),\(\abs*{B \setminus A} \geq \abs*{B}-\abs*{A}\).
\end{problem}

\begin{proof}
    由于\(B=(B \cap A) \cup (B \setminus A)\),根据外测度的次可加性,
    
    有\(\abs*{B}=\abs*{(B \cap A) \cup (B \setminus A)} \leq \abs*{B \setminus A}+\abs*{B \cap A} \leq \abs*{B \setminus A}+\abs*{A}\).
\end{proof}

\begin{comment}
\begin{problem}[5]\label{2.A.5}
    设\(\mathcal{A}\)是\(\mathbb{R}\)的闭子集集合,满足\(\bigcap_{F \in \mathcal{A}}F=\varnothing\).

    证明:若\(\mathcal{A}\)包含至少一个紧集,那么存在\(F_1, \dots, F_n \in \mathcal{A}\)使得\(\bigcap_{i=1}^n F_i=\varnothing\).
\end{problem}

\begin{proof}
    设紧集为\(K\),那么\(K \cap \bigcap_{F \in \mathcal{A}}F=\varnothing\),也即\(K \subset (\bigcap_{F \in \mathcal{A}}F)^c=\bigcup_{F \in \mathcal{A}}F^c\).

    于是\(\bigcup_{F \in \mathcal{A}}F^c\)构成了\(K\)的开覆盖序列,那么\(K\)拥有一组有限子覆盖\(\bigcup_{i=1}^n F_i^c\).

    得到\(K \subset \bigcup_{i=1}^n F_i^c=(\bigcap_{i=1}^n F_i)^c\),即\(K \cap \bigcap_{i=1}^n F_i=\varnothing\).
\end{proof}
\end{comment}

\begin{comment}
    \begin{problem}[7]\label{2.A.7}
        证明:若\(I_1, \dots, I_n\)是互不相交的开区间,那么\(\abs*{\bigcup_{k=1}^n I_k}=\sum_{k=1}^n \ell(I_k)\).
    \end{problem}

    \begin{proof}
        对于\(n=1\),情况显然成立.现设\(\abs*{\bigcup_{k=1}^{n-1} I_k}=\sum_{k=1}^{n-1} \ell(I_k)\).

        设\(I_n=(a_n,b_n)\).若\(\ell(I_n)=\infty\),那么\(\abs*{\bigcup_{k=1}^n I_k}=\sum_{k=1}^n \ell(I_k)=\infty\),故设\(\ell(I_n)<\infty\).

        由于\(\bigcup_{k=1}^n I_k=\bigcup_{k=1}^{n-1} I_k \cup I_n\),故\(\abs*{\bigcup_{k=1}^n I_k} \leq \abs*{\bigcup_{k=1}^{n-1} I_k}+\ell(I_n)=\sum_{k=1}^n \ell(I_k)\).

        设\(\bigcup_{k=1}^n I_k\)的开覆盖为\(\bigcup_{i=1}^\infty G_i\).令\(G_i^1=G_i \cap (-\infty,a_n), G_i^2=G_i \cap (a_n,b_n), G_i^3=G_i \cap (b_n,\infty)\).

        所以\(G_1^1, G_1^3, G_2^1, G_2^3, \dots\)和\(G_1^2, G_2^2, \dots\)分别是\(\bigcup_{k=1}^{n-1} I_k\)和\(I_n\)的开覆盖序列.
        \begin{align*}
            \sum_{i=1}^\infty \ell(G_i)=\sum_{i=1}^\infty (\ell(G_i^1)+\ell(G_i^3))+\sum_{i=1}^\infty \ell(G_i^2)
            \geq \abs*{\bigcup_{k=1}^{n-1} I_k}+\abs*{I_n}
        \end{align*}
        取下确界得\(\sum_{k=1}^n \ell(I_k) \geq \abs*{\bigcup_{k=1}^{n-1} I_k}+\ell(I_n)\),即\(\abs*{\bigcup_{k=1}^n I_k}=\sum_{k=1}^n \ell(I_k)\).
    \end{proof}
\end{comment}

\begin{comment}
    \begin{problem}
        证明:\(\abs*{(a,b) \cup (c,d)}=(b-a)+(d-c)\)与\((a,b) \cap (c,d)=\varnothing\)等价.
    \end{problem}

    \begin{proof}
        不失一般性,设\(a \leq c\),那么\((a,b) \cap (c,d)=\varnothing\)等价于\(b \leq c\).

        令\(A=[b,c], B=(a,d)\),则\(\abs*{(a,b) \cup (c,d)}=\abs*{B \setminus A} \geq \abs*{B}-\abs*{A}=(d-a)+(b-c)\).

        结合\(\abs*{(a,b) \cup (c,d)} \leq (b-a)+(d-c)\),得到\(\abs*{(a,b) \cup (c,d)}=(b-a)+(d-c)\).

        另一方面,若\((a,b) \cap (c,d) \ne \varnothing\),也即\(b>c\),则\((a,b) \cup (c,d)=(a, \max(b,d))\).

        若\(b \geq d\),则\((b-a)+(d-c)=\abs*{(a,b) \cup (c,d)}=\abs*{(a,b)}=b-a\),于是\(\abs*{(c,d)}=0\),矛盾;

        若\(b \leq d\),则\((b-a)+(d-c)=\abs*{(a,b) \cup (c,d)}=\abs*{(a,d)}=d-a\),于是\(\abs*{(c,b)}=0\),矛盾.

        所以\((a,b) \cap (c,d) \ne \varnothing\)的假设不成立,必要性得证.
    \end{proof}
\end{comment}

\begin{comment}
\begin{problem}[8]\label{2.A.8}
    证明:若\(A \subseteq \mathbb{R}, t>0\),那么\(\abs*{A}=\abs*{A \cap (-t,t)}+\abs*{A \cap (-t,t)^c}\).
\end{problem}

\begin{proof}
    由于\(A=(A \cap (-t,t)) \cup (A \cap (-t,t)^c)\),因而有\(\abs*{A} \leq \abs*{A \cap (-t,t)}+\abs*{A \cap (-t,t)^c}\).

    设\(A\)的开覆盖为\(\bigcup_{k=1}^\infty I_k\).令\(I_k^1=I_k \cap (-\infty,-t), I_k^2=I_k \cap (-t,t), I_k^3=I_k \cap (t,\infty)\).

    所以\(I_1^1, I_1^3, I_2^1, I_2^3, \dots\)和\(I_1^2, I_2^2, \dots\)分别是\(A \cap (-t,t)^c\)和\(A \cap (-t,t)\)的开覆盖序列.
    \begin{align*}
        \sum_{k=1}^\infty \ell(I_k)=\sum_{k=1}^\infty (\ell(I_k^1)+\ell(I_k^3))+\sum_{k=1}^\infty \ell(I_k^2)
        \geq \abs*{A \cap (-t,t)}+\abs*{A \cap (-t,t)^c}
    \end{align*}
    取下确界得\(\abs*{A} \geq \abs*{A \cap (-t,t)}+\abs*{A \cap (-t,t)^c}\),得\(\abs*{A}=\abs*{A \cap (-t,t)}+\abs*{A \cap (-t,t)^c}\).
\end{proof}
\end{comment}

\begin{problem}[9]\label{2.A.9}
    证明:\(\abs*{A}=\lim_{t \to \infty}\abs*{A \cap (-t,t)}\).
\end{problem}

\begin{proof}
    令\(f: [0, \infty] \to [0, \infty]\)为\(f(t)=\abs*{A \cap (-t,t)}\),由外测度的单调性得\(f\)是增函数.

    同时\(f(t)=\abs*{A}-\abs*{A \cap (-t,t)^c} \leq \abs*{A}\),故\(f\)有上界,下证\(\sup f=\abs*{A}\).

    由\(f\)的单调性,\(\lim_{t \to \infty} f(t)=\sup f\),同时\(\sup f=\abs*{\bigcup_{t=1}^\infty A \cap (-t,t)}=\abs*{A}\).

    因此\(\abs*{A}=\sup f=\lim_{t \to \infty}\abs*{A \cap (-t,t)}\),证毕.
\end{proof}

\newpage

\begin{problem}[11]\label{2.A.11}
    设开集\(G=\bigcup_{k=1}^\infty I_k\),其中\(\{I_k\}_{k \in \mathbb{Z}^+}\)是互不相交的开区间序列,
    
    则有\(\abs*{G}=\sum_{k=1}^\infty \ell(I_k)\).
\end{problem}

\begin{proof}
    由于\(\{I_k\}_{k \in \mathbb{Z}^+}\)本身是\(G\)的可数开覆盖,故\(\sum_{k=1}^\infty \ell(I_k) \geq \abs*{G}\).

    取\(G\)的任意开区间覆盖\(\mathcal{S}=\{J_j\}_{j \in \mathbb{Z}^+}\),定义\(J_i \sim J_j\)当且仅当\(\exists J_1, \dots, J_m \in \mathcal{S}\),

    使得\(J_p=J_1, J_q=J_m, \forall p=1, \dots, m-1, J_p \cap J_{p+1} \ne \varnothing\).

    定义等价类\(\mathcal{U}_j=\{J_i \in \mathcal{S}: J_i \sim J_j\}\),有\([J_j]=\bigcup_{J_i \in \mathcal{U}_j} J_i, \ell([J_j]) \leq \sum_{J_i \in \mathcal{U}_j} \ell(J_i)\).

    {\kaishu 定义\(\mathcal{A}\)是所有等价类的任一代表元组成的集合,令\([J_\alpha], [J_\beta]\)为其元素.}
    
    因此\(\bigcup_{j=1}^\infty J_j=\bigcup_{[J_\alpha] \in \mathcal{A}} [J_\alpha]\)且\(\sum_{[J_\alpha] \in \mathcal{A}} \ell([J_\alpha]) \leq \sum_{j=1}^\infty \ell(J_j)\).若\([J_\alpha] \ne [J_\beta]\),则\([J_\alpha] \cap [J_\beta]=\varnothing\).
    
    因而\(\forall k \in \mathbb{Z}^+\)不可能同时有\(I_k \cap [J_\alpha] \ne \varnothing, I_k \cap [J_\beta] \ne \varnothing, \alpha \ne \beta\),即\(\exists [J_\alpha] \in \mathcal{A}, I_k \subseteq [J_\alpha]\).

    定义\(\mathcal{J}_\alpha=\{I_k \in \mathbb{Z}^+: I_k \subseteq [J_\alpha]\}\),从而\(\bigcup_{k=1}^\infty I_k=\bigcup_{[J_\alpha] \in \mathcal{A}} \bigcup_{I_k \in [\mathcal{J}_\alpha]} I_k \subseteq \bigcup_{[J_\alpha] \in \mathcal{A}} [J_\alpha]\).

    因此\(\sum_{k=1}^\infty \ell(I_k) \leq \sum_{[J_\alpha] \in \mathcal{A}} \ell([J_\alpha]) \leq \sum_{j=1}^\infty \ell(J_j)\),故\(\sum_{k=1}^\infty \ell(I_k) \leq \abs*{G}, \abs*{G}=\sum_{k=1}^\infty \ell(I_k)\).

    {\kaishu 证明分为两步.首先将任意选取的开覆盖按照最大连通分支合并,按照可数次可加性得到第一个不等式;随后由于\(G\)的任意开区间均包含于这些开覆盖,于是它们只能属于其中一个分支,否则开区间将是分离集.将\(G\)的开区间分别归入所有等价类,得到第二个不等式.}
\end{proof}

\begin{lemma}\label{2.A.11.1}
    设\(G \subseteq \mathbb{R}\)是开集,则\(\forall E \in 2^\mathbb{R}, \abs*{E}=\inf\{\abs*{G}: G \supseteq E\}\).
\end{lemma}

\begin{proof}
    考虑\(E\)的任意开区间覆盖\(\{I_k\}_{k \in \mathbb{Z}^+}\),令\(G=\bigcup_{k=1}^\infty I_k\),则\(G \supseteq E\)是开集.定义
    \begin{align*}
        \mathcal{S}_1=\left\{\sum_{k=1}^\infty \ell(I_k): \bigcup_{k=1}^\infty I_k \supseteq E\right\},
        \mathcal{S}_2=\left\{\abs*{G}: G \supseteq E\right\}.
    \end{align*}
    由于\(\forall \{I_k\}_{k \in \mathbb{Z}^+}, \abs*{G} \leq \sum_{k=1}^\infty \ell(I_k)\),故\(\forall s_1 \in \mathcal{S}_1, \exists s_2 \in \mathcal{S}_2, s_2 \leq s_1\),即\(\inf \mathcal{S}_2 \leq \inf \mathcal{S}_1\).

    同时由于\(\forall G \supseteq E, G=\bigcup_{k=1}^\infty I_k \supseteq E, \mathcal{S}_2 \ni \abs*{G}=\sum_{k=1}^\infty \ell(I_k) \in \mathcal{S}_1\).

    因此\(\mathcal{S}_2 \subseteq \mathcal{S}_1, \inf \mathcal{S}_2 \geq \inf \mathcal{S}_1\),即\(\inf \mathcal{S}_1=\inf \mathcal{S}_2\),得到\(\abs*{E}=\inf\{\abs*{G}: G \supseteq E\}\).

    {\kaishu 一方面,对于任意一个开覆盖,按最大联通分支合并后总能得到一个测度更小的开集,从而前者的下确界大于后者;另一方面后者作为一种特殊的开覆盖是前者的子集,故后者的下确界大于前者.它表明开集是一种最有效率的覆盖方式.}
\end{proof}

\begin{comment}
    \begin{problem}[10]\label{2.A.10}
        证明:\(\abs*{[0,1] \setminus \mathbb{Q}}=1\).
    \end{problem}

    \begin{proof}
        \(\abs*{[0,1] \setminus \mathbb{Q}} \geq \abs*{[0,1]}-\abs*{\mathbb{Q}}=\abs*{[0,1]}=1\)且\(\abs*{[0,1] \setminus \mathbb{Q}} \leq \abs*{[0,1]}=1\),
        得\(\abs*{[0,1] \setminus \mathbb{Q}}=1\).
    \end{proof}
\end{comment}

\begin{comment}
\begin{problem}[11]\label{2.A.11}
    证明:若\(I_1, I_2, \dots\)是互不相交的开区间序列,那么\(\abs*{\bigcup_{k=1}^\infty I_k}=\sum_{k=1}^\infty \ell(I_k)\).
\end{problem}

\begin{proof}
    若\(\abs*{\bigcup_{k=1}^\infty I_k}=\infty\),则\(\abs*{\bigcup_{k=1}^\infty I_k}=\sum_{k=1}^\infty \ell(I_k)=\infty\),故设\(\abs*{\bigcup_{k=1}^\infty I_k}<\infty\).
    
    据有限可加性,得\(\forall n \in \mathbb{N}, \abs*{\bigcup_{k=1}^n I_k}=\sum_{k=1}^n \ell(I_k)\).定义\(c_n=\sum_{k=1}^n \ell(I_k)=\abs*{\bigcup_{k=1}^n I_k}\).
    
    由于\(c_n=\abs*{\bigcup_{k=1}^n I_k} \leq \abs*{\bigcup_{k=1}^\infty I_k}\),故\(\{c_n\}\)单调有界收敛于\(c\).因此\(c=\sum_{k=1}^\infty \ell(I_k) \leq \abs*{\bigcup_{k=1}^\infty I_k}\).
    
    结合\(\abs*{\bigcup_{k=1}^\infty I_k} \leq \sum_{k=1}^\infty \ell(I_k)\),有\(\abs*{\bigcup_{k=1}^\infty I_k}=\sum_{k=1}^\infty \ell(I_k)\).
\end{proof}
\end{comment}

\begin{comment}
    \begin{problem}[12]\label{2.A.12}
        设\(r_1, r_2, \dots\)是有理数序列,令\(F=(\bigcup_{k=1}^\infty (r_k-\frac{1}{2^k}, r_k+\frac{1}{2^k}))^c\).

        a.证明\(F\)是闭集. \quad b.证明包含于\(F\)的区间最多含一个元素. \quad c.证明\(\abs*{F}=\infty\).
    \end{problem}

    \begin{proof}[证明a]
        {\kaishu 开集对可数并封闭且闭集是开集的补集.}
        
        于是\(\bigcup_{k=1}^\infty (r_k-\frac{1}{2^k}, r_k+\frac{1}{2^k})\)是开集,\((\bigcup_{k=1}^\infty (r_k-\frac{1}{2^k}, r_k+\frac{1}{2^k}))^c\)是闭集.
    \end{proof}

    \begin{proof}[证明b]
        设区间\((a,b) \subset F\),则存在\(r_k \in (a,b)\),从而\((r_k-\frac{1}{2^k}, r_k+\frac{1}{2^k}) \cap F \ne \varnothing\),矛盾.
    \end{proof}
\end{comment}