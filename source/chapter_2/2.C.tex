\section{2.C Measures and Their Properties}

\begin{problem}[1]\label{2.C.1}
    证明:不存在测度空间\((X, \mathcal{S}, \mu)\)使得\(\{\mu(E): E \in \mathcal{S}\}=[0,1)\).
\end{problem}

\begin{proof}
    显然\(\mu(X)<1\),因此\(\exists E \in \mathcal{S}, \mu(E) \in (\mu(X), 1)\).而\(\forall E \in \mathcal{S}, E \subseteq X, \mu(E) \leq \mu(X)\).
\end{proof}

\begin{problem}[2]\label{2.C.2}
    设\(\mu\)是\((\mathbb{Z}^+, 2^{\mathbb{Z}^+})\)上的测度.

    证明:存在序列\(w_1, w_2, \dots \in [0,\infty]\)使得\(\forall E \subseteq \mathbb{Z}^+, \mu(E)=\sum_{k \in E} w_k\).
\end{problem}

\begin{proof}
    令\(\forall k \in \mathbb{Z}^+, w_k=\mu(\{k\})\),那么\(\mu(E)=\mu(\bigcup_{k \in E} \{k\})=\sum_{k \in E} \mu(\{k\})=\sum_{k \in E} w_k\).
\end{proof}

\begin{problem}[3]\label{2.C.3}
    构造\((\mathbb{Z}^+, 2^{\mathbb{Z}^+})\)上的测度\(\mu\)使得\(\{\mu(E): E \subseteq \mathbb{Z}^+\}=[0,1]\).
\end{problem}

\begin{proof}
    令\(\forall k \in \mathbb{Z}^+, \mu(\{k\})=1/2^k\).对于\(x \in [0,1]\),其{\kaishu 二进制展开}为\(x=(0.b_1 b_2 \dots)_2\).

    其中\(\forall k \in \mathbb{Z}^+, b_k \in \{0,1\}\),从而\(x=\sum_{k=1}^\infty b_k/2^k\).令\(A_x=\{k \in \mathbb{Z}^+: b_k=1\}\).
    
    那么\(x=\sum_{k \in A_x} 1/2^k=\sum_{k \in A_x} \mu(\{k\})\).

    {\kaishu 分母为\(2^N\)的有理数有两种展开式,所有实数都有至少一个二进制展开形式.}
\end{proof}

\begin{problem}[4]\label{2.C.4}
    构造测度空间\((X, \mathcal{S}, \mu)\)满足\(\{\mu(E): E \in \mathcal{S}\}=\{\infty\} \cup \bigcup_{k=0}^\infty [3k,3k+1]\).
\end{problem}

\begin{proof}
    设\(X=(0, \infty)\),令\(\mathcal{S}=\{\bigcup_{k \in K_1} (k-1,k) \cup K_2: K_1, K_2 \subseteq \mathbb{Z}^+\}\).

    定义测度\(\mu\)为\(\forall k \in \mathbb{Z}^+, \mu((k-1,k))=3, \mu(\{k\})=1/2^k\),则\(\mu(X)=\infty\).
    
    \(\forall k \in \mathbb{Z}^+, \forall x \in [3k,3k+1], x=\mu(\bigcup_{k=1}^\infty (k-1,k))+c\),其中\(c \in [0,1]\).

    {\kaishu 根据\cref{2.C.3},这样的\(c=\sum_{k \in A_c} 1/2^k=\sum_{k \in A_c} \mu(\{k\})\)可以被构造}.
\end{proof}

\begin{problem}[5]\label{2.C.5}
    设\((X, \mathcal{S}, \mu)\)是满足\(\mu(X)<\infty\)的测度空间.若存在\(\mathcal{A} \subseteq \mathcal{S}\),

    \(\mathcal{A}\)中的集合互不相交,且\(\forall A \in \mathcal{A}, \mu(A)>0\).证明:\(\mathcal{A}\)是可数集.
\end{problem}

\begin{proof}
    令\(\mathcal{A}_n=\{A \in \mathcal{A}: \mu(A)>\mu(X)/n\}\),从而\(\mathcal{A}=\bigcup_{n=1}^\infty \mathcal{A}_n\).

    对于任意\(n \in \mathbb{Z}^+\),\(\mathcal{A}_n\)中的元素不能超过\(n\)个,否则\(\mu(\bigcup_{A \in \mathcal{A}_n} A)>\mu(X)\).

    从而对于任意\(n \in \mathbb{Z}^+\),\(\mathcal{A}_n\)是有限集,因此\(\mathcal{A}=\bigcup_{n=1}^\infty \mathcal{A}_n\)是至多可数的.
\end{proof}

\begin{problem}[6]\label{2.C.6}
    若存在测度空间\((X, \mathcal{S}, \mu)\)满足\(\{\mu(E): E \in \mathcal{S}\}=[0,1] \cup [3,c]\),

    给出所有可能的\(c \in [3,\infty)\).
\end{problem}

\begin{proof}
    显然\(\mu(X)=c\).\(\forall E \in \mathcal{S}, \mu(E) \in [0,1] \cup [3,c], \mu(X \setminus E)=c-\mu(E) \in [0,1] \cup [3,c]\).

    只有\(c=4\)满足条件.令\(X=\mathbb{N}, \mathcal{S}=2^{\mathbb{N}}, \forall k \in \mathbb{Z}^+, \mu(k)=1/2^k, \mu(0)=3\).
\end{proof}

\newpage

\begin{problem}[7]\label{2.C.7}
    给出测度空间\((X, \mathcal{S}, \mu)\)满足\(\{\mu(E): E \in \mathcal{S}\}=[0,1] \cup [3,\infty]\).
\end{problem}

\begin{proof}
    设\(X=(0, \infty)\),令\(\mathcal{S}=\{\bigcup_{k \in K_1} (k-1,k) \cup K_2: K_1, K_2 \subseteq \mathbb{Z}^+\}\).

    定义测度\(\mu\)为\(\forall k \in \mathbb{Z}^+, \mu((k-1,k))=k+2, \mu(\{k\})=1/2^k\),则\(\mu(X)=\infty\).

    \(\forall k \geq 3, \forall x \in [k,k+1], x=\mu((k-1,k))+c\),其中\(c \in [0,1]\).

    {\kaishu 根据\cref{2.C.3},这样的\(c=\sum_{k \in A_c} 1/2^k=\sum_{k \in A_c} \mu(\{k\})\)可以被构造}.
\end{proof}

\begin{problem}[8]\label{2.C.8}
    给出测度空间\((X, \mathcal{S})\),一个集合族\(\mathcal{A} \subseteq \mathcal{S}\)和其上的两个测度\(\mu, \nu\).

    其中\(\sigma(\mathcal{A})=\mathcal{S}\)且\(\mu(X)=\nu(X)<\infty, \forall A \in \mathcal{A}, \mu(A)=\nu(A)\)但\(\mu \ne \nu\).
\end{problem}

\begin{proof}
    {\kaishu 本题的关键在于平衡\(\sigma-\)代数的生成能力和刚性之间的关系.
    
    对于一个原子\(\sigma-\)代数,\(\mathcal{A}\)必须通过可数交并生成\(\sigma-\)代数的所有原子.
    
    但是\(\mathcal{A}\)如直接包含其中所有原子,那么将因\(\sigma-\)代数的刚性导致所有集合测度唯一.
    
    因此我们需要选择\(\mathcal{A}\)为一些中间集合,从而\(\mathcal{A}\)自己通过集合运算得到\(\mathcal{S}\)的所有原子.}

    令\(X=\{1,2,3,4\}, \mathcal{S}=2^X, \mathcal{A}=\{\{1,2\}, \{1,3\}\}\),下证\(\sigma(\mathcal{A})=\mathcal{S}\).

    \(\{1\}=\{1,2\} \cap \{1,3\}, \{2\}=\{1,2\} \cap \{1,3\}^c, \{3\}=\{1,2\}^c \cap \{1,3\}, \{4\}=\{1,2\}^c \cap \{1,3\}^c\).

    于是\(\sigma(\mathcal{A})=\mathcal{S}\).令\(\mu(\{1\})=\mu(\{2\})=\mu(\{3\})=\mu(\{4\})=0.5\).

    令\(\nu(\{1\})=\nu(\{4\})=0.6, \nu(\{2\})=\nu(\{3\})=0.4\),构造完毕.
\end{proof}

\begin{problem}[10]\label{2.C.10}
    给出一个测度空间\((X, \mathcal{S}, \mu)\)和\(E_1 \supseteq E_2 \supseteq \dots \in \mathcal{S}\)但不满足上连续性,
    
    即\(\lim_{k \to \infty} \mu(E_k) \ne \bigcap_{k=1}^\infty \mu(E_k)\).
\end{problem}

\begin{proof}
    设\(X=\mathbb{N}, \mathcal{S}=2^X, \forall k \in \mathbb{N}, \mu(\{k\})=1\),令\(E_k=\{k, k+1, \dots\}\).

    设\(x \in \bigcap_{k=1}^\infty \mu(E_k)\),但\(\exists k \in \mathbb{N}, k>x\),从而\(x \notin \bigcap_{k=1}^\infty \mu(E_k), \bigcap_{k=1}^\infty \mu(E_k)=\varnothing\).

    结合\(\forall k \in \mathbb{N}, \mu(E_k)=\infty\),从而\(\infty=\lim_{k \to \infty} \mu(E_k) \ne \bigcap_{k=1}^\infty \mu(E_k)=0\).
\end{proof}

\begin{problem}[12]\label{2.C.12}
    设\(\mathcal{S}\)是\(X\)上的可数-余可数\(\sigma-\)代数,给出\((X, \mathcal{S})\)上测度的特征刻画.
\end{problem}

\begin{proof}
    {\kaishu 一个观察:令所有可数集测度为\(0\),所有余可数集测度为\(\infty\)是最保守的策略.

    如果\(X\)本身是可数集,那么它将成为一种纯原子测度,此时单点可以取有限值.
    
    因而只有可数点测度为正,而其补集的测度是一个常数,类似“宇宙微波背景辐射”.}

    若\(A=\{x \in X: \mu(\{x\})>0\}\)是不可数集且\(A_n=\{x \in X: \mu(\{x\})>1/n\}\),

    那么\(A=\bigcup_{n=1}^\infty A_n, \exists n \in \mathbb{Z}^+, A_n\)是不可数集.对于所有余可数集\(D\),\(D \cap A_n\)是不可数集.

    根据单调性\(\mu(D) \geq \mu(D \cap A_n)=\infty\),从而所有余可数集都有\(\mu(D)=\infty\).

    可数集\(C\)的测度\(\mu(C) \in [0, \infty]\).{\kaishu 这是一类病态测度.}

    若\(A=\{x \in X: \mu(\{x\})>0\}\)是可数集,那么测度\(\mu\)将被\((A, \left.\mu\right|_A, \mu(X \setminus A))\)决定.
\end{proof}